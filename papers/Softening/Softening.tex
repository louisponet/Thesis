\chapter{Mechanical softening in ferroelectric domain walls \label{ch:Softening}}
\section{Introduction}
We have so far studied the behavior of ordered materials under the assumption of perfect homogeneity.
This forms the base for understanding the more complex inhomogeneous scenarios that appear in reality because of the spontaneous symmetry breaking that occurs at the phase transition temperature $T_c$.

As usual, we can associate an \gls{OP} with the order that develops as a result of this symmetry breaking. It is thus zero above $T_c$ and nonzero below.
The energy potential for the \gls{OP} obeys the same symmetries as the parent structure, and will thus have as many degenerate minima as the dimensions of the transformation operator that describes the broken symmetry.
Each of these minima are associated with a different direction of the \gls{OP} which can be transformed into each other by applying this operator.
During a phase transition in a large system, regions that are separated in space will each individually acquire one of these equivalent values of the \gls{OP}.
Such regions are referred to as domains, with the narrow boundaries between them called domain walls inside of which the \gls{OP} is interpolated.

We already encountered examples of such OPs: in GeTe the ferroelectric polarization describes the order of the electric dipoles inside the crystal; in GdMn$_2$O$_5$ we found two primary OPs, $\bm{L}_1$ and $\bm{L}_2$, describing the antiferromagnetic order inside each Mn chain, and one slave order parameter $P_b$ that denotes the resulting polarization; in chromium we discussed a similar antiferromagnetic \gls{OP} with, in that case, an associated periodic lattice displacement.

In the case of GdMn$_2$O$_5$ we have shown that there are four combinations of the primary OPs that have the same energy, and how they can be transformed into each other by the time-reversal and inversion symmetry operators (see Fig.~\ref{fig:GdMn2O5_heatmap}). 
In chromium the antiferromagnetic \gls{OP} has two energy equivalent values, again related by time-reversal symmetry and visualized in Fig.~\ref{fig:Cr_energy_surfaces}(a).

We now turn back, to the discussion on inhomogeneity, domains, and the \gls{DW}.
There are, actually, some ways to achieve a completely homogeneous situation, even in relatively large systems.
One example is through poling, where an external field is applied during the phase transition to coax the system towards one of the orientations of the \gls{OP}.
Another is through expelling the \glspl{DW} in the sample by keeping it close to the phase transition.
This happens because at the \glspl{DW} the \gls{OP} is forced away from the minimum, making them energetically less favorable than a uniform domain.
Consequently, by keeping the system close to the phase transition for an extended period, it is sometimes possible to expel them from the material.

In the case of ferromagnetic or ferroelectric materials, the homogeneity is usually limited by the depolarizing fields that develop at the domain surface (see Fig.~\ref{fig:BTO_depolarizing_schematic}).
These favor an antiparallel orientation of the \gls{OP} in neighboring domains, making it energetically highly unfavorable, and ultimately impossible, to have a large single uniform domain when the sample size increases.
\begin{figure}[h]
	\IncludeGraphics{depolarizing_schematic}
	\caption{\label{fig:BTO_depolarizing_schematic}{\bf Depolarizing fields.} a) A single domain. b) At a domain wall between two domains with opposite polarization. In both panels, the dipoles are shown by the ovals leading to the polarization denoted by the yellow arrows. The associated depolarizing electric fields are shown by the red arrows. The orientation of these fields is such that for the single domain in panel (a) the depolarizing energy is large compared with that of the two domains in panel (b).}    
\end{figure}

It is thus clear that \glspl{DW} will be present more often than not, with their density influenced by factors such as poling, cooling rate, and sample quality \cite{Griffin2012,Shen2018,Nataf2020}.
We focus here on the \glspl{DW} that are present in ferroelectric materials.

% Together with the inherent difference between \glspl{DW} and domains, this has indeed led to a great deal of research interest, both from the fundamental and technological point of view.
Another interesting observation is that, even though \glspl{DW} occupy only a fraction of the entire volume, they can lead to the biggest contributions to macroscopic effects \cite{Schranz2012}. 
Indeed, most of the research into \glspl{DW} was originally focused on their dynamics, since switching in ferroelectric materials is mainly a result of \gls{DW} movement \cite{Merz1954,Gao2013}.
This can be understood from the following energetic argument.
When an external electric field is applied to the sample, it is more favorable for the \glspl{DW} to move in order to grow the favored domains\footnote{Domains with the polarization along the electric field.}, rather than uniformly rotating the polarization in the domain as a whole.
The latter requires overcoming the prohibitively large energy barrier that separates different orientations of the polarization \cite{Tagantsev2010}.
This relatively low energy cost associated with the \gls{DW} movement will also turn out to be one of the main contributors to the main subject of this Chapter: the mechanical softening at purely ferroelectric \glspl{DW}.

Because of the scaling down of technological devices into the nanoscale realm, the focus has more and more shifted towards the \glspl{DW} themselves as the subject of study, rather than just as being the boundaries between domains.
The physics that govern the region of the \gls{DW} is usually richer than that of the domains. One of the reasons is that, as we mentioned before, the \gls{OP} is locally forced away from the energetic minimum.
In the case of a double-well potential (or so-called Mexican hat potential), associated with many second-order phase transitions, the \gls{OP} at the center of the wall is at the top of the barrier.
The curvature of the energy potential is inverted at this top, which has been argued to make perturbing the \gls{OP} easier at the \gls{DW} \cite{Scott2012}.
The electronic band structure is also different due to the local structural differences in the wall compared with the domains, which can lead to enhanced conductivity, photovoltages, and current rectification \cite{Inoue2015,Korbel2018,Korbel2020,Huyan2019}.

In the extreme case, certain ferroelectric \glspl{DW} can even become conductive, opening the door to very promising technological applications \cite{Seidel2009}.
One example is in information storage devices, where the small size of the \glspl{DW} is beneficial for the information density.
It was shown by Seidel et al.\cite{Seidel2009} that the conducting \glspl{DW} can be created or destroyed at will by applying electric fields between opposing electrodes. This is used to write the ``data'' by changing the conductivity of the device in well-defined increments.
Probing the conductivity is then equivalent to reading the ``data''. 

While this kind of technological promise has historically driven most of the research of \glspl{DW} towards their electronic properties, interest in their mechanical characteristics has increased in more recent years.
This field is much less developed and has many outstanding questions.

Before continuing, it is useful to first define the elastic strain tensor $\varepsilon_{ij}$ as the symmetrized gradient of the deformation $u$:
\begin{equation}
\varepsilon_{ij} = \frac{1}{2}(\partial_i u_j + \partial_j u_i),
\end{equation}
where the indices $i,j$ iterate through $x, y, z$.
The vector $\bm{u}(\bm{r})$ relates points $\bm{r}$ in the reference structure to $\bm{r} + \bm{u}(\bm{r})$ in the deformed structure.
The strain arises in ferroelectric materials as a secondary \gls{OP} due to the electrostrictive coupling to the polarization $P$.
Its contribution to the free energy density can be written as:
\begin{equation}
	\label{eq:BTO_electrostriction}
	f_{q}=-q_{ijkl}\varepsilon_{ij}P_{k}P_{l},
\end{equation}
where $q_{ijkl}$ denotes the strength of the coupling, the indices $i, j, k, l$ again iterate through $x, y, z$, and Einstein summation over repeated indices is implied.
The longitudinal part, $q_{iiii} \varepsilon_{ii} (P_i)^2$, usually leads to the largest contribution to the strain, causing a tensile strain ($\varepsilon_{ii}>0$) and a stretching of the material along $P_i$~\cite{Marton2010}.

Due to the high degree of symmetry in crystals, many equivalent orientations of $P$ are possible, each with an associated strain. The \glspl{DW} separating these are equally numerous and have a distinct strain texture due to the variation of $P$ across the \gls{DW}.

For example, a 180$^\circ$ degree \gls{DW} separates domains with completely anti-parallel $P$ vectors.
Due to the square dependence on $P$ of the energy in Eq.~\eqref{eq:BTO_electrostriction}, $\varepsilon$ will be identical in both domains. Still, the different values of $P$ inside the wall versus in the domains will cause a local variation of $\varepsilon$. More precisely, the strain is diminished at the wall since $P=0$, leading to an indentation at the surface of the material.
As we will thoroughly discuss below, this local strain variation lies at the heart of the mechanical softening of the 180$^\circ$ \gls{DW}.

Other types of \gls{DW} such as 90$^\circ$ \glspl{DW} instead separate domains with different strain textures, which are referred to as twin domains, and are consequently separated by twin or ferroelastic \glspl{DW}.
See Refs.~\cite{Cao1991,Hu1998,Marton2010} for a more in-depth study of the possible domains and domain walls in ferroelectrics.

The mechanical properties of ferroelectric \glspl{DW} are not only of fundamental but also of technological interest.
It has been shown that ferroelectric-ferroelastic \glspl{DW} can be moved by applying stress \cite{Schneider2001}.
This may not be so surprising, since the different strain textures of the domains interact differently with the mechanical perturbation, causing a force imbalance on the \gls{DW} and an associated movement.
More interesting, perhaps, is the fact that the polarization can be directly influenced by mechanical means, for example through the flexoelectric effect, whose contribution to the energy density can be written as:
\begin{equation}
	\label{eq:BTO_flexoelectricity}
	f_{fl} = \frac{1}{2}\Gamma_{ijkl}(\varepsilon_{ij}\partial_kP_l-P_l\partial_k\varepsilon_{ij}),
\end{equation}
where $\Gamma_{ijkl}$ is the flexoelectric tensor.
An applied strain gradient $\partial_j\varepsilon_{kl}$ will thus appear as an effective internal electric field $E_{i} = -\frac{\partial f}{\partial P_i} = \Gamma_{ijkl}\partial_j\varepsilon_{kl}$, coupling directly to $P_i$.
While this effect is generally small (i.e. the $\Gamma_{ijkl}$ are small), the size of strain gradients scales inversely with the size of the sample, meaning that it becomes increasingly more important at the nanoscale of current state-of-the-art electronic devices.
Thus, using a tip in contact with the material to apply such a strain gradient allows one to mechanically write domain patterns and \glspl{DW} at will \cite{Lu2012,Abdollahi2015, Cordero-Edwards2017,Cordero-Edwards2019}.

In this Chapter we discuss the mechanical softening of 180$^\circ$, purely ferroelectric \glspl{DW}. They are non ferroelastic in the sense that they separate domains that have identical strain textures.
A similar softening has been previously observed and studied for ferroelastic \glspl{DW} \cite{Lee2003,Scott2012}, but purely ferroelectric \glspl{DW} have largely flown under the radar of such mechanical studies.
One reason is that the width of 180$^\circ$ \glspl{DW} is on the order of a couple of unit cells (5 - 20 \AA \cite{Zhirnov1959}), and since they do not separate domains with different strains, they were perceived too small to be detected by mechanical means.
This is because these studies usually involve the application of a tip to the surface of the sample, and the contact radius of such a tip is at least 100 unit cells.
As mentioned before, however, even 180$^\circ$ \glspl{DW} have a distinct strain texture associated with them, which extends much further than the narrow region in which the polarization reverses.
This texture can be picked up mechanically by the applied tip and ultimately allows one to make the experimental observations that will be discussed in the next Section, which were the impetus of the research presented in this Chapter.

From the practical point of view, this mechanical detection of ferroelectric \glspl{DW} means that they can be probed without applying a bias voltage, which may prove useful for conducting ferroelectrics. Moreover, since phonon spectra are determined by the stiffness of the material, the softer \glspl{DW} will harbor phonon modes with lower speeds. In turn, this may result in effects such as phonon refraction or even total internal reflection inside the walls, turning the \glspl{DW} into ``phonon waveguides''. Because heat is transported by phonons, this would allow for heat propagation through the \glspl{DW} with little dissipation. From the ability to create and destroy 180$^\circ$ \glspl{DW} by voltage, regular patterns with internal elastic contrast could be created, effectively leading to a phononic crystal.
This is interesting, since periodically poled ferroelectric crystals are already used in photonic applications \cite{Ferraro2014}.

We start in Sec.~\ref{sec:BTO_exp} by introducing the experimental techniques and observations that lie at the base of this Chapter. Afterwards, we continue with a short overview of the crystalline properties of BaTiO$_3$ in Sec.~\ref{sec:BTO_crystal}, an archetypal example of a ferroelectric perovskite.
This is followed by a detailed theoretical study in Sec.~\ref{sec:BTO_theory}, a description of our numerical simulations in Sec.~\ref{sec:BTO_methods}, and a discussion of the obtained results in Sec.~\ref{sec:BTO_results}.
Finally, we conclude by summarizing the results in Sec.~\ref{sec:BTO_conclusion}.

\section{Experimental methods and results \label{sec:BTO_exp}}
In this Section, we summarize the experiments performed by the group of Prof. Catalan that were the impetus of this research \cite{Stefani2020}.
In order to determine the generality of the \gls{DW} softening, three single crystal samples were studied: LiNbO$_3$, BaTiO$_3$, and PbTiO$_3$. It was found that all three show the same behavior, allowing us to focus on BaTiO$_3$.
The results of the experimental measurements are shown in Fig.~\ref{fig:BTO_experiment}.

\Gls{PFM} \cite{Harnagea2001} was used to determine the domain structure of the sample, displayed in panels (a) and (c).
In \gls{PFM}, the mechanical response of an electrically conductive tip in contact with the sample is measured when an alternating current voltage is applied to the sample.
Because of the piezoelectricity that is present in any ferroelectric material, this alternating current field leads to a periodic expansion and contraction of the sample, which is detected through the deflection of a laser beam that shines on the cantilever that holds the tip.
As we will discuss further below (see \ref{sec:BTO_piezoelectricity}), the piezoelectric coefficients have opposite signs for the up versus down polarized domains, meaning that the phase of the oscillation can be measured as a probe for the orientation of the polarization.
These phases are exactly what is shown in panels (a,c) of Fig.~\ref{fig:BTO_experiment}. The amplitude of the oscillation gives an idea of the absolute value of the polarization, which leads to the darker regions of the \glspl{DW} in panel (d) of the same figure.
Since the \gls{PFM} measurements are performed for an out-of-plane direction, the variations in the color signature of the in--plane polarization in the bottom right domain of panels (c-f) in Fig.~\ref{fig:BTO_experiment} is an indication of the noise of the measurements rather than an actual orientation of the polarization, since any in-plane direction should result in the same angle.
In order to accurately map the in--plane polarized domains, one would have to use \gls{PFM} in a perpendicular configuration, however, we are mostly interested in the out-of-plane polarization domains so these latter were omitted in these experiments.

Turning back to the analysis of the domains in Fig.~\ref{fig:BTO_experiment}, we can identify multiple different kinds: in the region shown in panels (a,b) domains with out-of-plane ($c$-direction) polarization are found, whereas panels (c-f) display a more complex region with both out-of-plane and in--plane ($a$-direction) polarization.
The top-left portion of these latter panels demonstrates the existence of bubble domains with out-of-plane polarization and curved 180$^\circ$ \glspl{DW}.
The bottom right portion shows an area with in-plane domains separated by straight 180$^\circ$ \glspl{DW}.
On the boundary between these two regions, we find a 90$^\circ$ twin \gls{DW}, which is different from 180$^\circ$ \glspl{DW} because it separates domains with different polarization as well as strain.

\begin{figure}
	\IncludeGraphics{experiment.png}
	\caption{\label{fig:BTO_experiment} {\bf Experimental Measurements} Two regions of the BTO sample are characterized. Panels (a,b) show a region with mostly straight \glspl{DW}. Panels (c-f) show a more complex region with bubble domains, and a twin wall separating the top-left area with out-of-plane polarization from the bottom right area with in-plane polarization (along the $a$-direction). Panels (a,c) show the \gls{PFM} phase, and panels (b,d) the \gls{CRFM} measurements demonstrating softer \gls{DW} regions (brown) and stiffer domains (purple). Panel (e) shows the out-of-plane \gls{PFM} amplitude, and panel (f) displays a schematic representation of the polarization, reconstructed from these measurements. The red vectors indicate some orientations of the polarization, and the numbers at the \glspl{DW} denote the angle between the polarization of the neighboring domains.}
\end{figure}

In order to probe the spatial variation of the stiffness of the sample, measurements based on \gls{CRFM} were performed.
This technique is based on scanning probe microscopy where an atomic force microscopy tip is placed on a cantilever and is brought into contact with the surface of the sample (see Fig.~\ref{fig:BTO_experimental_schematic} for a pictorial representation).
\begin{figure}
	\IncludeGraphics{experimental_schematic}
	\caption{\label{fig:BTO_experimental_schematic} {\bf Schematic of the \gls{CRFM} setup.} a) The tip is placed on a cantilever and brought into contact with the sample surface. The combined cantilever-surface system acts as a series of two springs, with spring constants indicated by $k_{\rm tip}$ and $k_{\rm sample}$, respectively. b) A pictorial representation of the 180$^\circ$ domain wall. The polarization direction is indicated by the arrows, and up and down domains are colored blue and red, respectively. The increased depth of indentation at the wall indicates the softening effect.}
\end{figure}
The maximal mechanical load was 20 micro Newtons, and the surface contact radius of the tip is estimated to be 7 nm.
The combined system of the cantilever and the sample then acts as a series of springs, whose resonance frequency can be measured as a probe of the local stiffness of the sample~\cite{Rabe2000}.
In a nutshell, Hooke's law states that the resonance frequency of a linear spring is given by $\omega = \sqrt{\frac{k}{m}}$, thus when the sample has a higher stiffness $k$, it leads to a higher resonance frequency of the tip-sample spring.
A map of the local stiffness of the sample is then created by scanning the tip across the sample while measuring this resonance frequency, as shown in Fig.~\ref{fig:BTO_experiment}(b,e).
The only limiting factor of the spatial resolution is the scanning speed and total duration of the experiment.  
A clear contrast can be observed between soft areas (brown) close to the wall and harder areas (purple) inside the domains, with an apparent reduction of the stiffness by $\approx 19\%$ at the walls.

The experimental observations hint at the strain texture that is associated with the ferroelectric \glspl{DW} as a large contributor to the softening.
Indeed, the softer regions (brown in Fig.~\ref{fig:BTO_experiment}(b,e)) extend up to 70 nm away from the \glspl{DW}, much farther than the width of the region wherein the polarization reverses (a couple of unit cells or 5 - 20 \AA \cite{Zhirnov1959}).
This argument is strengthened by the fact that the soft regions are even more pronounced around the bubble \glspl{DW} in panel (e) than around the straight \glspl{DW} in panel (b).
Indeed, it is known that strain textures of round \glspl{DW} decay as a power-law as compared with the exponential decay away from straight \glspl{DW} \cite{Landau1960}, whereas the length scale associated with the reversal of the polarization remains the same for both types of \gls{DW}.

In order to verify that the application of the tip to the surface does not change the domain morphology, up to 10 sequential measurements were performed, the results of which did not show a significant variation.
As mentioned in the introduction, it is possible to switch the polarization mechanically through the flexoelectric effect (see Eq.\ref{eq:BTO_flexoelectricity}), but the tip needs to be applied with much greater force than in these experiments.
Finally, the \gls{CRFM} tip is conductive and grounded in a short-circuit configuration during the measurements, allowing for polarization charge screening.

This concludes the overview of the experimental measurements performed by C. Stefani et al. in Ref.~\cite{Stefani2020} that lie at the base of this Chapter.
We now turn to a short summary of BaTiO$_3$.

\section{Barium titanate \label{sec:BTO_crystal}}
As mentioned previously, BaTiO$_3$ is not special from the point of view of the \gls{DW} softening, and all 180$^\circ$ ferroelectric \glspl{DW} should demonstrate the same behavior.

It is chosen here because it is a well studied archetypal example of an ABO$_3$ perovskite.
It has a ferroelectric polarization $P \approx 30 \times 10^5 \mu $C/m$^2$ at room temperature \cite{Mason1948, VonHippel1950, Ghosez1994}, and due to its excellent dielectric, piezoelectric and photorefractive properties, it can be found in many devices like capacitors \cite{American1963}, electromechanical transducers \cite{Schofield1957} and nonlinear optics \cite{Ramakanth2015}.

The crystalline structure of BaTiO$_3$ is shown in Fig.~\ref{fig:BTO_crystal}.
Starting from the cubic paraelectric $Pm3m$ phase at high temperature (panel (a)), it undergoes phase transitions at 393\,K, 273\,K and 183\,K to the ferroelectric tetragonal $P4mm$ (panel (b)), orthorhombic $Amm2$ and rhombohedral $R3m$ phases, respectively \cite{Mason1948, VonHippel1950, Marton2010}.
The microscopic origin of the ferroelectricity in BaTiO$_3$ has previously been studied with variational \gls{DFT} calculations \cite{Ghosez1995}, uncovering the intricate role that the hybridization between Ti and O orbitals plays.
This leads to the Born effective charges of the Ti and O ions to be quite different from what would be expected from a purely ionic picture.
The Born effective charge tensor $Z^*_{\kappa, \gamma\alpha}$ is defined in terms of the change in the $\gamma$ component of the polarization that occurs due to an infinitesimal periodic displacement $\delta$ of atom $\kappa$ in the direction $\alpha$:
\begin{equation}
Z^*_{\kappa, \gamma\alpha} = Z_{\kappa} \delta_{\gamma\alpha} + \Delta Z_{\kappa, \gamma\alpha}.
\end{equation}
There are thus two contributions: $Z_{\kappa}$ is the ionic charge of the atom, and $\Delta Z_{\kappa, \gamma\alpha}$ is the electronic contribution due to the displacement.
$Z_{\kappa}$ is then equal to +2 for Ba, +4 for Ti, and -2 for O, if the electron charge is taken to be -1.
From the \gls{DFT} calculations, it was found that the electronic contribution $\Delta Z_{\kappa, \gamma\alpha}$ due to hybridization is quite significant.
This leads to the total Born effective charges for Ti and O to be $Z^* = 7.29$, and $Z^*_{||} = -5.75$, respectively.
The $||$ in the latter denotes that the value was calculated for a movement of the oxygen along the O--Ti bond.
These large electronic contributions stem from the strong variation of the hybridization between O $2p$ and Ti $3d$ orbitals as the atoms are displaced.
For further details, see Ref.~\cite{Ghosez1995}.

All experiments were performed at room temperature, so we focus on the tetragonal phase shown in Fig.~\ref{fig:BTO_crystal}(b).
\begin{figure}[h]
	\IncludeGraphics{crystal.png}
	\caption{\label{fig:BTO_crystal}{\bf Crystal structure of BaTiO$_3$.} a) Paraelectric $Pm3m$ phase. b) Unit cell of the tetragonal $P4mm$ phase with the associated polarization in yellow. The Oxygen octahedra around Ti are displayed as the shaded polygons.}
\end{figure}

\section{Theory \label{sec:BTO_theory}}
Three possible mechanisms to explain the softening of purely ferroelectric \glspl{DW} were put forward by Tsuji et al. \cite{Tsuji2005}: the accumulation of defects (oxygen vacancies) at \glspl{DW}, reduced depolarization energy created by piezoelectricity at the \glspl{DW}, and the existence of \gls{DW} localized phonon modes~\cite{Chen2020} (e.g. breathing and sliding). We will discuss each of these contributions, with a more detailed investigation into the last one.
First, we introduce the continuum \gls{GLD} model that forms the backbone of the theoretical description.
We also show how the strain texture of the wall appears, and how more naive arguments fail to explain the softening.

\subsection{Ginzburg-Landau-Devonshire model}
The model that we use is based on the developments in Refs.~\cite{Zhirnov1959,L.N.Bulaevskii1963,Marton2010}.
We chose a continuum model because of the mesoscopic length scales that are involved with the problem.
One could argue that this simplification is detrimental to the description of the very sharp \glspl{DW}.
%, sporting a width on the order of a couple of unit cells (5 - 20 \AA) \cite{Zhirnov1959}.
Especially to describe the movement of the \gls{DW}, it is necessary to take the Peierls-Nabarro barriers into account that exist due to the lattice structure.
Naively, it seems that only atomistic first-principles based calculations manage to reproduce these accurately \cite{Meyer2002}, since in a perfect continuum picture no such barriers exist.
However, as we will touch on in further detail when the simulation technique is discussed in Section~\ref{sec:BTO_methods}, a numerical analog to these inter-cell barriers appears~\cite{Marton2018}.

A fully microscopic ab-initio simulation\footnote{Using \gls{DFT} calculations, for example.} of the strain mediated softening is infeasible from the computational point of view, requiring a dynamic calculation describing the movements of 10s of thousands of atoms during a period of many seconds.
% Indeed, looking at the experimental measurements in Fig.~\ref{fig:BTO_experiment}(b), the region where the softening is observed is around 70 nm wide.

We therefore turn our attention to the \gls{GLD} model.
The free energy density in terms of the primary \gls{OP} for the ferroelectric polarization $P$ and elastic strain slave \gls{OP} $\varepsilon_{ij} = \frac{1}{2}(\partial_i u_j + \partial_j u_i)$ can be written as:
\begin{eqnarray}\label{eq:BTO_energy}
&&f = f_{L}+f_{G}+f_{c}+f_{q}+f_{fl},\\
&&f_{L} = \alpha_{ij}(T)P_iP_j + \alpha_{ijkl}(T)P_{i}P_{j}P_{k}P_{l} + \alpha_{ijklmn} P_i P_j P_k P_l P_m P_n,\\
&&f_{G} = \frac{1}{2}G_{ijkl}\partial_i P_j\partial_k P_l,\\
&&f_{c} = \frac{1}{2}C_{ijkl}\varepsilon_{ij}\varepsilon_{kl},\\
&&f_{q}= - q_{ijkl}\varepsilon_{ij}P_{k}P_{l},\\
&&f_{fl}=\frac{1}{2}\Gamma_{ijkl}(\varepsilon_{ij}\partial_k P_l - P_k\partial_l\varepsilon_{ij}),
\end{eqnarray}
where $\partial_i$ signify spatial derivatives, and the indices enumerate ${x,y,z}$ with an implied summation over repeated ones.
The first term is the Landau free energy which describes the phase transition and formation of the ferroelectic polarization $P$.
Both the second and fourth-order coefficients $\alpha_{ij}$ and $\alpha_{ijkl}$ are temperature dependent and switch sign at the phase transition temperature.
This means that a sixth-order term with $\alpha_{ijklmn}$ has to be included in order to bound $P$ below the transition temperature.
The second term $f_G$ denotes the Ginzburg contribution to the energy, penalizing the spatial variation of $P$.  
This term influences the width of \glspl{DW}, where a larger $G$ leads to wider walls and vice versa.
The elastic energy density is described by $f_c$ with stiffness tensor $C_{ijkl}$, which has the form of the standard Hooke's law.
$f_q$ signifies the electrostriction, which is the main term that couples the polarization to the strain, and causes the domains to be stretched along the direction of the polarization.
It is the balance between this term and the elastic one that leads to the strain texture associated with the wall.
Lastly, to be complete, we can include the flexoelectric contribution $f_{fl}$, coupling gradients of strain to $P$ and vice versa.
In the results here reported, we chose to neglect it since we found from our simulations that it leads to tiny effects that do not influence the behavior we seek to explain.
The coefficients of the model for BaTiO$_3$ are reported in Tab.~\ref{tab:BTO_param}, where due to the parent cubic symmetry, only a limited set of these is unique.
For example $C_{xxxx} = C_{yyyy} = C_{zzzz}$, and $C_{xyxy} = C_{yxyx} = C_{yzyz} = C_{zyzy} = C_{xzxz} = C_{zxzx}$, and similar relations exist for G, q, and f. 

\begin{table}
\begin{tabular}{|c|c|c|c|c|c|}
	\hline
	 $\alpha$ & & $G$ & Jm$^3$C$^{-2}$& $q$ & JmC$^{-2}$\\
	 \hline
	 $\alpha_1$ & $3.34\cdot 10^5 (T - 381)$  JmC$^{-2}$& $G_{11}$ & $51 \cdot 10^{-11}$ & $q_{11}$ & $14.2 \cdot 10^{9}$ \\
	 $\alpha_{11}$ & $4.68 \cdot 10^6 (T-393) - 2.02 \cdot 10^8$ Jm$^5$C$^{-4}$& $G_{12}$ & $-2 \cdot 10^{-11}$ & $q_{12}$ & $-0.74 \cdot 10^{9}$ \\
	 $\alpha_{12}$ & $3.23 \cdot 10^8$ Jm$^5$C$^{-4}$ & $G_{44}$ & $2 \cdot 10^{-11}$  & $q_{44}$ & $1.57 \cdot 10^{9}$ \\
	 \cline{3-6}
	 & & $C$ &Jm$^{-3}$\\
	 \cline{3-4}
	 $\alpha_{111}$ & $-5.52 \cdot 10^7 (T - 393) + 2.76 \cdot 10^6 $ Jm$^9$C$^{-4}$ & $C_{11}$ & $27.5 \cdot 10^{10}$\\
	 $\alpha_{112}$ &  $4.47 \cdot 10^9 $Jm$^9$C$^{-4}$  & $C_{12}$ & $17.9 \cdot 10^{10}$\\
	 $\alpha_{123}$ & $4.91 \cdot 10^9$Jm$^9$C$^{-4}$  &$C_{44}$ & $5.43 \cdot 10^{10}$\\
	 \cline{1-4}
\end{tabular}
\caption{{\bf \gls{GLD} model parameters for BaTiO$_3$.} Voigt notation \cite{Voigt} is used to limit the amount of indices, e.g. $C_{11} = C_{zzzz}$, $C_{12} = C_{xxzz}$ and $C_{44}= C_{xzxz}$.  All coefficients are taken from Ref.~\cite{Marton2010}. The temperature T is taken to be 300 K in the simulations.\label{tab:BTO_param}}
\end{table}
To find the static equilibrium state in the most general sense, the integrated free energy $F[P,\varepsilon] = \int d^3x \,f[P(x), \varepsilon(x)])$ needs to be minimized, for which the variational method can be used.
This leads to the well-known Euler-Lagrange equations \cite{Cao1991, Marton2010}:
\begin{align}
	\label{eq:BTO_euler}
	\frac{\partial}{\partial x_j}\left( \frac{\partial f}{\partial P_{i,j}}\right) - \frac{\partial f}{\partial P_i} &= 0, \\
	\frac{\partial}{\partial x_j}\frac{\partial f}{\partial \varepsilon_{ij}} = 0,
\end{align}
where $P_{i,j} = \partial_j P_i$ and the last equation is a result of the stress-free mechanical equilibrium condition.

We start our analysis of the model with the one-dimensional case of a single domain with uniform $P_z$ (which we write $P$ in the following) and $\varepsilon_{zz}$ (written as $\varepsilon$) and no external forces, for which Eq.~\ref{eq:BTO_energy} reduces to:
\begin{equation}
	f = \alpha_1 P^2 + \alpha_{11} P^4 + \alpha_{111} P^6 - q_{11} \varepsilon P^2,
\end{equation}
and Eq.~\eqref{eq:BTO_euler} subsequently leads to:  
\begin{align}
	\frac{\partial f}{\partial \varepsilon} &= 0 \Leftrightarrow \varepsilon = \frac{qP^2}{C}\label{eq:BTO_e0},\\
	\frac{\partial f}{\partial P^2} &= 0 = \alpha_1 + (2 \alpha_{11} - \frac{q^2}{C})P^2 + 3\alpha_{111}P^4 \\
	& \Leftrightarrow \\
	P^2 &= \frac{-(2\alpha_{11} - \frac{q^2}{C}) + \sqrt{(2\alpha_{11}-\frac{q^2}{C})^2 - 12 \alpha_{111} \alpha_1}}{6\alpha_{111}}.
\end{align}
If the parameters of Tab.~\ref{tab:BTO_param} are filled in, we find $P = \pm 0.311$ C/m$^2$ and $\varepsilon = 5 \cdot 10^{-3}$.
The value of the polarization is higher than the one found when electrostriction is not included ($P = 0.265$ C/m$^2$ \cite{Marton2010}), because the system can lower its energy through $-qP^2\varepsilon$ by increasing $P$ and having tensile strain $\varepsilon > 0$. This is limited by the elastic energy penalty due to $C$.
Similarly, when all terms are included in the full numerical simulation, we find for the domains that $P_z$ is higher still, with values of $\pm 0.363$ C/m$^2$ and $1.4 \cdot 10^{-2}$ for the polarization and strain, respectively.

The effective stiffness $\tilde{C}_{ijkl}$ can be found by taking the double derivative of the free energy with respect to $\varepsilon$:
\begin{equation}
	\tilde{C}_{ijkl} = \frac{\partial^2 f}{\partial \varepsilon_{ij} \partial \varepsilon_{kl}},
\end{equation}
or, making the same assumptions as above (only $P_z$ and $\varepsilon_{zz}$), we arrive at an effective stiffness of
\begin{equation}
	\label{eq:BTO_domainC}
	\tilde{C} = C - \frac{q^2}{2\left(\alpha_{11}^2 - 3 \alpha_{111}\left(\alpha_1  - \frac{q^2 P^2}{C}\right)\right)}.
\end{equation}
Again filling in all the parameters, we find an effective stiffness $\tilde{C}_d = 0.778 \,C$ in the domain.
At the wall $P$ is zero, and, if there was no shear strain penalty, so is $\varepsilon$. This would then lead to an effective stiffness of $\tilde{C}_w = 0.642 \, C = 0.82\, \tilde{C}_d$, according to Eq.~\eqref{eq:BTO_domainC}.
While this looks like a triumph for the theory (since the experimental measurements found a $19\%$ softer wall), in reality, the shear elastic constants $C_{44}$ and compatibility relations lead to a nonzero strain $\varepsilon_{zz}$ at the wall.
From our simulations, we find that $\varepsilon_{zz}$ at the surface of the wall has a value of around 40\% of the one in the domain (see Fig.~\ref{fig:BTO_wall}(b)).
Filling this value into Eq.~\eqref{eq:BTO_domainC} leads to $\tilde{C}_w = 0.79 \, C = 1.01 \tilde{C}_d$, i.e. this naive derivation leads to a stiffer rather than softer wall.
Our simulations confirm this result, which can thus not explain the experimentally observed softening.

We, therefore, turn to the three previously mentioned, less naive mechanisms: defects, piezoelectricity, and domain wall movement described by localized phonons.

\begin{figure}[h]
	\IncludeGraphics{domainwall.png}
	\caption{\label{fig:BTO_wall} {\bf Domain wall profile.} The surface profile an out-of-plane the 180$^\circ$ \gls{DW} is shown. a) The polarization $P_z$ is denoted by the red and blue coloring. b) The magnitude of $\varepsilon_{zz}$ is encoded in the false colors. The enhanced deformation of the geometry illustrates the effect of the strain texture associated with the wall.}
\end{figure}

\subsection{Defects}
A thorough theoretical study on the behavior of defects in BaTiO$_3$ \glspl{DW} was done by Xiao et al. in Ref.~\cite{Xiao2005}.
It was shown there that a large accumulation of defects was found to exist at 90$^\circ$ \glspl{DW}, but not in 180$^\circ$ \glspl{DW}.
Looking again at the experimental measurements in Fig.~\ref{fig:BTO_experiment}(b,e,f) we can make an argument that supports this claim.
The supposed mechanism for softening due to defects is a weakening of the interatomic bonds since the largest portion of defects are oxygen vacancies~\cite{Tsuji2005}.
This should be fairly isotropic, or at least orthotropic, in the tetragonal phase of BaTiO$_3$ since it is close to the parent cubic structure.
Defect mediated softening should thus not depend on the orientation of the polarization.
However, if we compare in Fig.~\ref{fig:BTO_experiment}(e) the stiffness signatures of the 180$^\circ$ \glspl{DW} with out-of-plane polarization, and those with in-plane polarization, we see that only the former demonstrate the softening.
This region of the sample is small enough so that variations in the quality of the sample and the defect density are small.
From these two arguments, we can safely exclude defects as a contributing factor to the \gls{DW} softening.

\subsection{Piezoelectricity \label{sec:BTO_piezoelectricity}}
The second mechanism that could lead to softening can not be neglected, however \cite{Tsuji2005,Stefani2020}.
When the tip is applied to the sample it causes a non-uniform pressure field at the surface.
It is larger at the center of the contact region and smaller at the edges and decays into the material.
This in turn leads to a non-uniform strain which causes a non-uniform polarization through the piezoelectricity $\partial_l P_i \sim e_{ijk} \partial_l \varepsilon_{jk}$, where $e_{ijk}$ are the piezoelectric coefficients.
Since the polarization is constituted by electric dipoles, their non-uniform distribution causes bound charges.
The tip thus effectively creates internal depolarizing fields, similar to how bound charges at the surface of domains lead to depolarizing fields, as was discussed in the introduction (see Fig.~\ref{fig:BTO_depolarizing_schematic}). 
As we will show now, these resulting fields are larger in the domain compared with those in the \glspl{DW}.
For illustrative purposes, we look at an example situation, in which the tip is applied along the $z$-direction, and the \gls{DW} lies in the $yz$-plane. See Fig.~\ref{fig:BTO_depolarizing} for further details.
It is important to bear in mind that the components of the piezoelectric tensor all switch signs across the \gls{DW}, i.e. $e_{ijk}(P_z) = - e_{ijk}(-P_z)$. 
Two distinct contributions can then be identified, resulting from the longitudinal and shear parts of the piezoelectric tensor, shown in Fig.~\ref{fig:BTO_depolarizing} (a) and (b), respectively.

The longitudinal piezoelectricity leads to a change of the polarization which is given by $P_z - P^0_z \sim e_{zzz} \varepsilon_{zz}$, where $P^0_z$ denotes the value of $P_z$ without an applied tip.
Moreover, since the pressure field due to the tip leads to a high strain $\varepsilon_{zz}$ at the surface that decays into the material, causing $\partial_{z} P_z \sim e_{zzz} \partial_z \varepsilon_{zz}$ to also be nonzero, which creates additional bound charges.
As shown by the + and - in Fig.~\ref{fig:BTO_depolarizing}(b), the bound charges in the domain are all identical.
The resulting depolarizing field and associated energy contribution are large.

If the tip is applied at the \gls{DW}, however, positive bound charges appear on one side and negative on the other, because of the opposite signs of $e_{zzz}$ in the two domains.
The depolarizing fields compensate in this case, and the resulting energy penalty is lower.
This could lead to a softening at the \gls{DW} since this lower energy penalty allows for a deeper indentation and thus results in a softer appearing material.
However, as discussed in the experimental Section, the tip is conductive and grounded at all times.
This causes short-circuit conditions which completely screen the bound charges created through longitudinal piezoelectricity.
\begin{figure}[h]
	\IncludeGraphics{depolarizing.png}
	\caption{\label{fig:BTO_depolarizing}{\bf Piezoelectric effect.} A pictorial representation of the contribution of the shear (a) and longitudinal (b) piezoelectric effect. The black arrows denote the direction of the polarization, with the wall located on the boundary between the purple and yellow shaded domains. The tip is represented by the grey bell-shapes, with the resulting change in bound charge in red and blue, and by the + and -. An example of the depolarizing field is shown by the light blue arrows in panel (a). All changes to the polarization due to the applied tip are exaggerated for clarity.}
\end{figure}

We, therefore, look whether the shear part of the piezoelectricity (see Fig.~\ref{fig:BTO_depolarizing}(a)) can lead to the softening.
When a tip is applied, this part generates a small in-plane component to the polarization, $P_x \sim e_{xzz}\,\varepsilon_{zz}$.
Because of the round shape of the tip, the pressure field is non-uniform and the resulting strain varies as $\partial_x\,\varepsilon_{zz}$
In turn, the polarization also varies as $\partial_x P_x\neq 0$, and associated bound charges are created along the $x$-direction.
Moreover, $\partial_x \varepsilon_{zz}$ is negative on the left side of the tip and positive on the right, causing the resulting bound charges to be oriented in a head-head or tail-tail configuration.
This leads to large depolarizing fields in the domain.

If the tip is applied at the \gls{DW}, instead, where the piezoelectric components switch sign, the bound charges along $x$ are created in a head to tail configuration.
This leads to a significantly lower depolarizing field compared with the domain case, and the reduced energy penalty results in an apparent \gls{DW} softening.
In this case, the bound charges can not be screened effectively by the short-circuited tip, ultimately resulting in the first nonzero contribution to the softening of ferroelectric \glspl{DW}. 

Simulations similar to those that will be discussed later were performed by our collaborators for a 2D geometry, confirming this mechanism (see Ref.~\cite{Stefani2020} for further details).
It was found that that the shear piezoelectricity (Fig.~\ref{fig:BTO_depolarizing}(a)) leads to the wall appearing 5\%-10\% softer than the domains.
However, this piezoelectric effect is a result of the reversal of the polarization at the \gls{DW}, and it can thus not account for the wide region of softening observed in the experiments (70 nm).
We, therefore, turn to the strain texture of the \gls{DW} as the possible source of this long-range softening, since it has a much wider profile than the \gls{DW} itself.

\subsection{Domain wall localized phonons}
The crux of the matter is that due to the fundamentally different structure of the \gls{DW}, additional soft phonon modes appear which are not present in the domain \cite{Chen2020}.
The two most important ones are the breathing and sliding modes, shown in Fig.~\ref{fig:BTO_breathing_sliding}(a,b), respectively.
These can be excited mechanically by the applied tip, ultimately making the \gls{DW} appear softer.
Both are a result of the previously described diminished strain at the wall, resulting in a dip at the surface of the material, as shown in Fig.~\ref{fig:BTO_wall}.
The breathing mode is associated with a change in the width of this dip (see Fig.~\ref{fig:BTO_breathing_sliding}(a)).
When a tip with a wider radius than the dip is pressed into the material, it will apply a force on either side of the wall, increasing its width.
This is favorable for the Ginzburg part $\partial_i P_j \partial_k P_l$ of the free energy density since wider walls have lower $\partial_i P_j$.
When the tip is pressed at the domain, we saw that through the piezoelectric effect, the polarization decreases locally.
This leads to a nonzero $\partial_i P_j$, in turn resulting in a nonzero contribution to the Ginzburg energy which was previously zero for the domain.
The energy difference between these two cases makes the \gls{DW} appear softer than the domains.
However, a similar argument can be made as for the previously discussed piezoelectric contribution in that the tip needs to be in contact with both sides of the \gls{DW} to maximize the excitation of the breathing mode.
Since its contact radius is only around 7 nm, it is hard to believe that the excitation of the breathing mode contributes significantly to the softening observed when the tip is applied 30 nm away from the wall.
Moreover, from the two localized \gls{DW} modes, the breathing is significantly harder than the sliding one, because it is associated with an expensive elastic deformation of the sample.
The sliding mode, shown in Fig.~\ref{fig:BTO_breathing_sliding}(b), instead describes a rigid motion of the \gls{DW} perpendicular to its plane, thus not incurring such elastic penalty. 
\begin{figure}
	\IncludeGraphics{breathing_sliding}
	\caption{\label{fig:BTO_breathing_sliding}{\bf Breathing and Sliding modes.} a) Demonstration of the widening of the \gls{DW} upon application of a tip, representing a breathing mode. The higher the force of the tip, the wider the wall becomes. The wall is not centered at 0 because of the numerical Peierls-Nabarro and mesh dimensions, however, the applied force is centered exactly above the wall. b) Pictorial demonstration of the sliding mode, which gets excited by the tip, making the wall slide towards it.}
\end{figure}

This brings us, finally, to what we claim as the main contributor to the softening effect.
Indeed, closer inspection of the energy terms in Eq.~\eqref{eq:BTO_energy} leads to the conclusion that no energy penalties are incurred by the rigid sliding \gls{DW} motion in the perfect continuum case.
If the tip is then applied in an area where the strain texture associated with the wall is present, it will try to move this profile underneath it in order to achieve mechanical equilibrium.
This is because the tensile strain caused by the electrostriction $f_q$ is opposite to the compressive force applied by the tip.
To restore mechanical equilibrium, the material needs to balance these two forces.
Since the tensile strain is already diminished at the \gls{DW}, moving it underneath the tip is a relatively inexpensive way to partly achieve this balance.
The movement of the preexisting dip towards the tip then leads to relatively large deformation, making the material appear soft.
One can say, equivalently, that the presence and excitation of the soft sliding mode lead to a lower energy penalty when the tip is applied.
In the domain, this sliding mode is not present leading to a higher energy penalty for the deformation and thus a higher observed stiffness.

While in the ideal continuum case, the sliding mode is completely free, in reality, this is not the case.
As mentioned before, in a real material there are Peierls-Nabarro barriers between the unit cells, possible pinning defects, and electrostatic effects that will increase the energy of the sliding mode.
While it is extremely hard to make an analytical description of the behavior, we can still make statements about two extreme situations.
If the force applied by the tip is large enough, the wall will fully slide towards it, maximizing the elastic energy gain.
If instead, the force is very small, the wall remains inside the original Peierls-Nabarro potential but deviates from the equilibrium position.
% \lp{The situation that happens in the real material is more like a mix between the two, the top part of the wall bends almost completely towards teh tip, but it's not moved as a whole because the bottom/bending electrostatics pins it. Can we say that these things are causing the potential for the entire wall to behave like the one we describe below?}

A simple free energy expansion can be made to illustrate the latter case.
The wall at position $x_{DW}$ (originally zero) is pinned by a parabolic potential with strength $a$, and perturbed by a tip applying a force $F_z$ at $x_{tip}$, leading to the energy:
\begin{equation}
	E = E_0 - F_z u_z (x_{tip} - x_{DW}) + \frac{1}{2}a x_{DW}^2.
\end{equation}
We expand this equation under the assumption of a small $x_{DW}$, i.e. that the wall does not move far from the $x_{DW}=0$ equilibrium case.
Minimizing the energy under this assumption, we obtain $x_{DW} = -F u'(x_{tip})/a$, with a correction to the stiffness $\delta C = u'(x_{tip})^2/a^2$.
From this, we can conclude that the effect is greatest when tip is applied where $u'(x_{tip})$ is large, i.e. within the strain variation around the wall caused by the above discussed electrostrictive coupling in Eq.~\eqref{eq:BTO_electrostriction}.

In reality, depending on the size of the tip, where it is applied, and the size of the force, a combination of the two above mentioned extremes will occur.
From the experimental observations, we can posit that the applied force is large enough to fully slide the top of the wall towards the center of the tip if it is positioned close enough to the wall.
In doing so, the dip at the surface of the wall moves underneath the tip maximizing the elastic energy gain.
As discussed before, this leads to a large apparent softening and is most likely to be what happens within the 70 nm region around \glspl{DW}.
Since it is clear from the repeated experimental measurements that the walls do not move permanently, we claim that the lower portion of the wall remains at the center of its original Peierls-Nabarro potential.
This is probably due to pinning by defects or the substrate.

Due to the complexity of the model, and the interaction of the strain with the tip, we forfeit the hunt for a purely analytical description and instead proceed with numerical simulations to verify the above statements.

\section{Methods \label{sec:BTO_methods}}
There are generally speaking two ways of minimizing the total free energy $F=\int_V d\bm{r} f$ of the system with volume $V$: either through solving the system of Euler-Lagrange equations in Eq.~\eqref{eq:BTO_euler}, or by direct minimization of $F$.
Both are completely equivalent on a mathematical level but require different numerical implementations.
In our code, we chose the latter.
Moreover, since minimizing $F$ w.r.t $\varepsilon$ requires a careful consideration of the compatibility conditions between the different elements (so that the material does not break)\cite{Marton2007}, we chose to use the displacement $u$ instead.
This means we rewrite the equations in Eq.~\eqref{eq:BTO_energy} by filling in $\varepsilon_{ij} = \frac{1}{2}(\partial_i u_j + \partial_j u_i)$.

To solve the model, we thus have to compute and minimize this total free energy by integrating the free density in Eq.~\eqref{eq:BTO_energy}, written in terms of the values and gradients of the two order parameters $P$ and $u$, over the entire system.
In the continuum formulation, these have a value at each point in space and are thus sometimes referred to as fields, with the resulting models often referred to as phase-field models.
Finding the values of $P$ and $u$ that minimize the total free energy $F[P,u]=\int_V d\bm{r} f(\bm{P}(\bm{r}),\bm{u}(\bm{r}))$ then leads to the equilibrium configuration of the system.

We perform this integration numerically, due to the complexity of Eq.~\eqref{eq:BTO_energy} and the highly heterogeneous situation that occurs when a tip is applied.
The \gls{FEM} provides a framework to do exactly that.
Its ability to handle complex geometries and large length scales makes it one of the most well-established methods to numerically solve realistic mathematical and physical models (see appendix ~\ref{ch:FEM} for further details).

Using the parameters from previous ab-initio results \cite{Marton2010} and reported in Tab.~\ref{tab:BTO_param}, we first optimized the equilibrium situation for both a uniform domain and a single domain wall. The result of the latter is shown in Fig.~\ref{fig:BTO_wall}.
A Gaussian force field, $\frac{a}{\sigma \sqrt{2\pi}}e^{-\frac{(x-x_{\rm tip})^2}{2\sigma^2}}$ with $\sigma = 1 \cdot 10^{-9}$ representing the contact radius and $a = 1 \cdot 10^{-18} \frac{eV}{nm^3}$, was then applied at various positions $x_{\rm tip}$ throughout both geometries.

The single-domain simulation is used to gauge the influence of finite-size effects that are inherently present in our open boundary condition simulations, using a geometry of finite dimensions $(L_x, L_y, L_z)$.
If the tip is applied to the side of the geometry, in the case of a single domain, it deforms more than when it is applied to the center because there is less material on one of the sides of the tip to support it.
We have tested the extent to which this contributes to our simulations, as demonstrated in Fig.~\ref{fig:BTO_numerical_effects}(a) for different lengths $L_x$ of the geometry. As one could expect, the effect is most influential in simulations with a small geometry.
Since this is nonphysical, we use the domain simulation to calculate the amount of such spurious additional deformation and compensate for it in the \gls{DW} simulation assuming that the finite-size effect is identical since the geometry is the same in both cases.

To conclude this Section, we would like to comment on the issue of the Peierls-Nabarro barriers \cite{Peierls1940,Nabarro1947}, since they influence the movement of the \gls{DW}.
In essence, they are a result of the non-homogeneity of the crystalline lattice on the unit-cell length scale.
This causes the \gls{DW} to have a preferred center inside the unit cells, usually either on the edge or in the middle.
A thorough study from first principles was performed by Meyer et al. \cite{Meyer2002}, where it was correctly stated that only by using this degree of granularity, one can achieve quantitative ab-initio results.
The pure continuum \gls{GLD} model we use does not account for the physical Peierls-Nabarro barriers, as we discussed previously, making the \gls{DW} sliding completely penalty free, which is at odds with the behavior in real materials.
We are thus faced with the commonly reappearing issue of requiring to simulate physics that occur at different length scales.

In this case, however, we are to some extent saved by a spurious numerical effect that mimics the naturally occurring Peierls-Nabarro barriers.
This was investigated by Marton in Ref.~\cite{Marton2018} for the case of BaTiO$_3$.
It boils down to the fact that it is numerically favorable to situate the center of the wall at a node rather than inside an element, because of the approximated interpolation of the fields.
In this way, the simulation avoids sampling the paraelectric state, which is at the top of the double-well potential of the Landau part $f_l$ of the energy in Eq.~\eqref{eq:BTO_energy}.
When the wall is forced to move across the grid due to the applied tip, it is forced to overcome this high energy configuration in order to transition between equilibrium positions.
This in turn results in the numerical analog of the Peierls-Nabarro barrier.

\section{Results and discussion \label{sec:BTO_results}}
Since we implemented our own version of the Finite Element Method, we first performed some consistency checks with previous calculations before turning to the simulations of the \gls{DW} softening.

The \gls{DW} energy we obtained is 15.2 mJ/m$^2$, very close to the value of 16 mJ/m$^2$ previously found from a first-principles calculation in Ref.~\cite{Padilla1996}.
Furthermore, we verified that it is not influenced by the size of the numerical geometry or the density of the mesh, indicating that the numerics of our method does not influence the energetics of the static wall.

To confirm the existence and consistent behavior of the numerical Peierls-Nabarro barriers, we performed simulations where we rigidly shifted the wall from its central position inside the grid, similar to the procedure followed in Ref.~\cite{Marton2018}.
The energy was then calculated at each step in order to get an indication of the numerical Peierls-Nabarro barriers of our mesh.
The results for different element sizes are shown in Fig.~\ref{fig:BTO_numerical_effects}(c,d).
The barriers are very close to the harmonic shape of the real Peierls-Nabarro barriers for all element sizes, where larger sizes lead to larger barrier energies and periodicity, as expected.
In the following simulations, we chose the element size to be 0.3 nm which offers a good balance between the amount of pinning and the computational cost. 
\begin{figure}[h!]
	\IncludeGraphics{numerical_effects.png}
	\caption{\label{fig:BTO_numerical_effects}{\bf Numerical effects.} a) Maximum indentation of the domain by the applied tip, depending on its center along the $x$-dimension of the geometry with length $L_x$. b) Visual demonstration of the asymmetric deformation and $\varepsilon_{zz}$ due to open boundary conditions, where the tip is applied close to the edge of the sample. c) The evolution of the energy when the wall is shifted with $\delta_w$ away from the original position at 0. This mimics the Peierls-Nabarro barriers of the crystal. The periodicity of the lattice is mimicked by mesh, where the distance between barrier maxima is given by the element size. It is also clear that large element sizes lead to larger barriers. d) The dependence of the Peierls-Nabarro barrier height on the element size.}
\end{figure}

We now turn to the \gls{DW}-localized modes.
First, we demonstrate the influence of the breathing mode, shown in Fig.~\ref{fig:BTO_breathing_sliding}(a).
The absolute value of the polarization is used as an indication of the \gls{DW} width, and a tip was applied to the center of the wall with varying force.
We can see that the wall indeed widens slightly when higher forces are applied, but that the effect is small, and the necessary force to achieve any widening is significantly larger than the experimental one.

As discussed in the theory Section~\ref{sec:BTO_theory}, the domain wall sliding mode is a more likely candidate to explain the long-range \gls{DW} softening.
To mimic the experimental conditions, we applied the tip force field at various distances away from the wall.
In order to minimize the impact of the numerics, more specifically the sampling of the force field, we always centered the tip exactly at a mesh node.
This means that the total force applied to the sample by the tip is identical for all simulations.
The relative softening is then estimated as the ratio between the surface deformations in the domain and at the wall.
Two sets of simulations were performed, one for which the bottom of the wall was pinned to the original equilibrium position in order to mimic a bending of the \gls{DW}, and one where we let the wall as a whole slide freely.
The former behaves very similarly to the latter, except that it acquires an additional penalty due to the increased \gls{DW} area and non-zero Peierls-Nabarro barriers that the \gls{DW} is located on.
The results of the former are shown in Fig.~\ref{fig:BTO_bending_sim}, and of the latter in Fig.~\ref{fig:BTO_sliding_sim}.
As seen in Fig.~\ref{fig:BTO_bending_sim}(e) and Fig.~\ref{fig:BTO_sliding_sim}(e), and predicted by the above statements, the wall is softest not in the center, but rather at the point where $u'$ or $\varepsilon_{xz}$ is highest, leading to the highest force applied to the wall making it bend or slide.
Comparing the two simulation protocols, we can also observe that when the wall is allowed to slide, the region where the softening takes place is much wider, and the degree of softening is higher.
We find a maximum softening of about 20\% in this case, which is remarkably close to the experimental measurements.
\begin{figure}
	\IncludeGraphics{2D_bending}
	\caption{\label{fig:BTO_bending_sim}{\bf Bending Wall} Simulations where the bottom of the wall is kept fixed to the equilibrium without a tip. In each panel, the coloring indicates the position where the tip is applied to the sample. a) Surface profile of the wall. b) Polarization distribution at the surface. Notice the movement of the wall center when the tip is applied. c) Deformation of the surface after applying the tip. d) Maximum absolute value of the deformation in (c). e) Relative stiffness depending on where the tip is applied. f) Strain profile when the tip is too far fromthe  wall to make it bend. The wall is represented by the white line and does not move due to the tip. g) Bending of the wall when the tip is applied close to the wall. The full and dashed white lines denote the position of the wall before and after the tip is applied, respectively.}
\end{figure}
\begin{figure}
	\IncludeGraphics{2D_sliding}
	\caption{\label{fig:BTO_sliding_sim}{\bf Sliding Wall} Simulations where the bottom of the wall is allowed to change. The panels denote the same as in Fig.~\ref{fig:BTO_bending_sim}. Allowing the wall to fully slide leads to a bigger softening and wider softening region.}
\end{figure}

The situation in the real sample is probably a combination of these two, the argument for which is as follows.
Due to the relatively small $z$-dimension that we used for our simulations (20 nm) compared with the sample in the experiment (0.3 cm), a bending of the wall much more difficult to achieve in our simulations in comparison with reality because the relative increase of the \gls{DW} area upon bending is much larger. 
In the real sample the wall, therefore, most likely appears as rigidly sliding towards the tip inside the 70 nm in-plane region around the wall.
On the other hand, the wall is most likely pinned inside the material by some defects or the substrate and will move back towards the equilibrium position when the tip moves too far away from it. The full result on large lengthscales is therefore analogous to the bending of the wall in our simulations.

We claim that these lead to the main source of the observed softening in purely ferroelectric \glspl{DW}.
\section{Conclusions \label{sec:BTO_conclusion}}

In this Chapter we have demonstrated how purely ferroelectric, i.e. non-ferroelastic, 180$^\circ$ \glspl{DW} can appear mechanically softer than the domains they separate.
Using experimental observations as the starting point, we investigated many possible mechanisms and deduced which were most likely to contribute the most.
The main observation was that the region of the softening is relatively wide, whereas the ferroelectric \glspl{DW} are known to be very narrow.
This pointed us towards the strain texture as the main property of the \glspl{DW} that could lead to this softening behavior.
Indeed from earlier work by Chen et al. in Ref.~\cite{Chen2020} we knew that \glspl{DW} host localized phonon modes.
Focusing on the breathing and sliding phonon modes of the \gls{DW}, we performed Finite Element Method simulations using our in-house code to confirm our suspicions.
From our simulations, and further supported by intuitive arguments, we found that the sliding mode plays the biggest role in the \gls{DW} softening.
It is an interesting illustration of the fundamentally different behavior of \glspl{DW} in comparison with the domains they separate.

On the technological side, the mechanical softening could allow for the efficient mechanical distinction between \glspl{DW} and domains. If the domains represent a 0 bit and the walls a 1 bit, this would result in the efficient reading of data through the resonance frequency of the tip.
Furthermore, the sliding towards the tip of the indentation at the surface of the \gls{DW} could open the door to mechanically move pre-formed \glspl{DW} in a thin film sample.
The thin film should help reduce the \gls{DW} pinning by defects and Peierls-Nabarro barriers leading to a higher \gls{DW} mobility.
The fact that no elastic penalty is incurred through the sliding of purely ferroelectric 180$^\circ$ \glspl{DW} would make such mechanical manipulation of \glspl{DW} and domain structures in ferroelectrics very efficient.
This is in stark contrast to the previously studied twin or ferroelastic \glspl{DW}, which can also be excited mechanically, but at greatly elevated energy costs since their movement is associated with large elastic deformations.
