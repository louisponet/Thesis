%
\documentclass[phd, titlesmallcaps,copyrightpage,foronline,oneside]{SNSthesis}
\usepackage[utf8]{inputenc}
\usepackage{amsmath,xcolor}
% \usepackage[backend=biber]{biblatex}
% \usepackage[margin=30mm]{geometry}
% \renewcommand{\baselinestretch}{1.5}
% \usepackage{cancel}
\usepackage{siunitx}
\usepackage{graphicx}
% \usepackage{smartdiagram}
% \usepackage{subcaption}
% \usepackage[caption=false]{subfig}
% \captionsetup[subfigure]{labelformat=brace,position=top,singlelinecheck=off,justification=raggedright}
% \usepackage[normalem]{ulem}
% \usepackage{tikz}
\usepackage{bm}
\usepackage{fancyhdr}
\usepackage{cleveref}
\usepackage{tikz}
\usetikzlibrary{shapes,arrows,decorations.pathmorphing}

% \usepackage{natbib}
\usepackage{hyperref}
\usepackage[english]{babel}
\addto{\captionsenglish}{\renewcommand{\bibname}{References}}



\usepackage[linesnumbered,ruled]{algorithm2e}
\SetAlgoCaptionSeparator{}	% remove colon in caption
\makeatletter
\newenvironment{namedalgorithm}[1]{%
	\medskip
	\renewcommand{\algocf@algocfref}{#1}% update algorithm name
	\SetNlSty{textbf}{}{\enspace} % manual spacing after line-numbers (revtex)
	\DontPrintSemicolon % remove semicolon
	\begin{algorithm}[H] %
	}{\end{algorithm}%
	\medskip}
\makeatother

% Extractable work
\usepackage[percent]{overpic}

% \addbibresource{./Introduction/bib.bib}
% \addbibresource{./Theory/bib.bib}
% \addbibresource{./GdMn2O5/bib.bib}
% \addbibresource{./Rashba/bib.bib}
% \addbibresource{./Softening/bib.bib}
% \addbibresource{./Cr/bib.bib}

%to be used in main text
% \newcommand{\ket}[1]{$\left|#1\right\rangle$}
% \newcommand{\bra}[1]{$\langle #1\rvert$}
\newcommand{\ket}[1]{$|#1\rangle$}
\newcommand{\bra}[1]{$\langle #1\rvert$}

%To be used within math env
\newcommand{\Ket}[1]{\left|#1\right\rangle}
\newcommand{\Bra}[1]{\left\langle #1\right\rvert}
\newcommand{\BraKet}[1]{\left\langle #1\rvert #1\right\rangle}
\newcommand{\Sandwich}[2]{\left\langle #1\rvert #2\right\rangle}
% \newcommand{\Ket}[1]{|#1\rangle}
% \newcommand{\Bra}[1]{\langle #1\rvert}
% \newcommand{\BraKet}[1]{\langle #1\rvert #1\rangle}
% \newcommand{\Sandwich}[2]{\langle #1\rvert #2\rangle}

\newcommand{\kpoint}{$\bold{K}$ }
\newcommand{\Kpoint}{$\bold{K'}$ }
\newcommand{\gpoint}{$\bold{\Gamma}$}

\newcommand{\angstrom}{\mbox{\normalfont\AA} }
% Commands for Kets and Bras
\newcommand{\Wan}[2]{w_{#1}^{\bm{#2}}}
\newcommand{\tildeWan}[2]{\tilde{w}_{#1}^{\bm{#2}}}
\newcommand{\WanKetr}[2]{\Ket{\Wan{#1}{#2}(\bm{r})}}
\newcommand{\tildeWanKetr}[2]{\Ket{\tildeWan{#1}{#2}(\bm{r})}}
\newcommand{\WanKet}[2]{\Ket{\Wan{#1}{#2}}}
\newcommand{\tildeWanKet}[2]{\Ket{\tildeWan{#1}{#2}}}
\newcommand{\WanBrar}[2]{\Bra{\Wan{#1}{#2}(\bm{r})}}
\newcommand{\tildeWanBrar}[2]{\Bra{\tildeWan{#1}{#2}(\bm{r})}}
\newcommand{\WanBra}[2]{\Bra{\Wan{#1}{#2}}}
\newcommand{\tildeWanBra}[2]{\Bra{\tildeWan{#1}{#2}}}

\newcommand{\Bloch}[1]{\psi_{#1}^{\bm{k}}}
\newcommand{\tildeBloch}[1]{\tilde{\psi}_{#1}^{\bm{k}}}

\newcommand{\BlochKet}[1]{\Ket{\Bloch{#1}}}
\newcommand{\tildeBlochKet}[1]{\Ket{\tildeBloch{#1}}}
\newcommand{\BlochBra}[1]{\Bra{\Bloch{#1}}}
\newcommand{\tildeBlochBra}[1]{\Bra{\tildeBloch{#1}}}

\newcommand{\Blochr}[1]{\psi_{#1}^{\bm{k}}(\bm{r})}
\newcommand{\tildeBlochr}[1]{\tilde{\psi}_{#1}^{\bm{k}}(\bm{r})}

\newcommand{\BlochKetr}[1]{\Ket{\Blochr{#1}}}
\newcommand{\tildeBlochKetr}[1]{\Ket{\tildeBlochr{#1}}}
\newcommand{\BlochBrar}[1]{\Bra{\Blochr{#1}}}
\newcommand{\tildeBlochBrar}[1]{\Bra{\tildeBlochr{#1}}}

\newcommand{\unkr}[2]{u_{#1}^{\bm{#2}}(\bm{r})}
\newcommand{\unk}[2]{u_{#1}^{\bm{#2}}}

\newcommand{\unkrBra}[2]{\Bra{\unkr{#1}{#2}}}
\newcommand{\unkrKet}[2]{\Ket{\unkr{#1}{#2}}}
\newcommand{\unkBra}[2]{\Bra{\unk{#1}{#2}}}
\newcommand{\unkKet}[2]{\Ket{\unk{#1}{#2}}}

\newcommand{\inveikr}[1]{e^{-i \bm{k}\cdot \bm{#1}}}
\newcommand{\eikr}[1]{e^{i \bm{k}\cdot \bm{#1}}}

%For editing
\newcommand{\sa}[1]{{\color{violet}[#1]}}
\newcommand{\lp}[1]{{\color{red}[#1]}}

% To make subfiles graphics be relative
\def\Include#1{%
        \def\ChapterPath{#1/}%
        \def\GraphicsPath{\ChapterPath Images/}%
        \include{\ChapterPath#1}}

\newcommand\IncludeGraphics[2][width=\linewidth]{%
    \includegraphics[#1]{\GraphicsPath #2}}   

% \usetikzlibrary{shapes,arrows,decorations.pathmorphing}
\newcommand{\thesistitle}{Interplay between complex orders in functional materials}
\newcommand{\name}{Louis Ponet}
\newcommand{\email}{louis.ponet@iit.it; louis.ponet@sns.it}

\ifpdf
    \pdfinfo { 
              % /Title  (\thesistitle)
              /Title  (Interplay between complex orders in functional materials)
               /Creator (pdflatex) % probably this
               /Producer (LaTeX with hyperref) % something like this
               /Author (\name \email)
               %/CreationDate (D:20100204000000)  %format D:YYYYMMDDhhmmss
               % if /CreationDate is left out it will default to the file creation date
               /Keywords (Physics)}
    \pdfcatalog { /PageMode (/UseOutlines) /OpenAction (fitbh)  }
\fi




% 	{\large Department of Nanosciences, Scuola Normale Superiore di Pisa}\\
% 	{\large Quantum Materials Theory, Istituto Italiano di Tecnologia}}

% \author{Louis Ponet}
\date{today}

\begin{document}
\frontmatter

% acknowledgements, titlepage, abstract, list of publications
\title{\thesistitle}

\ifthenelse{\boolean{foronline}}{
  \author{\href{mailto:\email}{\name}}
  \department{Scienze}
}{
  \author{\name}
  \department{Scienze}
}


\titlepage

\chapter*{Abstract}
In an effort to reduce the sheer complexity associated with the many degrees of freedom encountered in Condensed Matter, it is beneficial to separate the full system under investigation into smaller subsystems.
This separation is most often performed on the basis of energy and time scales, allowing one to focus on the subsystem with the energy scale of interest.
Approximations for the influence of the decoupled subsystems can then be made treating them either as a smeared out average, leading to mean-field approximations, or as constant background potentials, resulting in adiabatic approximations. This procedure relies on the weak interaction, or coupling, between the subsystems.
For example, in the case of the electron-phonon interaction, the influence of the electrons on the phonons, describing the displacements of the ions, can be approximated as an average charge field, with the fluctuations cancelling out on the time scales involved with the phonon dynamics.
The electrons, on the other hand, experience the ionic potential as a constant background, remaining at the instantaneous ground-state with respect to it. This is, of course, the well-known Born-Oppenheimer or adiabatic approximation.
When the interaction between the subsystems is non-negligible, however, these approximations can fail spectacularly.
A prime example is superconductivity, where the electron-phonon coupling forms the linchpin for the exotic physics that is observed in these materials.

In this Thesis we will investigate similar cases, where the interaction between the subsystems becomes as important as the other energetics that govern them.
This leads to novel and unexpected behavior in these materials, which we study using analytical and numerical tools, often inspired by experimental measurements.

In Chapter~\ref{ch:Rashba} we commence by investigating the underlying mechanisms that lead to the large Rashba-like spin-splitting in the electronic band structure of ferroelectric semiconductors with large atomic SOC.
We uncover that, contrary to what is generally accepted, it is not the purely relativistic Rashba-Bychkov effect that lies at the core of this spin-splitting, but rather a combination of electrostatic effects with the appearance of a nonzero orbital angular momentum of the eigenstates.

This is followed in Chapter~\ref{ch:GdMn2O5} by a study on the very peculiar magnetoelectric switching that is observed in multiferroic GdMn$_2$O$_5$.
Depending on the angle between the crystalline $a$ axis and an externally applied magnetic field, we uncover three distinct switching regimes for the ferroelectric polarization, one of which demonstrates a novel four-state hysteresis loop.
Furthermore, this four-state regime displays a unidirectional 360$^\circ$ rotation of half the spins inside the material, during a double up--down sweep of the magnetic field.
This unexpected behavior is a prime example of the additional behavior that can appear as a result of strongly coupled orders.

In Chapter~\ref{ch:CrSDW} the focus will be shifted towards the aspect of dynamical control with increased granularity, that can be achieved in systems with strongly coupled orders.
We show that, in a thin film of elemental Chromium, an extremely precise control over a lattice vibration can be achieved through the excitations of a coupled spin density wave.
By applying a carefully chosen set of optical pulses, this allows us to indirectly modulate the amplitude of the lattice vibration, increasing or destroying it in the process.
We formulate and fit a Landau model based on the experimental measurements with two pulses, and speculate on the extended possibilities than could be achieved with a larger set of pulses. 

Finally, We conclude this Thesis in Chapter~\ref{ch:Softening} by a systematic study of the implications on the mechanical properties of 180$^\circ$ ferroelectric domain walls in BaTiO$_3$, due to the interaction between strain and polarization.
Surprisingly, these walls appear mechanically softer to a tip applied to the surface of the sample, even though they are not ferroelastic and thus do not separate domains with different strain textures.
Based on the experimental observations and a numerical simulation, we uncover that the mechanical interaction between the tip and domain wall localized phonons explains the apparent softening.
% \chapter*{Acknowledgements}
% \Include{Publications}
\setcounter{tocdepth}{1}
\tableofcontents
\mainmatter
\Include{Introduction}

% \Include{Rashba}
% \Include{GdMn2O5}
% \Include{Cr}

% \Include{Softening}
\appendix
\Include{DFT}
\Include{Wannier}
\Include{FEM}
\backmatter
% \bibliography{references}
\printindex
\end{document}
