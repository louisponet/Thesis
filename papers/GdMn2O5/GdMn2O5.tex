\chapter{Topological Multiferroic switching; coupling between Magnetism and Ferroelectricity in GdMn2O5}
\section{Introduction}
Efficient control of robust order has historically been one of the main technological drivers for research in condensed matter.
The magnetoelectric effect, i.e. the electric control of magnetic order or vice versa, is an obvious mechanism that could be exploited to achieve this control\cite{Spaldin2019,Khomskii2009,Fiebig2005,Fiebig2016,Cheong2007}.
However, the underlying physics are not particularly well understood and candidate materials are not plentiful.
One class of materials that has attracted much attention in recent years are the multiferroics, materials where multiple orders coexist, more specifically in this case: magnetic and ferroelectric order.

Here we specifically investigate GdMn$_2$O$_5$, an example of a type-II multiferroic\cite{Khomskii2009} with ferroelectricity arising due to a magnetic ordering that breaks inversion symmetry. In contrast to type-I multiferroics, the ferroelectricity and magnetic order arise from the same underlying physics, rather than from two different `subsystems` in the same material. This leads to an enhanced coupling between the two orders allowing for excellent cross order control (in the present case we investigate the control over the ferroelectric polarization by an external magnetic field).
Gd$^{3+}$ is special with respect to the other possible rare earths in the orthorhombic magnanites $R$Mn$_2$O$_5$, with $R$ being one of the rare earths, because it has a very isotropic electronic configuration (4$f^7$), i.e. there is no unquenched OAM. This means that the large spin (nominally $S=7/2$) can orient itself relatively freely to optimize the magnetic interactions with its neighboring Mn atoms.
This turns out to be one of the main reasons behind the very large electric polarization compared to other multiferroics, and even to other $R$Mn$_2$O$_5$ compounds.
The orthorhombic manganites all have a paramagnetic space group $Pbam$ \cite{Alfonso97a}, which gets lowered to $P_ab2_1a$ when the commensurate magnetic order locks in at $T_{N} \sim 33K$, with propagation vector $\bm{k} \sim (1/2, 0.0, 0.0)$, i.e. there is a unit cell doubling along the crystalline $a$ direction. Spins are ordered ferromagnetically along the crystalline $c$ direction. This magnetic transition goes hand in hand with a sharp anomaly in the dielectric constant $\varepsilon_b$, signalling the onset of the improper ferroelectric order along the $b$ direction \cite{Lee13}. When the temperature is lowered further, the polarization $P_b$ saturates to a maximum value of around $3600 \mu C/m^2$, which is the largest of all rare-earth magnanites, but still tiny compared to proper ferroelectrics like BaTiO$_3$ with $P \sim 2 \times 10^5 \mu C/m^2$.
The magnetic configuration features two AFM Mn chains per unit cell, that feature both $Mn^{3+}$ pyramids and $Mn^{4+}$ octahedra, as indicated by the light blue lines and purple polygons in Fig.~\ref{fig:GdMn2O5_unit_cell}. As in all other $R$Mn$_2$O$_5$ compounds, the Mn spins lie mostly along the easy axis of the Mn pyramids, making angles of $\pm 23.4^\circ$ with the $a$-axis.

%Even though the AFM intrachain superexchange interactions dominate over the AFM interchain interactions, there is nonetheless a geometric frustration due to the crystal structure, which can be seen most easily seen from the Mn pentagons surrouding the Gd atoms. This means that all AFM exchanges can not be satisfied at the same time, causing certain Mn bonds between the chains to have energetically unfavorable spin alignment. This leads to the first contribution to the ferroelectric polarization through Heisenberg exchange striction, lengthening bonds that have parallel spins and shortening those that have antiparallel ones.


This contribution is present in all orthorhombic manganites, but generally leads to a very tiny polarization \cite{Khomskii2009}.\lp{discuss about the electronic things in Khomskii and also the direct exchange optimizing FM bonds between Mn3+ and Mn4+}.


This also leads to a very high magnetoelectric effect, because the susceptibility of the large Gd spins means that they can be relatively easily switched by an external magnetic field, in turn causing a large variation in the polarization.  
\section{GdMn$_2$O$_5$}
Due to the Goodenough-Kanamori rules\lp{citation!} these chains are strongly coupled AFM, leading us to assign one AFM order parameter for each ($L_1$ and $L_2$), this amounts to taking the rigid spin approximation in each chain. However, due to the same rules, the chains also couple AFM between eachother. When considering the Mn pentagons formed by the two chains around each of the Gd atoms, it becomes clear that there is a magnetic frustration in the system. This will cause the chains to cant a little to be more perpendicular to eachother which can then lead to an energy gain through small canting. This frustration breaks inversion symmetry and leads to the first contribution to the ferroelectric polarization through magnetostriction. However, due to the spins of the Gd atoms, and their strong AFM coupling to the surrounding Mn atoms, there is a second contribution to the ferroelectricity. This happens because the Gd atom will move towards the closest tetrahedrally coordinated Mn atom in an effort to optimize the super-exchange energy gain with it. These two contributions add up to a polarization of around 3600 $\mu C/m^2$, which is tiny compared to proper ferroelectrics, but enormous for multiferroics. Lastly due to the size of the spin from the half-filled f-shell Gd atoms (nominally $S=\frac{7}{2}$), dipolar effects may play a role, and each of the Gd spins inside the unit cell need to be considered separately. 
\begin{figure}
	\IncludeGraphics{unit_cell}
	\caption{\label{fig:GdMn2O5_unit_cell}}
\end{figure}

\section{Theory}
In light of the above considerations we use a model with two antiferromagnetic order parameters signifying the two orientations of Mn chains, and eight classical spins that describe the different Gd atoms. The Hamiltonian then takes the form:
\begin{align}
    \label{eq:GdMn2O5_1}
    H=&\Gamma(\mathbf{L}_1\cdot \mathbf{L}_2)^2
    -\sum_{\alpha}\chi^{-1}((\mathbf{H}\cdot \mathbf{L}_\alpha)^2-H^2)\\
    &-K_L\sum_\alpha(\mathbf{L}_\alpha\cdot \mathbf{n}_\alpha)^2\\
    %\sum_{i,j}J_{ij} \mathbf{S}_i\cdot \mathbf{S}_j 
    &+\frac{1}{2}(g \mu_B)^2\sum_{i\neq j}\left(\frac{\mathbf{S}_i\cdot \mathbf{S}_j}{r_{ij}^3}-3\frac{(\mathbf{S}_i\cdot \mathbf{r}_{ij})(\mathbf{S}_j\cdot \mathbf{r}_{ij})}{r_{ij}^5}\right)\\
    &-\sum_i\left( K_S(\mathbf{N}_i\cdot \mathbf{S}_i)^2 + g\mu_\mathrm{B} \mathbf{H} \cdot \mathbf{S}_i\right) + \sum_{i,\alpha}V_{i\alpha}\mathbf{S}_i\cdot \mathbf{L}_\alpha,
\end{align}

where the first term originates from the competition of the interchain exchange $J_\perp$ and the intrachain AFM exchange $J_\|$ \cite{Sushkov2008}, with $\Gamma\sim\frac{J_{\perp}^2}{J_\|}>0$ determined by the energy gain on the spin canting, possible when $L_1$ and $L_2$ are not collinear. The term with $\chi$ represents the gained Zeeman energy when the Mn spins are slightly canted from the purely AFM order inside the chains; terms with $K_L$ and $K_S$ refer to easy-axis anisotropy constants of Mn and Gd spins, respectively.
Anisotropy axes for $L_1$, $L_2$ are unit vectors chosen to be aligned with the threefold axes of the tetrahedrally coordinated Mn atoms, i.e. $n_\alpha=\pm23.4^\circ$ respectively. The anisotropy axes for the Gd atoms, $N_i$, are also unit vectors that alternate as $\pm 12^\circ$. The third line describes the dipole-dipole interactions between Gd ions, which were restricted to five nearest neighbors. It was checked that including further neighbors up to $8.5\angstrom$ away didn't lead to qualitative changes of the results. The Heisenberg exchange constants $V_{i,\alpha}$ describe Mn-Gd interactions, where $V_{i,\alpha}=v_1$ for the exchange constant between the Gd and the nearest (tetragonally coordinated) Mn atom and its chain, and $v_2$ the exchange with the other Mn chain. The model parameters used here are $J_\perp = 1.89$~meV, $J_\parallel = 26.67$~meV, $K_L = 5.27$~meV, $K_S = 0.2$~meV, $v_1 = 3.33$~meV, $v_2 = 0.15$~meV. The electric polarization $P_b$, induced by the Heisenberg exchange striction, is given by
\begin{align}
    P_b=&\alpha (\vec{L}_1\cdot \vec{L}_2)+\nonumber\\
    &(\vec{S}_1-\vec{S}_5)(\beta \vec{L}_2 + \gamma \vec{L}_1) +\nonumber\\
    &(\vec{S}_2-\vec{S}_6)(\beta \vec{L}_1 + \gamma \vec{L}_2) +\\ 
    &(\vec{S}_3-\vec{S}_7)(\beta \vec{L}_2 - \gamma \vec{L}_1) +\nonumber\\ 
    &(\vec{S}_4-\vec{S}_8)(\beta \vec{L}_1 - \gamma \vec{L}_2)\nonumber,
\end{align}
with the model parameters used to fit the experimental data, $\alpha = 0.06\:\mu$C/cm$^2$, $\beta = 0.04\:\mu$C/cm$^2$, $\gamma = 0.06\:\mu$C/cm$^2$. The results are shown in Fig.~\ref{fig:theory_Gd1}

\begin{figure*}[t]
    \centering
    \IncludeGraphics[width=\linewidth]{fig_theory.png}
    %\includegraphics[width=\linewidth]{fig/L1L2_heatmap.png}
    \caption{\label{fig:theory_Gd1} (a-c) Evolution of electric polarization $P_b$ during the magnetic field sweep cycle for various magnetic field orientations. The four states are marked by the numbers, and the switching paths indicated by insets and arrows next to the graphs. The colors are used to indicate sequential sweep branches. $\nu$ signifies the winding number of each switching regime.
    (d-f) Trajectories (in yellow) of the AFM order parameter orientations $(\phi_{L_1},\phi_{L_2})$ through the field sweep cycles in different regimes. The color map shows the energy landscape at an intermediate field inside the hysteresis region. 
    (g) NEB showing the evolution of transition barriers between states 1,2,3,4 as the magnetic field aligned with with magic angle is swept through the hysteresis region. The plots are shifted vertically for clarity. The curve colors encode the corresponding magnetic field strength. The trajectory of the energy minimum due to field sweeps is shown with blue balls and arrows.
    Coordinated changes of the state energies and barrier asymmetry with magnetic field enable the topological behavior. 
    (h) Spin configurations of the four states, labeled $1\dots 4$ in (a).
    (k) Schematic evolution of the barriers connecting the state 2 to states 1 and 3 away from $H^*$. Saddle point states are denoted by $\tilde{1}$ and $\tilde{3}$.
    }
\end{figure*}

We proceed by identifying the 2 low and 2 high $P_b$ states in Fig.~\ref{fig:theory_Gd1}(b) as 1,3 and 2,4 respectively. The magnetic configuration of these states is displayed in panel (h) of the same figure. We then perform a nudged elastic band calculation between these states at different magnetic field strengths while remaining in the 4-state hysteresis region, resulting in the evolution of barriers between these states as shown in panel (g), where the color coding is used to denote the magnetic field strength, and the energy graphs are offset for clarity. The arrows and blue balls denote the evolution during the double magnetic field sweep. As expected, there are two degenerate minima at each value of $H$, at low field these are located at states 1 and 3, then as the field is ramped up they move to favor states 2 and 4. We see that during the sweep the barriers between the different states evolve asymmetrically. As we start in state 1, when the field is ramped up first the barrier towards state 2 decreases faster than the one towards state 4, causing the system to move from 1 to 2. Then as the magnetic field is lowered, the barrier from 2 to 3 increases slower than from 2 to 1. When ultimately state 2 goes from being metastable to a saddle point, the state spills towards state 3, and so on.
In order to try and understand where the asymmetric evolution of the barriers comes from, we look at the particular $H$-field strength where the barriers between state 2 and 1, and 2 and 3 are the same height. This is highlighted by the red dashed box around the blue graph. The inset of panel (k) provides a zoom on this situation. The states on top of the barriers are denoted by a tilde over the state towards the barrier is oriented, i.e. $\tilde{1}$ signifies the state on the top of the barrier from state 2 towards state 1. We then perform a Taylor expansion in terms of $H$, around $H^*$:
\begin{equation}
	F(H) = F(H^*) + \left.\frac{\partial F}{\partial H}\right\rvert_{H=H^*} (H-H^*) + ...,
\end{equation}
for each of the states and subtract the result. Using $\frac{\partial F}{\partial H} = M$ we find
\begin{equation}
	F_{\tilde{1}}(H) - F_{\tilde{2}} = (M_{\tilde{1}} - M_{\tilde{2}}) (H - H^*).
\end{equation}
\lp{show how to calculate M?}
This means that to first order the evolution of the barrier asymmetry is given by the difference magnetization of the states on top of the barriers. Indeed we find that the magnetizations are different from our simulations, confirming that this is at least a part of the reason for the asymmetric evolution of the barriers. Moreover, due to the symmetry of the system, this asymmetry is opposite when the field is swept up, as compared with when the field is swept down. This causes the unidirectional movements through the four states.

\section{Simplified Model}
Having found a description for the situation in the complicated material GdMn$_2$O$_5$ using the model \ref{eq:GdMn2O5_1}, one may wonder what the minimal requirements are to have a similar four-state behavior where the spins rotate 360$^\circ$ while the applied field only oscillates along a single axis.
We look to the spin configurations of Fig.~\ref{fig:theory_Gd1}(h) for inspiration. It is clear that, although necessary for the $P_b$ behavior, it seems that the chain with Mn moments most parallel to the applied field does the full rotation, while the other chain merely osciallates around its starting position. The Gd moments seem to follow the behavior of the chain they are most strongly coupled to. This warrants an attempt to explain the observed behavior using only a single chain with its Gd moments.   
In the particular case where the magic angle is along $\alpha = +10^\circ$, we keep $L_1$, $S_2$, $S_3$, $S_6$ and $S_7$ as the variables in the model. We also assume that the easy axis of Gd is not that important.
This leaves us with the following Hamiltonian, which we will split up in two parts, one with the dipolar terms $H_{dip}$ and one with all the other terms $H'$:
\begin{align}
	H &= H' + H_{dip} \\
	H' &= J_1(\mathbf{S}_2 + \mathbf{S}_3 - (\mathbf{S}_6 + \mathbf{S}_7))\cdot\mathbf{L}_1 - g \mu_b (\mathbf{S}_2 + \mathbf{S}_3 + \mathbf{S}_6 + \mathbf{S}_7) \cdot \mathbf{H} \\
	&+ K_L (\mathbf{L} \cdot \mathbf{n}) \\
	H_{dip} &= \frac{1}{2}(g \mu_B)^2\sum_{i\neq j}\left(\frac{\mathbf{S}_i\cdot \mathbf{S}_j}{r_{ij}^3}-3\frac{(\mathbf{S}_i\cdot \mathbf{r}_{ij})(\mathbf{S}_j\cdot \mathbf{r}_{ij})}{r_{ij}^5}\right).
\end{align}

Due to the symmetries of this model and the geometry of the material we end up with two copies of one spin in the first half of the magnetic unit cell and one in the second half of the unit cell, with the chain inbetween (see Fig.~\ref{fig:GdMn2O5_simple_model}).
\begin{figure}
	\IncludeGraphics{double_cell.png}
	\caption{\label{fig:GdMn2O5_simple_model} Simplified single chain model.}
\end{figure}

This causes $S_2 = S_3$ and $S_6 = S_7$, and thus allows us to look at a single one of the two copies. Again keep only nearest neighbor dipolar terms, and assuming no anisotropy for the Gd spins, we end up with
\begin{align}
	H' &= 2J_1(\mathbf{S}_3 - \mathbf{S}_6)\cdot\mathbf{L}_1 - 2g \mu_b (\mathbf{S}_3 + \mathbf{S}_6) \cdot \mathbf{H} \\
	&+ K_L (\mathbf{L} \cdot \mathbf{n}) \\
	H_{dip} &= (g \mu_B)^2\left(\frac{\mathbf{S}_3\cdot \mathbf{S}_6}{r_{36}^3}-3\frac{(\mathbf{S}_3\cdot \mathbf{r}_{36})(\mathbf{S}_6\cdot \mathbf{r}_{36})}{r_{36}^5}\right).
\end{align}

\printbibliography
