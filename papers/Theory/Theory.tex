\chapter{Theory}
\section{Spin-orbit coupling}
\section{Wannier Functions}

	In a lot of the work presented in this Thesis we construct models to describe the physics that result in a certain effect in real materials.
	These models are most of the time defined in terms of a limited set of local atomic-like orbitals, since these offer the most intuitive understanding most of the time, and the biggest contribution to particular effects can often be traced back to only a limited subset of all the orbitals.

	If we want to apply these models to real materials to get not only qualitative, but also quantitative results, we need to perform a fit to either experimentally obtained values, or first-principles simulations.
	The most feasible way is usually the latter, which requires us to find a bridge between the plane waves that are used in a lot of first-principles DFT codes, and the localized basis set of that the model is built from.

	Wannier functions offer a rigorous and flexible solution to this question, rewriting the Bloch functions as a discrete Fourier transform of (exponentially) localized, cell-periodic wavefunctions.

	This construction is flexible in the sense that it is not unique. Bloch functions have a gauge freedom at each k value, which in turn will have an effect on the Wannier functions.
	One gauge that is commonly used is the one that localized the functions as much as possible, leading to the so-called Maximally Localized Wannier Functions (MLWF).
	Other gauges can be chosen such that the obtained functions have certain symmetries or have maximum similarity to atomic orbitals.


