\documentclass[a4, UTF8]{article}
\usepackage{amsmath}
\usepackage{bm}
\begin{document}
\title{List of remarks and minor corrections\\
\large ``Interplay between Complex Orders in Functional Materials''\\
by\\
Dott Louis Ponet, Scuola Normale Superiore}
\date{}
\maketitle
{\bf Page 10:} The candidate states: ``Without loss of generality, we choose this point to be the gamma
point.'' This does not appear to be obvious, because not all T-invariant points are equivalent. Can
this be better explained?

This is true. What is meant here is that for the purpose of the derivation only $\bm k_r = \bm k - \bm \bm k_0$ matters. Whether $\bm k_0 = \bm 0$ or another T-invariant point changes what the $|u_n, \sigma>$ are in Eq.(2.14).
\end{document}
