\chapter{Coupling between spin and strain density waves}
\section{Introduction}
In the previously discussed situations, the coupling between the orders was relatively static.
In the case of GeTe the polarization influenced the sign of the spin splitting, but was assumed static in the description.
While there was an element of a time evolution in the GdMn$_2$O$_5$ it was assumed that the magnetic field varied at a slow enough rate such that both the polarization as the magnetic order were always at equilibrium.
Similarly, while the experiments performed on BaTiO$_3$ where dependent on a resonance frequency, this was assumed to not influence the material itself and thus the description was again static.
New physics arise, however, when two orders with different timescales for their dynamics are coupled.
This is the case in the material we study here, Chromium, an itinerant antiferromagnetic metal \cite{Kulikov1984,Fawcett1988}.
It is the hallmark example of a material where a spin density wave (SDW) develops due to local repulsive interactions of the electron gas, combined with a nesting of the fermi surface.
Due to magnetostriction, this spin density wave then causes a periodic lattice displacement (PLD) which decreases the bondlengths between the spins that have large magnitude.
The SDW is formed by electrons and thus thermalizes on very quick timescales when a perturbation is applied\cite{Nicholson2016}.
The coupled PLD, however, denotes the movement of atoms and thus has much slower dynamics.
With ultrafast pump-probe experiments that have a below picosecond resolution, these PLD dynamics can be studied and, more imporantly, controlled very precisely through excitation of the much faster responding SDW.
Among other things, this allows for the suppression of the intermediate state and reproducing adiabatic transitions on much faster timescales which is termed ``shortcuts to adiabaticity'' in quantum technology \cite{Torrontegui2013,Deffner2014,Zhou2017}.
This also allows to apply a specific set of pulses in order for the PLD oscillation to follow a particular envelope function, as will be shown theoretically below.

\section{Experimental methods}
Before focusing on the theoretical description of the coupling between the SDW and PLD, we take a look at the experimental techniques employed by A. Singer's group and the results that we are seeking to reproduce.
A thin film (28 nm) of Cr was used, supporting around seven periods of the SDW with a N\'eel temperature $T_N$ of 290 K, slightly lower than in the bulk where $T_N\approx$ 308 K.
The film was illuminated by two sequential 40 fs optical pulses at times $\tau_1$ and $\tau_2$, heating up the electronic subsystem and the SDW which is a part of it.
Due to magnetostriction, this SDW is coupled with a PLD which is probed through the Bragg peaks by an x-ray free-electron laser (XFEL)\cite{Singer2015}.
XFELs also allow for higher peak selectivity (narrow and intense satellite Bragg peaks), such that it allows for measurements performed on thin films even deposited on thick substrates.
The scattering intensity of the Bragg peak is directly proportional to the magnitude of the PLD, offering both phase and amplitude information of the PLD oscillation that occurs when the SDW is (partially) melted by the optical pulses.
The high resolution ultrafast measurements that the XFEL enables (fs timescales), allows to follow the evolution of the PLD in time.
The acoustic phonon associated with the PLD oscillation has a wavevector normal to the material surface and an oscillation period of approximately 450 fs.
The heat exchange rate between the lattice (or bath) and the electrons is quite high, leading to a thermalization between the two subsystems within a picosecond after heating.
Even though some heat is deposited in the system, and the heat capacity is not infinite, the final temperature stays below the N\'eel temperature leading to a recovery of the SDW and eventually the PLD, as can be seen by the non-zero oscillation equilibrium at the end of the measurements in Fig.~\ref{fig:Cr_experimental}.
Further thermalization of the Cr lattice to the substrate occurs on the nanosecond timescale, out of the scope of our measurements.
Due to the slight increase in temperature of the system (around 45 K), this equilibrium amplitude of the PLD is slightly lower after the pulses, compared with before.
The damping time of the PLD is around 3 ps.
It is important to realize that the SDW order is not directly accessible through these kinds of measurements, so we are effectively studying the effect of exciting one order parameter through the reaction of the other by the coupling between them.

The most interesting observation is that by changing the timing of the second pulse $\tau_2$, the dynamics of the PLD can be controlled to a high degree.
In Fig.~\ref{fig:Cr_experimental}(b-c) it can be seen that the oscillation amplitude in the first one is almost completely destroyed by the second pulse, whereas in panel c it is increased slightly.
Panel a of the same figure shows a more complete visualization of this effect, where the horizontal bands with low amplitude signify the destructive interference and those with high amplitude the constructive situation.

The maximum PLD amplitude that is reached in the experiments, after the second pulse is 150\% of the original amplitude.
This is significantly higher than the excitation amplitudes associated with the conventional displacive excitation mechanism, where the ratio is about one \cite{Singer2015,Zeiger1992}. 
When the second pulse arrives before the SDW had time to cool down, the maximum oscillation amplitude is lowered as can be seen from the horizontal band around $\tau_2 - \tau_1 = 0$ ps in Fig.~\ref{fig:Cr_experimental}(a).
As will be further discussed below, this signifies that the SDW momentarily exceeds $T_N$, and additional heating does not contribute to a larger excitation of the PLD phonon.

Experiments performed at higher fluence, i.e. such that the SDW gets heated above $T_N$ from a single pulse, show that the second pulse does not impact the oscillation of the PLD, as can be seen from Fig.~\ref{fig:Cr_experiment2}(c).
If control is the goal, it is thus important that the SDW is kept below $T_N$.
Moreover, the SDW order parameter varies the most with temperature when it is close to the critical point, increasing the pulse efficiency.
The XFEL also shows the absence of topological defects by the absence of a widening of the satellite peak associated with the PLD, as compared with the peaks of the material itself, supporting the assumption that through the use of a thin film, the optical pulses excite the material homogeneously. 
\begin{figure}
\IncludeGraphics{Experimental}
\caption{\label{fig:Cr_experimental}{\bf Controlled enhancement and destruction of the excited state.} {\bf (a)} A laser pulse excites the system by destroying the magnetic order; a second pulse either excites the released phonon further or stops the excitation. Schematic energy surfaces show the PLD state as the orange point. Inset: x-ray scattering from periodic atomic displacement is detected on a fringe of the main peak from the crystalline film. {\bf (b-c)} Amplitude of the PLD in two extreme control cases. Solid lines are experimental data (empty circles) - connected. The dashed red line marks the time of second pulse arrival. {\bf (b)} $\tau_2 - \tau_1 = 620$ fs, {\bf (c)} $\tau_2 - \tau_1 = 845$ fs. Insets show the Laue fringe with the satellite peak at maximum and minimum PLD values.}
\end{figure}
\begin{figure}
\IncludeGraphics{experimental2}
\caption{\label{fig:Cr_experimental2}{\bf Experimental maps of the excited state depending on probe and pump delay.} { \bf (a)} Map of the PLD amplitude before and after excitation by two pulses with a fluence of 1.45 mJ/cm$^2$. {\bf (b)} Same as (a) but with the second pulse twice less intense than the first. {\bf (c)} Magnitude of the PLD in ``enhancement'' and ``suppression'' conditions at laser pulse fluence of 9.5 mJ/cm$^2$ for the first and 4.7 mJ/cm$^2$ for the second. The dashed line marks the arrival time of the second pulse, $\tau_2 - \tau_1$ = 1295 fs for the top panel and 1065 fs for the bottom.}
\end{figure}

\section{Theory}
\lp{some explanation of Peierls instabilities etc, the things that cause the SDW in the first place?}

% To understand how modulation of bond lengths changes the magnetic exchange between spins of the constituent ions, it is instructive to recall where the Heisenberg exchange form comes from, namely the Hubbard model:
% \begin{equation}
% 	H = \sum_{<i,j>}t_{ij} c_{i,\sigma}^{\dag} c_{j,\sigma}  +  U \sum_i U n_i^{\uparrow} n_i^{\downarrow}
% \end{equation}
% where $<i,j>$ denotes nearest neighbors, and $t_{ij}$ the hopping between them, and $U_i$ the on-site coulomb repulsion active when the on-site orbital is occupied by more than one electron. 
% In the usual case, $U >> t_{ij}$ and thus a Shrieffer-Wolf transformation can be performed to separate the high energy from the low energy subspace. This amounts to performing a second order perturbation theory, allowing the Hubbard model to effectively be rewritten in terms of the on-site spins, leading to the well-known Heisenberg model:
% \begin{equation}
% 	\label{eq:Heisenberg}
% 	H = \sum_{<i,j>}J_{ij} \mathbf{S}_i \cdot \mathbf{S}_j,
% \end{equation}
% with the coupling constant (often referred to as the magnetic exchange) $J_{ij}=t_{ij}^2/U$. Since hopping parameters are a result of the overlap between orbitals on neighboring sites, it is clear that they depend on the length of the bond $r$ between them, i.e. $t_{ij} \sim r$ to first order. 
% In materials where SDW are formed, this magnetostriction will lead to PLD leading to strain waves. 

As was already aluded to, we can describe the effect of the photon pulses by considering two subsystems with two temperatures.
The first is the electronic subsystem, that thermalizes on a 100 fs timescale, as was shown by high-resolution ARPES measurements~\cite{Nicholson2016}, and can thus be considered to be always thermal on the picosecond timescales that we are interested in, allowing us to assign a temperature to it.
The SDW is formed by electronic states and thus belongs to the electronic subsystem~\cite{Nicholson2016}, with the same temperature $T_L$.
The second subsystem is the lattice (phonon modes excluding the mode associated with the PLD) which we denote as the bath which is always assumed to be thermal, with temperature $T_b$.
After the electronic system absorbs the heat from the pulses, it will exchange heat with this bath, cooling it down within one picosecond.
This process can be described by the so-called two temperature model:
\begin{align}
	\label{eq:Cr_twotemp}
	C_L \dot{T}_L &= -k(T_L(t) - T_b(t)) + Q_ph(t) \\
	C_b \dot{T}_b &= -k(T_b(t) - T_L(t)),\nonumber
\end{align}
with $k$ the heat transfer rate, $C_L$ and $C_b$ the heat capacities of the electronic degrees of freedom and bath, respectively.
The dots signify time derivatives.
As one could expect, we find that the heat capacity of the bath is larger than that of the electronic system by an order of magnitude\lp{ref}.
Finally, the heat injected by the pulses is modelled by a gaussian $Q_ph(t) = A e^{\frac{-(t - t_0)^2}{\tau^2}}$, with $A$ the strength, $\tau$ the duration and $t_0$ the time delay.

The changes to $T_L$ through heating and subsequent cooling affects the amplitude SDW order parameter $L$ and indirectly the one describing the PLD $y$, as described by a Landau-type theory~\cite{Khomskii2010}.
These order parameters are related to the Fourier component of the SDW with wavevector $q$, i.e. $L = S_q$, and to the acoustic phonon amplitude $y = u_{2q}$.
The phonon mode is the second harmonic ($2q$) of the SDW because the magnetostriction acts on the magnitude of the spins, and not the phase, so that during one period of the SDW oscillation two periods of the PLD occur.
This leads to the following expression for the total free energy:
\begin{equation}
	\label{eq:Cr_landau}
	F(L, y, T_L) = \frac{\alpha}{2}(T_L - T_c) L^2 + \frac{\beta}{4} L^4 - g L^2 y + \frac{\omega_0}{2} y^2 + \frac{b}{4} y^4,
\end{equation}
where $L$, $y$ and $T_L$ are the time dependent variables.
The double well potential that leads to the SDW phase transition is characterised by $\alpha$ and $\beta$, with the temperature of the SDW given by $T_L$ and critical temperature $T_c$ below which the SDW order sets in.
The magnetostrictive coupling between the two order parameters is described by term with $\gamma$.
Only even orders of $L$ appear in the free energy, since the energy is time reversal even, but $L$ is time reversal odd.
The PLD order parameter $y$ has a zero equilibrium value without the presence or interaction with $L$, since it is not the primary order parameter.
The fourth order term $\frac{b}{4}y^4$ is only included to provide a better fit to some of the anistropic features of observed by the experiment, not to bound the energy potential in terms of $y$, as would be required if $y$ was the primary order parameter with a negative second order term, like in the case of $L$.
It is the interaction with the primary order parameter $L$ that provides the ``force'' to move $y$ up in its own potential, leading to the nonzero equilibrium value.
While this sounds trivial, this concept is the key to understanding the observed physics.

The time evolution of the system can be described using the Langrangian:
\begin{equation}
    \mathcal{L}(L, y, \dot{L}, \dot{y}, t) = \frac{m_L \dot{L}(t)^2}{2} + \frac{m_y \dot{y}(t)^2}{2} - F(L, y, t),
\end{equation}
with associated Euler-Lagrange equations:
\begin{align}
    \frac{\partial \mathcal{L}}{\partial L} - \frac{d}{dt}\frac{\partial \mathcal{L}}{\partial \dot{L}} &= \gamma_y \dot{L}\\
    \frac{\partial \mathcal{L}}{\partial y} - \frac{d}{dt}\frac{\partial \mathcal{L}}{\partial \dot{y}} &= \gamma_L \dot{y}\nonumber
\end{align}
where the $\gamma$ denote the damping parameters for both order parameters.
Substituting Eq.~\ref{eq:Cr_landau} leads to 
\begin{align}
	\label{eq:Cr_euler_lagrange}
	    m_L \Ddot{L} &= -\alpha(T_L-T_c)L - \beta L^3 - \gamma_L \dot{L} + 2g L y \\
	    m_y \Ddot{y} &= -\omega_0^2 y  - b y^3 - \gamma_y \dot{y} + gL^2.\nonumber
\end{align}
These equations, together with Eqs.~\ref{eq:Cr_twotemp} describing the temperature evolution of the SDW ($T_L$) under influence of the optical pulses, fully describe the dynamics that are experimentally observed.
One obvious remark can be made here, in that to be completely exact, the energy dissipated through the damping terms in Eq.~\ref{eq:Cr_euler_lagrange} should be absorbed into the bath and thus influence $T_b$. However, the case can be made that since this is only a single mode, its contribution to the heating of the bath will be negligible compared with the one due to the thermalization of all the electronic degrees of freedom.

\section{Methods}
To solve the time evolution of $L$ and $y$ throught the differential equations in Eqs.~\ref{eq:Cr_euler_lagrange} we used the numerical integration methods implemented in the \href{https://github.com/SciML/DifferentialEquations.jl}{DifferentialEquations.jl} package~\cite{rackauckas2017differentialequations}. More specifically the Tsit5 algorithm was used, which has adaptive timestepping to capture the sharp pulses.
Originally the dynamics were fully solved both for $L$ and $y$, but it was found during the fitting process that the dynamics of $L$ are significantly faster than those of $y$, i.e. both the ``mass'' of the SDW order is orders of magnitude smaller than that for $y$ parameter, and the parameters describing the Landau free energy potential are orders of magnitude larger. This all leads to the SDW almost perfectly tracking its instantaneous minimum on the timescales that are of interest.
Solving dynamics with significantly different timescales is in general hard from the numerical point of view, and while there are other ways around this, we opted to take $m_L = 0$ and use the instantaneous minimum in the equation describing the dynamics of $y$.
This minimum is found by minimizing the Landau free energy in Eq.~\ref{eq:Cr_landau} for a given $y$ and $T_L$ in terms of $L$ such that $\frac{\partial F}{\partial L} = 0$, leading to:
\begin{equation}
	L(t) = \pm \sqrt{\frac{- \alpha (T_L - T_c) + 2 g y}{\beta}}.
\end{equation}

The starting temperature of the bath was fixed at 115 K, and $T_c$ was observed to be at 290 K, consistent with previous observations that in general the N\'eel temperature of a thin film is lower than that of the bulk material (bulk Cr has a N\'eel temperature of $\approx$ 310 K).
Most parameters of the model described above were not known a priori and thus had to be fitted to the experimental measurements. To aid with the fitting, judicious starting values could be chosen for some parameters. For example, it was known that the pulse width $\tau$ was under 100 fs, the oscillation frequency of the PLD $\omega_0 \approx$ 14 or equivalently a period of around 450 fs. It was also known that in general the second pulse had a fluency of around 80\% that of the first, and we also chose the initial heat capacities for the bath and electronic degrees of freedom to have a ratio $\frac{c_b}{c_L}$ close to 7, which was known from previous experiments \lp{citation}.

For each set of trial parameters, the time evolution of the system was solved on an interval of -2 ps to 8 ps, where the lower bound is chosen so that the numerical integration starts from a completely equilibrium initial condition.
This is needed because when the sharp pulse arrives around 0 ps, some energy already enters the system slightly before 0 ps due to the gaussian shape.
The error of the solution $\tilde{x}$ w.r.t. the experimental measurements $x$ is then the mean square sum $err = \sum_{i=1}^n \frac{(x_i - \tilde{x}_i)^2}{n}$, where $i$ denote the measurement points.
The numerical optimization was done through the \href{https://github.com/JuliaNLSolvers/Optim.jl}{Optim.jl} package ~\cite{mogensen2018optim}, where it was found that the Nelder Mead simplex algorithm \cite{Nelder1965} works best for this very non-linear problem.

\section{Results}
The experimental results we use as a basis to fit our model to are shown in Fig.~\ref{fig:Cr_experimental}.
In the numerical model, we found in earlier trials that the dynamics of the SDW order parameter $L$ is orders of magnitude faster than the ones from the PLD $y$, as expected.
This can also be seen from Fig.~\ref{fig:Cr_energy_surfaces} since the energy potential is a lot flatter for $y$ than for $L$, leading to a slower time evolution.
This difference in dynamics makes it extremely hard to solve the differential equations numerically, we therefore assumed that at each timestep the $L$ order parameter is in equilibrium in its instantaneous energy potential that depends on time through $T_L$.
This is equivalent to the limit of the mass $m_L$ in Eq.~\ref{eq:L_diffeq} going to zero.
The value of $L$ at a given $T_L$ and $y$ can be found by minimizing the Landau free energy \ref{eq:Cr_landau} 
\begin{equation}
	\label{eq:Cr_L0}
	L_0 = \pm \sqrt{\frac{\alpha (T_L - T_c) + 2 g y}{\beta}}.
\end{equation}
This eliminates the need to evaluate Eq.~\ref{eq:L_diffeq}, instead using Eq.~\ref{eq:Cr_L0} to evaluate $L$ in the partial differential equation for the evolution of $y$.

We then took eleven representative experiments, which can be thought of as horizontal slices of Fig.~\ref{fig:Cr_experimental}(a), in order to fit the model parameters to get the best total fit accross all datasets, allowing only the pulse fluence to fluctuate between sets.
The parameters are
\begin{align}
	\alpha &= 6039, \beta = \SI{7.97e7}, g = 0.52, \gamma_L = 25.0, \\
	\omega_0 &= 14.1, b = \SI{3.38e8}, \gamma_y = 0.76,\\
	f &= 64.43, c_b = 3.58, c_L = 0.36, k = 1.29, \tau = 0.074
\end{align}

The results of this fitting procedure is shown in Fig.~\ref{fig:Cr_theoretical_fit}, showing an excellent agreement between the theory and experiment, accross the board.
\begin{figure}
\IncludeGraphics{exp_fits.pdf}
\caption{\label{fig:Cr_theoretical_fit} {\bf Comparison of theoretical fit vs Experiment.} {\bf (a-b)} Two examples of fits to constructive experiments. {\bf (c-d)} The same for destructive interference.}
\end{figure}

To get a deeper understanding of the underlying effect, we look at the evolution of the free energy surfaces for both order parameters, as shown in Fig.~\ref{fig:Cr_energy_surfaces}.
The characteristic double well potential for $L\neq0$ equilibrium is clearly visible, and as expected, when the pulses hit and $T_L$ increases in the term $\alpha(T_L-T_c)L^2$ of Eq.~\ref{eq:Cr_landau}, we see that the potential flattens causing the the minimum of $L$ to very quickly change, as discussed above.
This in turn causes the single-well potential of $y$ to shift as quickly.
An oscillation of $y$ will occur due to its relatively slow dynamics, and instantaneous shift of its energy landscape as $L$ varies.
While the temperature $T_L$ decreases again, $L$ and the minimum of the $y$ potential shift back towards the original equilibrium position.
The oscillation of $y$ remains for quite a long time while this shift is occurring since the damping is not that big (of the order of 4ps).

This presents us already with the first hint at the benefit of indirectly exciting the PLD through the SDW.
In a hypothetical system where the PLD is the primary order parameter, and no other coupled order parameters are present, the only possibility to excite a similar oscillation is to heat it up.
This would shift the minimum of the hypothetical energy potential to 0, and again due to the slow dynamics an oscillation would occur.
However, as with any normal oscillator, the amplitude of the oscillation would never exceed the starting value, as the thermalization, and restoration of the potential will occur on much longer timescales.
Here, however, due to the relatively fast thermalization of the SDW with the bath, the potential surface for $y$ recovers on a faster timescale than it would on its own, leading to a relative movement of the potential opposite to the movement of $y$ itself.
This relative movement transfers additional momentum to $y$ as the SDW is thermalizing and the potential is being restored, leading ultimately to a higher oscillation amplitude, even when only as single pulse is applied.
This was observed experimentally in previous measurements by A. Singer's group and reported in Ref.~\cite{Singer2015}. 

One final remark to make w.r.t. to the behavior after a single pulse, is that since it is the position or magnitude of $L$ which is the driver of the energy surface of $y$, the effect is most efficiently generated when $L$ is close to $T_c$ because a small change in temperature causes a large variation of the equilibrium position of $L$ as shown by Eq.~\ref{eq:Cr_L0} and corresponding familiar graph in Fig.~\ref{blabla}\lp{make an image of general sqrt(T-Tc) behavior}.    

Having multiple pulses that can rapidly change the potential surface for $y$ of course opens up many more possibilities, allowing for additional constructive interference (Fig.~\ref{fig:Cr_theoretical_fit}(a-b)) or, by shifting the potential surface in the same direction as the movement of $y$, complete destructive interference (see Fig.~\ref{fig:Cr_theoretical_fit}(c-d)).
This can be leveraged by applying a whole train of pulses in order to create any graph your heart desires.
Having this understanding, we continue by investigating what can be achieved with a longer pulse train.
\begin{figure}
	\begin{subfigure}{0.5\textwidth}
		\IncludeGraphics{energy_surface_L.pdf}
	\end{subfigure}
	\begin{subfigure}{0.5\textwidth}
		\IncludeGraphics{energy_surface_y.pdf}
	\end{subfigure}
	\caption{\label{fig:Cr_energy_surfaces} Time evolution of the energy surfaces of the order parameters.}
\end{figure}

First we discuss the strategy to achieve maximal amplitude.
The maximal change in the value of $L$, and thus the potential surface for $y$, can be achieved by heating it all the way to $T_c$.
It is then important to keep this initial shift of the potential for $y$ in place until $y$ crosses the minimum, converting as much potential energy into kinetic energy.
This can be achieved by heating $L$ to slightly above $T_c$, where the additional heat and finite cooling rate will keep the potential shift present until $y$ passes through the minimum, gaining the most momentum.
Fig.~\ref{fig:Cr_maximum_amp} demonstrates this by the red peaks of $T_L$ which signify that it is above $T_c$.
Exactly at this point $L$ should start to reform causing the potential surface of $y$ to start moving in the opposite direction to its movement, which in the end supplies it with additional momentum, as was described above in relation to the larger than 100\% relative amplitude of $y$ after one pulse.
Since the sign of $L$ does not matter for the position of the potential surface of $y$, it can only cause a shift in one direction, meaning that, similar to someone pushing a swing, the ideal intervals for the subsequent pulses are close to multiples of the period of $y$, if the goal is maximum oscillation amplitude.
The amount of periods depends on the cooling rate of $L$, and damping rate of $y$.
The former influences how much $T_L$ can cool back down within one period and thus the size of the maximum kinetic energy gain of $y$, while the latter is the main source of kinetic energy loss.From our simulations, unless the damping is really negligible, it is always best to supply a set of pulses at each oscillation period of $y$.
The results in Fig.~\ref{fig:Cr_maximum_amp} showcase this situation, where the heat capacity of the bath was artificially increased to infinity to mimic a possible strong cooled regime which would be ideal for maximizing the oscillation amplitude.
With the damping rate of $y$ taken as in the experiment, the maximal amplitude thus achieved is about 400\% the original value, quite a remarkable increase.  
\begin{figure}
	\IncludeGraphics{maximumamplitude.pdf}
	\caption{\label{fig:Cr_maximum_amp} {\bf Two schemes for achieving maximum amplitude.} {\bf (a)} One set of pulses is applied within each period. {\bf (b)} One set of pulses every two periods. The damping is clearly the limiting factor and warrants as many pulses per timeframe in order to achieve the maximum amplitude.}
\end{figure}

We can go one step further, however, and design a pulse train such that the oscillation amplitude follows a particular ``signal'' envelope function, demonstrating the incredibly high degree of control of the PLD that this system allows for.  
To limit the dimensionality of the manifold of possible solutions, we followed the ruleset:
\begin{itemize}
	\item Only one fluence can be used per pulse 
	\item Pulses are grouped in sets per period of oscillation
	\item The first pulse group only has a single pulse, purely for technical reasons
	\item The maximum allowed pulses per group is fixed
	\item The groups are then fitted sequentially, since the later pulses don't influence the oscillation caused by the earlier ones
\end{itemize}
Adhering to these rules, we tested this procedure on different envelope functions, showcased in Fig.~\ref{fig:Cr_control}.
Similar to before, the assumption of an infinite $C_b$ was made here, this is done so the equilibrium position of $L$ does not change, otherwise causing a general downwards slope of the oscillation of $y$ as see from the lower final equilibrium positions of $y$ in panels (a-b) of Fig.~\ref{fig:Cr_theoretical_fit}.
This showcases that with a carefully chosen pulse train we can achieve indirect, but optimal control of the PLD order parameter.
\begin{figure}
	\IncludeGraphics{fits.pdf}
	\caption{\label{fig:Cr_control} {\bf Two examples of optimal control.} {\bf (a)} One period of a sinusoid followed by a sawtooth with the same width and height. {\bf (b)} Gaussian followed by an exponential and inverse exponential.}
\end{figure}

\section{Conclusions}
This chapter has shown that the coupling between the SDW and PLD through magnetostriction in Chromium is another prime example of a cross-order control, where in this case the perturbation applied by optical pulses to the SDW leads to an excitation in the PLD mode.
Moreover, the slower dynamics of the PLD compared to the SDW opens up new behavior and additional dynamic control that would not be possible without the coupling.
If the PLD would have dynamics on the same timescale as the SDW, changing its potential through excitation of the latter would not be very impactful since it would follow the minimum of the potential as fast as the excited SDW would change it.

As demonstrated by both the experiments and the theory, this allows for either increasing the oscillation amplitude dramatically (up to 400\% the original value), or destroy it at a moment's notice, allowing for a shortcut to adiabaticity where the final state is at higher temperature than the original one.

Whether this can be exploited for technological means is probably a bit of a stretch, seen as XFELs are rather large. However, this understanding and demonstration of the perfect control of a couple order parameter could open the road to finding other similar situations with more technologically viable properties.   


\printbibliography

