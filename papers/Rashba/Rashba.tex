\newcommand{\Unkr}{u_n(\bm{k}, \bm{r})}
% \renewcommand{\eikr}{$e^{i\bm{k}\cdot\bm{r}}$}
\newcommand{\Eikr}{e^{i\bm{k}\cdot\bm{r}}}
\chapter{Spin-momentum locking in high spin-orbit coupled ferroelectrics \label{ch:Rashba}}
\section{Introduction}
% Bulk ferroelectrics with large atomic spin-orbit coupling allow for electric control of spin-polarized states~\cite{DiSante2013,Ishizaka2011,Kim2014, Liebmann2016, Krempasky2015}, allowing for the switching of the spin texture by an externally applied electric field. 
% The underlying mechanism is, however, not well understood.
% Note: Not sure we should really focus on technology, also it's not particularly clear it wants to use both spin and charge, rather than just spin.
% The research area of spintronics aims to supersede standard electronics by using both the spin and charge degree of freedom of the carriers for information processing and storage. %\sa{https://doi.org/10.1146/annurev-conmatphys-070909-104123}
%
The research field of spintronics aims to understand the behavior of spins inside materials, and translate this understanding into the ability to actively control these degrees of freedom for possible technological applications \cite{Joshi2016}.
Many devices have been theorized, for example, spin field-effect transistors (spin-FET)\cite{Datta1990}, and storage devices which utilize spin-current and associated spin-transfer torque to efficiently manipulate magnetic domains \cite{Kent2015}, both in ferromagnets \cite{Nunez2011}, and antiferromagnets \cite{Jungwirth2016}.
In spite of fundamental interest and potential for applications, the actual realization of these spin-FET devices has been rather elusive.
One of the main culprits for this limited success is that the envisioned devices require very granular, ideally electric, control of the spin.

This is not straightforward, since the only direct coupling between the spins of the current carriers and the electric field is through relativistic effects. When an electron moves through an electric field, this field is rather perceived as a magnetic field, in the electron's co-moving frame. This magnetic field can be written as
\begin{equation}
	\bm{B} = - \frac{\bm v \times \bm E}{c^2},
\end{equation}
where $\bm v$ denotes the velocity of the electron, $\bm E$ the electric field, and $c$ the speed of light.
The spin of the electron will interact with this magnetic field through the Zeeman energy $H_Z = -\frac{g_s\mu_B}{\hbar} \bm{\sigma}$, where $g_s \approx 2$, $\mu_B$ is the Bohr magneton and $\bm{\sigma}$ is the spin of the electron.
The spins will thus start to precess around $\bm B$ created by the applied $\bm E$, and eventually align with it.
However, in nearly all real materials, the speed $\bm v$ of the current carriers is so low, and the applied electric fields so small, that the $c^2$ in the denominator completely overpowers the numerator, leading ultimately to a tiny effect.
One class of materials for which the carrier speeds are relatively high are the topological insulators \cite{Kane2005a,Novoselov2005,CastroNeto2009,Fu2007,Fu2006,Pesin2012}.
In these materials, a peculiar band structure effect leads to Dirac cones and Dirac points. They get their name from the necessity of using the Dirac equation in order to accurately describe the relativistic behavior of electrons (and holes) at these points.
It turns out that due to the band structure peculiarities, electrons (and holes) appear to have zero effective mass leading to a large Fermi velocity, on the order of \SI{10e6}m/s$^2$ for graphene \cite{Novoselov2005}.   
However, even in this case, the influence of external electric fields on the carriers' spin is extremely limited.
Indeed, for a field strength of 1 V/nm, the size of the magnetic field $\bm B$ that results from Eq.~\ref{eq:Rashba_B} is only on the order of \SI{1e-2} T, leading to a Zeeman energy of the spins of only \SI{1e-3} meV.
This is all to say that such a direct electric control of the spins is quite infeasible.

Leveraging the properties of particular materials, on the other hand, does lead to the sought after electric control, albeit indirectly.
One class of such materials are the ferroelectric semiconductors with large atomic spin-orbit coupling (SOC) \cite{Picozzi2014,DiSante2013,Ishizaka2011,Kim2014}.
Spin-orbit coupling is essentially identical to the previously discussed effect.
The ``atomic'' denotion refers to the manifestation of this effect due to the potential (and electric fields) around the atomic cores.
The atomic potentials are much greater than those from applied external fields, and the electrons have a much higher momenta close to the ionic cores.
Consequently, the spin-orbit coupling in atoms can lead to significant consequences and is one of the main drivers of the effect discussed in this Chapter.
This will be discussed in much greater detail in the next section.

Because of the ferroelectricity, and inversion symmetry breaking, an internal electric field is generated through the dipoles associated with the polarization.
Combined with the high atomic SOC, a sizeable linear energy splitting between spin-polarized states can appear in these materials \cite{DiSante2013}, contrary to the tiny effect described above.
This $k$ dependent splitting manifests itself in the bandstructure as conical intersections surrounding time-reversal (TR) invariant points of the Brillouin Zone (BZ) (see Fig.~\ref{fig:Rashba_intro_dispersion}).
These points are special in the sense that the Kramers degeneracy \cite{Kramerstheorem} is enforced through the translational symmetry of the lattice, rather than by inversion symmetry.
If we take a spin up Bloch state at such a TR invariant point, by applying the time-reversal operator to the wavevector ($\bm k \rightarrow \bm {-k}$, $\bm{\sigma} \rightarrow -\bm{\sigma}$), it is brought to a point on the opposite side of the BZ, and can be translated back to the original point through a reciprocal lattice translation.
Since the system obeys both TR symmetry and translational symmetry, this down spin state has to have the same energy as the original up spin state, enforcing the Kramers degeneracy.
So, even though the inversion symmetry brakes the Kramers degeneracy at general $k$-points, leading to the spin splitting, it is still present at these TR invariant points, leading to the intersection of cones in Fig.~\ref{fig:Rashba_intro_dispersion}.

Due to the definite spin-polarization of these bands, current carriers \footnote{holes in the case of Fig.~\ref{fig:Rashba_intro_dispersion}} traveling through the material will tend to align their spins to this spin-polarization.
Since the sign of the electric field creating the spin-splitting is a result of the orientation of the ferroelectric polarization, it can be reversed through the application of a sufficiently strong external electric field.
The spin-polarized states have been observed both experimentally \cite{Ishizaka2011,Liebmann2016,Krempasky2015} and in {\it ab-initio} density functional theory (DFT) simulations \cite{DiSante2013}.
It is, however, often not well understood what the underlying microscopic mechanisms are that lead to the observed large splitting.

We start by investigating multiple contributions to the $\bm{k}$ dependent spin-splitting, and comment on their respective magnitudes and symmetry requirements. 
All lead to an energy term of the form:
\begin{equation}
	\label{eq:Rashba_form}
	H_R(\bm{k}) = \alpha_R \frac{\bm{E}}{|\bm{E}|} \cdot (\bm{k} \times \bm{\sigma}),
\end{equation}
with $\bm{\sigma}$ denoting the electron spin operator, and $\bm{E}$ the electric field.
Some of the microscopic effects that will be discussed are well known, while others are more obscure.
We will see that the more obscure ones actually dominate in most cases, and that the largest contributions stem from a combination of electrostatic and relativistic effects.

As a hallmark example of materials that demonstrate a large spin-splitting, we focus in this Chapter on Germanium Telluride (GeTe).
Its dispersion is exactly the one we used to demonstrate the conical shape of the bandstructure, displayed Fig.~\ref{fig:Rashba_intro_dispersion}.
To investigate how the microscopic effects manifest themselves in GeTe, we use DFT followed by a Wannierization in order to gain access to the local, real-space, properties of the Bloch functions.
\begin{figure}[h]
	\begin{subfigure}[b]{0.49\textwidth}
	\caption{}
	\IncludeGraphics{intro_dispersion_svg.png}
	\end{subfigure}
	\begin{subfigure}[b]{0.49\textwidth}
	\caption{}
	\IncludeGraphics{cone.png}
	\end{subfigure}
	\caption{\label{fig:Rashba_intro_dispersion}
		{\bf Large Rashba splitting} a) The band dispersion of the first valence band in GeTe around the $Z$-point of the Brillouin zone (see Fig.~\ref{fig:Rashba_bands_dos} for details). The blue and red graphs shows the non-relativistic (NSOC) and relativistic (SOC) bandstructure, respectively. The $u-d$ and $u'-d'$ labels designate the up and down spin-polarized bands, where the prime signifies that the orientation of the spin quantization axis depends on $\bm{k}$. b) Intersecting conical energy surface due to the spin-splitting. The spin texture is indicated by the arrows. This chiral spin-texture is what leads to the changing up-down spin axis.}
\end{figure}

We finish with a conclusion, summarizing the observations and make the claim that the linear spin-splitting attributed to the relativistic Rashba effect is almost exclusively a result of different effects. 

\section{Rashba-Bychkov Effect \label{sec:Rashba_relativistic}}
%############################ NEW ############################
Before turning to the less well known effects, we summarize the main points of the original Rashba-Bychkov effect, first derived in their seminal 1959 paper \cite{Rashba1959SymmetryAr}.
It is a purely relativistic effect that can be derived from a second order expansion of the Dirac-equation in small parameter $1/c$:
\begin{equation}
	\label{eq:Rashba_dirac}
	H \Ket{\psi} = \left[\frac{\bm{p}^2}{2m} - e V - \frac{e \hbar}{4m^2c^2}(\bm{\sigma}\cdot[\bm{\nabla}V \times \bm{p}]) + \frac{\bm{p}^2}{8m^2c^2} V - \frac{\bm{p}^4}{8m^3c^2}\right]\Ket{\psi} = E\Ket{\psi}
\end{equation}
where $\Ket{\psi}$ is a two component spinor, $V$ denotes the electric potential, $\bm{\sigma}$ a vector of Pauli-matrices $(\sigma_x, \sigma_y, \sigma_z)$, $m$ and $e$ the electron mass and charge, respectively.
$\bm \nabla = (\frac{\partial}{\partial x}, \frac{\partial}{\partial y}, \frac{\partial }{\partial z})$ denotes spatial derivatives, and $\bm{p} = -i\hbar \bm{\nabla}$ the canonical momentum.
The first two terms are the non-relativistic part of the Hamiltonian, the third one represents the SOC, the fourth is known as the Darwin effect and the fifth is the relativistic correction to the effective electron mass. 
As is common in literature, we introduce the spin-orbit coupling constant $\lambda = \frac{e \hbar}{4m^2c^2}$.
The potential $V(\bm r)$ is periodic in a crystal, leading to eigenstates given by the Bloch wavefunctions $\BlochKetr{n} = \eikr{r} \Ket{\unkr{n}{k}}$, where $\Ket{\unkr{n}{k}}$ denotes the cell-periodic part, $\eikr{r}$ the envelope function, and $n$ the band index.
In the following we drop $\bm{r}$ in the wavefunctions, and inner products correspond to an integration over all space:

\begin{equation}
	\Sandwich{\phi_n^{\bm{k}}}{\psi_m^{\bm{k}'}} = \int d\bm{r} e^{-i (\bm k - \bm k') \cdot \bm r} u_n^{\bm k }(\bm r)^* u_m^{\bm k'}(\bm r).
\end{equation}

To obtain the eigenvalue equation for $\Ket{\unk{n}{k}}$, we insert $\BlochKet{n}$ in Eq.~\ref{eq:Rashba_dirac}, and carry out the differentiation
\begin{equation}
\bm{p} \, \Eikr\Ket{\unk{n}{k}} = -i \hbar \bm \nabla \Eikr\Ket{\unk{n}{k}} = \Eikr(\bm p + \hbar \bm k)\Ket{\unk{n}{k}},
\end{equation}
and similarly substitute $\bm{p}^2 \rightarrow (\bm{p}+\bm{k})^2$.
This leads to the following equation for $\Ket{\unk{n}{k}}$:
\begin{align}
	\label{eq:Rashba_unk_Vs}
	E_n \Ket{\unk{n}{k}} =& \left(V_0 + V_1  +  V_2 + V_3 \right) \Ket{\unk{n}{k}} \\
	V_0^{\bm{k}} =& \frac{p^2}{2m} - eV + \frac{\hbar^2 k^2}{2m} \\
	V_1^{\bm{k}} =& \hbar\frac{\bm{k}\cdot\bm{p}}{m} \\
	V_2^{\bm{k}} =& -\lambda \bm{\sigma} \cdot ( \bm{\nabla}V \times \bm{k}) \\
	V_3^{\bm{k}} =& -\lambda \bm{\sigma} \cdot ( \bm{\nabla}V \times \bm{p}).
\end{align}
We neglected the last two terms of Eq.~\ref{eq:Rashba_dirac} since they are exceedingly small and don't contribute to the linear form of Eq.~\ref{eq:Rashba_form}.

Before turning to a $\bm{k} \cdot \bm{p}$ perturbative expansion of the above equations, we would like to address the question of which electric fields $\bm{\nabla}V$ will contribute to $V_2$ and $V_3$, i.e. that the former is the well-known first-order Rashba-Bychkov term, whereas the latter represents the atomic SOC.
Both terms originate from the partial application of the third contribution in Eq.~\ref{eq:Rashba_dirac}.
To do this, we separate $\bm{\nabla}V$ into two contributions, one coming from the potential wells created by the atoms, and another originating from the internal field created by the ferroelectric polarization (it is assumed that no external fields are applied):
\begin{equation}
	\bm{\nabla} V = \bm{E} = \bm{E}_{at} + \bm{E}_{P}
\end{equation}
These contributions, together with the two parts of the Bloch functions (i.e. the cell periodic $\unkr{n}{k}$, and envelope function $\eikr{r}$) are pictorially shown in Fig.~\ref{fig:Efield_cell_drawing}. 
\begin{figure}[h]
~\centering
\IncludeGraphics[width=\linewidth]{Rashba_potentials.png}\caption{\label{fig:Efield_cell_drawing}{\bf Potentials and wavefunctions inside the unit cell.} Pictorial representation of the different electric potentials ($V$) and fields ($E$) in a 2D square lattice. The blue spheres show the atoms, which produce the spherically symmetric red potential $V_a$ and field $E_a$. The purple slab denotes a uniform ferroelectric polarzation potential $V_p$ which in this case is directed along the crystalline $y$-direction, creating field $E_p$. The green and yellow surfaces denote two components of the Bloch function: $u_n(k) e^{ikx}$, where the yellow is the envelope plane wave part and the green the cell periodic part. $k$ and $p_u$ denote the two components to the momentum of the bloch function, the former from the envelope function, and the latter the contribution from the periodic part.}
\end{figure}

Looking at the picture, it becomes clear that the contribution of the atomic potential applied to the envelope function is zero because while $\bm{k}$ is a, $\bm{E}_{at}$ is odd throughout the unit cell, leading to the contributions on either side of the potential wells to cancel out.
Thus, the only contribution to $V_2$ comes from the uniform (even) $\bm{E}_P$, which is in general very small compared to the atomic one $\bm{E}_{at}$.
A similar argument can be applied to the contribution to $V_3$: due to the shape of the periodic part of the wavefunction, necessarily having zero total $\bm{p}$ \footnote{The system would not be in an equilibrium state if the periodic part had nonzero momentum.}, only the contribution from $\bm{E}_{at}$ will be nonzero.
This is essentially the well-known atomic SOC and can be rather large due to the large fields close to the ions. 

We now turn to a perturbative expansion around a high-symmetry, time reversal (TR) invariant point $\bm{k}_0$ inside the first BZ.
It is assumed that Eq.~\ref{eq:Rashba_unk_Vs} can be solved for such a point, and where the two spin states are necessarily degenerate due to the TR symmetry.
Due to the broken inversion symmetry and inclusion of the SOC terms ($V_2$,$V_3$), this degeneracy will be lifted for $\bm{k}$-points which are not TR invariant.
We denote the periodic parts of the two degenerate Bloch states by $\Ket{u_n^\downarrow}$ and $\Ket{u_n^\uparrow}$.
It is important to realize that the orientation of spin axis of the eigenstates of Eq.~\ref{eq:Rashba_unk_Vs} depends on the direction of both $\bm{k}$ and $\bm{P}$, as will become clear later.
This means that for each $\bm{k}$ point, the actual orientation of the up and down spins varies.
Without loss of generality we take $\bm{k}_0 = \bm{0}$ and $E^{\uparrow,\downarrow}_n(\bm{0}) = 0$.
Expanding in the deviation away from the high-symmetry point, in the usual $\bm{k}\cdot\bm{p}$ sense, and keeping only linear $\bm{k}$ terms, and terms up to second order in $1/c$, we get
\begin{align}
	\label{eq:kp_expansion}
	E_n^{\sigma_1}(\bm{k}) =& - V_2^{\sigma_1}(\bm{k}) + \\
		& \sum_{m,\sigma_2 \neq n,\sigma_1}\frac{\Bra{u_n^{\sigma_1}} V_1(\bm{k}) \Ket{u_m^{\sigma_2}}\Bra{u_m^{\sigma_2}} V_3(\bm{k}) \Ket{u_n^{\sigma_1}} + h.c.}{E_n^{\sigma_1} - E_m^{\sigma_2}}.
\end{align}

Referring back to the earlier discussion on the origin of various terms in Eq.~\ref{eq:Rashba_unk_Vs}, and in particular what fields contribute to $V_2$ or $V_3$, we find that, if $u_n$ and $u_m$ have contributions that originate from an atom with a strong SOC, the first and second term contribute with the same order of magnitude.
This is true even though the latter is of higher order in the perturbation theory.

From the point of view of symmetries, another requirement for the second term to be nonzero is that $u_n$ and $u_m$ have contributions with different parity, since $\bm{p}$ is odd in spatial coordinates, $\bm{k} \cdot \bm{p}$ is only non-zero when one of the constituent orbitals is odd with respect to a spatial direction and the other even.
One example could be a $p_y$ orbital and a $s$-$p_z$ hybdridized one, which would be created by the ferroelectricity with electric polarization along the $z$-axis.
This hybridization will be discussed in further detail in the context of a toy Tight-Binding model below.

We can thus arrive to the conclusion that, while these contributions are allowed and therefore present in inversion broken systems, they will contribute very little to the linear spin splitting due to the tiny prefactors (i.e. $\lambda \approx 10^{-6}$eV) involved.
It was thought for a long time that at interfaces, where the uniform $\bm{\nabla}V$ in $V_2$ can be relatively big, the spin-splitting could be explained from this purely relativistic argument, but even there the contributions we discussed up to now are generally much smaller than what is observed in reality. 

We thus need to seek a different explanation for the large splittings that are observed, especially for those in bulk materials, which brings us to the rest of this Chapter in which it will be shown that the combination of electrostatics with strong atomic SOC ($V_3$ in Eq.~\ref{eq:Rashba_unk_Vs}) can lead to the observed behavior.
This will lead to a similar second order perturbative effect as in the second term of Eq.~\ref{eq:kp_expansion}, but where the relativistic $V_1$ is substituted by a contribution from electrostatics, which is not plagued by the small relativistic prefactor.

\section{Orbital Rashba Effect}

As will be demonstrated throughout this section, it turns out that the key lies in the realization that Bloch functions at non-TR invariant $k$-points, generate nonzero intercell electric dipoles if they have a nonzero orbital angular momentum (OAM)\cite{Petersen2000,Park2011,Go2016}.
These dipoles couple to any electric field, e.g. the internal one assiated with ferroelectric polarization, with the inverse effect leading to the generation of Bloch functions with nonzero OAM.
This OAM then couples to the spin of the electrons throught the atomic SOC, leading to a splitting with the size determined (in part) by the weight of the ion around which the OAM develops.
We will thus focus in this section on how this OAM appears, since it is well-known that in crystals it is generally said to be quenched.
Indeed, ignoring atomic SOC, the influence of neighboring ions on the charge distribution of the atomic orbitals that form the Bloch functions is such that it favors orbitals with zero OAM.
This is directly translated into the commonly used language of $p_x$, $p_y$ and $p_z$ orbitals in the case of $L=1$.

Nonetheless, there exist two mechanisms that favor nonzero OAM.
Firstly, atomic SOC gains energy through $\mathcal{H}_{SOC} = -\lambda_a \hat{\bm{L}} \cdot \hat{\bm{\sigma}}$ when the OAM is nonzero. Even at TR-invariant $k$-points this leads to orbitals with some OAM.
When expanding for small $\bm{k}$, a linear varying term will appear in the energy dispersion that originates from the plane wave $\eikr{r}$ envelope of the Bloch functions.
Secondly, even without inclusion of atomic SOC, it will be shown that the OAM of the Bloch functions appears in a chiral texture as one moves away from the high-symmetry $k$-point, similarly to how the relativistic Rashba effect leads to a chiral spin texture, due to the earlier discussed intercell dipoles.
If one then includes again the atomic SOC, this linear-in-k $\bm{L}$ results in a linear variation of the energy with either positive or negative slope depending on the spin orientation in relation to $\hat{\bm{L}}$.

We now proceed by giving a pedagogical derivation of these mechanisms based on a tight-binding model \cite{Petersen2000,Go2016}.

\subsection{Tight-Binding model}
The tight-binding model is defined on a 2D square layer with lattice parameter $a$, one atom per unit cell, and four Wannier orbitals centered on that atom.
These orbitals resemble the angular character of an $s$-orbital and three $p$-orbitals: $\Ket{s^{\bm{n}}}$, $\Ket{x^{\bm{n}}}$, $\Ket{y^{\bm{n}}}$, $\Ket{z^{\bm{n}}}$, where $\bm{n}$ denotes the unit-cell indices ($n_x$,$n_y$) to which the Wannier function belongs to.
To simplify notation, we omit $\bm{0}$ in writing the WFs of the central unit cell.
We furthermore assume that these orbitals have a Gaussian radial shape $\Sandwich{\bm{r}}{s_{\bm{n}}} = e^{-\frac{|\bm{r}-\bm{n}a|^2}{a_0}}$, $\Sandwich{\bm{r}}{\alpha_{\bm{n}}} = \alpha e^{-\frac{|\bm{r}-\bm{n}a|^2}{a_0}}$ with $\alpha = x, y, z$, and inner products implying integrals over space.
The reason for choosing Gaussians is to make solving the overlap integrals easier, changing instead to a different radial shape for the orbitals does not lead to any qualitative changes to the derivation below.
The bare tight-binding Hamiltonian is denoted as $\hat{H}_0$ and includes the usual hopping parameters due to overlap $t_{\alpha\beta}^{\bm{n}\bm{n}'} = \Bra{\alpha^{\bm{n}}}\frac{\hat{\bm{p}}^2}{2m} + \hat{V}\Ket{\beta^{\bm{n}'}}$.
To mimick the inversion symmetry breaking in ferroelectric materials (i.e. with a polar space group), an electric field perpendicular to the layer ($z$ direction) is applied.
This allows extra hopping terms associated with $\hat{H}_{isb} = e (\hat{\bm{d}}\cdot \bm{E})$, with $\hat{\bm{d}}$ the electric dipole moment:
\begin{align}
	\label{eq:dipole}
	\Bra{s} \hat{H}_{isb} \Ket{z} &= 2 e E_z \theta_z^0\\
	\Bra{z} \hat{H}_{isb} \Ket{x^{\bm{n}}} &= e E_z \theta_z^n n_x\\
	\Bra{z} \hat{H}_{isb} \Ket{y^{\bm{n}}} &= e E_z \theta_z^n n_y
	% \hat{d}_z(\bm{n}) =& -ae^{-\frac{1}{2}\left(\frac{a|\bm{n}|}{a_0}\right)^2}\frac{\pi^{\frac{3}{2}}}{16\sqrt{2}}\left(\begin{matrix}0&0&0&2\\\\0&0&0&n_x\\\\0&0&0&n_y\\\\2&n_x&n_y&0\end{matrix}\right)\\
	% =&\theta_z^n\hat{d}^1_z(\bm{n}),
\end{align}
with $\theta_z^n = -ae^{-\frac{1}{2}\left(\frac{a|\bm{n}|}{a_0}\right)^2}\frac{\pi^{\frac{3}{2}}}{16\sqrt{2}}$, other terms of $\hat{H}_{isb}$ are zero.
Fig.~\ref{fig:Rashba_overlapdip} shows pictorially how these terms arise from the electric dipoles between the shifted orbitals. 
\begin{figure}[t]
~\centering
\IncludeGraphics[width=\linewidth]{overlapdip.png}\caption{\label{fig:Rashba_overlapdip} {\bf Overlap dipoles.} (a) On-site dipole from $\Ket{s}$ and $\Ket{z}$ hybridization, (b) Dipole due to overlap of shifted $\Ket{p}$ orbitals, the dashed line signifies a unit cell boundary. This figure was taken in part from \cite{Ponet2018}.}
\end{figure}
Since we are interested in ferroelectric materials, the internal field $E_z$ caused by the electric polarization is usually small.
This warrants a perturbative approach where $\hat{H}_{isb}$ is the perturbation on $\hat{H}_0$, leading to a hybridization between the $\Ket{s}$ and $\Ket{z}$ orbitals:
\begin{align}
	\Ket{\tilde{z}} &= \Ket{z} + \frac{\Bra{s} 2 e E_z \theta^n_z\Ket{z}}{\varepsilon_z - \varepsilon_s}\Ket{s}\\
	\Ket{\tilde{s}}   &= \Ket{s} + \frac{\Bra{z} 2 e E_z \theta^n_z\Ket{s}}{\varepsilon_s - \varepsilon_z}\Ket{z},
\end{align}
where $\varepsilon_s = \Bra{s} \hat{H}_0 \Ket{s}$ and $\varepsilon_z = \Bra{z} \hat{H}_0 \Ket{z}$.

We can then write the kinetic energy part of $\hat{H}_0$ and terms of this hybrid $\Ket{\tilde{z}}$ orbital in the central unit cell and the shifted $\Ket{x}$, $\Ket{y}$ orbitals, leading to:
\begin{align}
	\Bra{\tilde{z}} \hat{H}_0 \Ket{x^{\bm{n}}} &= \frac{2 e E_z \theta_z^n}{\varepsilon_z - \varepsilon_s}\Bra{s} \frac{-\bm{\nabla}^2}{2} \Ket{x^{\bm{n}}}\\
	&= \frac{4 e E_z (\theta_z^n)^2}{\varepsilon_z - \varepsilon_s} n_x (-5 + a^2 |\bm{n}|^2)\\
	\Bra{\tilde{z}} \hat{H}_0 \Ket{y^{\bm{n}}} &= \frac{4 e E_z (\theta_z^n)^2}{\varepsilon_z - \varepsilon_s} n_y (-5 + a^2 |\bm{n}|^2)
\end{align}

To construct $\hat{H}_0^{\bm{k}}$ and $\hat{H}_{isb}^{\bm{k}}$, one can fourier transform the Wannier Functions following Eq.~\ref{eq:Theory_wantok}:
\begin{equation}
	\label{eq:Rashba_wantok}
	\Ket{\alpha^{\bm{k}}} = \frac{1}{\sqrt{N}}\sum_{\bm{n}} e^{i \bm{k}\cdot \bm{n}}\Ket{\alpha^{\bm{n}}}
\end{equation}
with $\Ket{\alpha}$ one of the four aforementioned orbitals, $\bm{k}$ written in terms of crystalline coordinates ($\frac{2\pi}{a}$), and $N$ denoting the total amount of unit cells in the material. This results in:
\begin{align}
	\label{eq:Rashba_wantokH}
	\hat{H}_0^{\bm{k}} + \hat{H}_{isb}^{\bm{k}} &= \sum_{\bm{n}} \eikr{n} \left(\hat{H}_0^{\bm{n}} + \hat{H}_{isb}^{\bm{n}}\right)%\\
	% &= \sum_{\bm{n}} i \sin(\bm{k}\cdot \bm{n})(\hat{H}_0(\bm{n}) + \hat{H}_{isb}(\bm{n})) + k\mathrm{-even\,terms},
\end{align}
% keeping only the $\sin(\bm{k}\cdot\bm{n})$ part of the exponent since the $\cos(\bm{k}\cdot\bm{n})$ part does not result in linear-in-$k$ terms, in the small $\bm{k}$ expansion below we keep only these linear-in-$k$ terms.

If we then assume that at $k=0$ the Bloch functions are formed from $p_x$, $p_y$, $\tilde{p}_z$ and $\tilde{s}$ orbitals through Eq.~\ref{eq:Rashba_wantok}, a perturbation theory for small deviations $\bm{k}$ away from zero can be formulated as ($\bm{k}=0$ superscripts are omitted for simplicity):
\begin{equation}
	\Ket{\alpha^{\bm{k}}}=\Ket{\alpha} + \sum_{\beta \neq \alpha} \frac{\Bra{\beta} \hat{H}_0^{\bm{k}} + \hat{H}_{isb}^{\bm{k}} \Ket{\alpha}}{\varepsilon_{\alpha} - \varepsilon_{\beta}} \Ket{\beta}
\end{equation}

Gathering the linear-in-$k$ terms from the expansion of $\eikr{n}$ in Eq.~\ref{eq:Rashba_wantokH}, and assuming $\varepsilon_p = \Bra{x} \hat{H}_0 \Ket{x} = \Bra{y} \hat{H}_0 \Ket{y}$ at $\bm{k}=0$, we find
\begin{align}
	\Ket{x^{\bm{k}}} &= \Ket{x} + i \Theta \frac{k_x}{\varepsilon_p - \varepsilon_{\tilde{z}}}\Ket{\tilde{z}} \\
	\Ket{y^{\bm{k}}} &= \Ket{y} + i \Theta \frac{k_y}{\varepsilon_p - \varepsilon_{\tilde{z}}}\Ket{\tilde{z}}\\
	\Ket{\tilde{z}^{\bm{k}}} &= \Ket{\tilde{z}} + i \Theta \frac{1}{\varepsilon_{\tilde{z}} - \varepsilon_p} \left(k_x\Ket{x} + k_y\Ket{y}\right),
\end{align}
with $\Theta = \frac{\pi^{5/2}}{256 a^3 }\left(-16\sqrt{2} + \frac{3a\pi^{3/2}}{\varepsilon_z - \varepsilon_s}\right) e E_z$.
Then, using the definition of the OAM operators for $p$-orbitals:
\begin{equation}
	\hat{L}_x =\left(\begin{matrix}0&0&0\\\\0&0&-i\\\\0&i&0&\end{matrix}\right), \hat{L}_y =\left(\begin{matrix}0&0&i\\\\0&0&0\\\\-i&0&0&\end{matrix}\right), \hat{L}_z =\left(\begin{matrix}0&-i&0\\\\i&0&0\\\\0&0&0&\end{matrix}\right)
\end{equation}
we find that,
\begin{align}
	\label{eq:oam_from_k}
	\Bra{\tilde{z}^{\bm{k}}} \hat{L}_x \Ket{\tilde{z}^{\bm{k}}} &= - 2 \Theta \frac{k_y}{\varepsilon_{\tilde{z}} - \varepsilon_p}\\ 
	\Bra{\tilde{z}^{\bm{k}}} \hat{L}_y \Ket{\tilde{z}^{\bm{k}}} &= 2 \Theta \frac{k_x}{\varepsilon_{\tilde{z}}-\varepsilon_p} 
\end{align}
These expressions for $\hat{\bm{L}}$ can be filled into the expression for the atomic SOC $\hat{H}_{soc}= \lambda \hat{\bm{L}} \cdot \hat{\bm{\sigma}}$ to find the energy for $\Ket{\tilde{z}^{\bm{k}}}$:
\begin{equation}
	\label{eq:Rashba_from_OAM}
	\varepsilon^{\bm{k}} = \frac{2 \lambda \Theta}{\varepsilon_{\tilde{z}}-\varepsilon_p}(\bm{k} \times \bm{\sigma})
\end{equation}
which has the form of Eq.~\ref{eq:Rashba_form}.
We want to emphasize here that the only influence of the initial choice of the Gaussian radial shape of the orbitals is reflected in the prefactor in front of $e E_z$ in the definition of $\Theta$.

From this qualitative derivation, it is clear that the main reason behind the orbital Rashba effect can be traced back to the observation that Bloch functions with nonzero OAM have electric dipoles that couple to the inversion symmetry breaking electric field.
We identified two sources that lead to the effect.

The first one comes from the direct overlap dipoles between $p_x$ and $p_y$ orbitals with the $p_z$ orbital (panel (b) in Fig.~\ref{fig:Rashba_overlapdip})\cite{Petersen2000}.
An exaggerated demonstration for the case of $p_x + ip_z$ orbitals is shown in Fig.~\ref{fig:Rashba_interference}.
\begin{figure}[t!]
\IncludeGraphics[width=\linewidth]{interference.png}
\caption{\label{fig:Rashba_interference}{\bf Interference between orbitals with nonzero OAM.} Three neighboring unit cells are displayed, each with the same $p_x + ip_z$ orbital (thus having nonzero $l_y$). The wave functions of the left and right unit cells have their phase rotated by the plane-wave part $e^{i k_x R_x}$. The amplitude and phase of the wave function are encoded with the length and polar angle of the arrows. It is clear from the resulting phases (red arrows) that there exists constructive interference on the bottom half of the material, and destructive interference at the top half, which leads to the intercell dipoles. This figure was taken from Ref.~\cite{Ponet2018}.}
\end{figure}

The second is due to the hybridization of the $s$ and $p_z$ orbitals (panel (a) in Fig.~\ref{fig:Rashba_overlapdip}), and the kinetic energy term between the $s$ and neighboring $p_x$ and $p_y$ orbitals \cite{Go2016}.
These two terms are reflected in the two terms that contribute to $\Theta$, and lead to a chiral, linear-in-$k$ texture of the OAM.
Through the atomic SOC, this linear-in-$k$ OAM will couple to the spin and result in the final splitting with the Rashba-like form of Eq.~\ref{eq:Rashba_from_OAM}.

There is one final term that contributes to the linear variation of OAM with $\bm{k}$, which is present due to the unquenching of the OAM at the high symmetry point (e.g. $|k|=0$) due to atomic SOC~\cite{Park2011}.
If the Bloch functions at $\bm{k}=0$ are written as a linear combination of WFs that have angular character of $p$-orbitals
\begin{equation}
\Ket{\psi} = \sum_{\bm{n},\alpha} c_\alpha \Ket{\alpha^{\bm{n}}},
\end{equation}
the general formula for the OAM in terms of these $c_\alpha$ becomes
\begin{equation}
	\hat{L}_{\gamma}= i \epsilon_{\alpha \beta \gamma} c^*_\alpha c_\beta
\end{equation}
when $\alpha,\beta,\gamma$ designate $x,y,z$.
Thus, through the gain on $\hat{H}_{soc}$ if $\bm{j} = \bm{l} + \bm{\sigma}$, the orbitals at the high-symmetry point will now be a linear combination of what were orginally pure $p$-orbitals (due to the quenching we assumed that the bands at $\bm{k}=0$ had a single $c_\alpha \neq 0$ in the previous derivation).
This unquenching will not be complete in the sense that it won't create orbitals with maximal $\bm{j}$ as is the case for an isolated ion through the Hund's rules, since in a crystal the neighboring charge will limit the creation of OAM\footnote{This is the reason why without SOC the OAM is fully quenched.}.

Similar to the above derivation, a small-$k$ expansion for Bloch functions $\Ket{\psi^{\bm{k}}}$ with nonzero OAM can be performed around $|k|=0$ leading in general to:
\begin{align}
	\Ket{\psi^{\bm{k}}} =& \sum_{\bm{n},\alpha} c_\alpha^{\bm{k}} \eikr{n} \Ket{\alpha^{\bm{n}}}\\
	=& \sum_{\bm{n},\alpha} \left(c_\alpha + \bm{k}\left.\frac{\partial c_\alpha^{\bm{k}}}{\partial_{\bm{k}}}\right\rvert_{\bm{k}=0} \right)\left(1 + i\bm{k}\cdot\bm{n}\right)\Ket{\alpha^{\bm{n}}}.
\end{align}
Again we omit terms with $\bm{k}=0$ for clarity.
Focusing on the terms that vary linearly with $k$, the second term in the $c_{\alpha}$ expansion together with the first term in the exponent expansion is exactly the contribution that was discussed before.
The other term for the Orbital Rashba effect, due to unquenching, originates from combining the first term in the $c_\alpha$ expansion with the second in the exponent expansion, leading to the contribution
\begin{align}
	\label{eq:d_z_from_OAM}
	\varepsilon^{\bm{k}} = \Bra{\psi^{\bm{k}}} \hat{H}_{isb} \Ket{\psi^{\bm{k}}} =&  i\sum_{\bm{n},\alpha,\beta} c_\alpha^* c_\beta \bm{k}\cdot\bm{n}\Bra{\alpha} \hat{H}_{isb} \Ket{\beta^{\bm{n}}}\\
	=&\frac{-\pi^{5/2}}{8\sqrt{2}a^3}i(c^*_xc_zk_x + c^*_yc_z k_y) e E_z  \\
	=&\frac{\pi^{5/2}}{8\sqrt{2}a^3}(L_y k_x - L_x k_y) e E_z.
\end{align}
These expressions show us that, due to ${\hat{H}_{soc} = \lambda \hat{\bm{L}}\cdot\hat{\bm{\sigma}}}$ and $\hat{H}_{isb} = e E_z d_z$, an additional Rashba-like term appears in the energy dispersion, due to the creation of orbitals with nonzero $L_y$ and $L_x$ at the high-symmetry point.

With the understanding that one can find Rashba like dispersions, coming not from the usually considered purely relativistic, but also from electrostatic mechanisms, we now look at a concrete example that behaves very similar to the above toy model, GeTe.

\section{Germanium Telluride}
\begin{figure}[h]
\IncludeGraphics[width=\linewidth]{crystal.png}
\caption{\label{fig:Rashba_crystal}{\bf Crystal structure of GeTe.}~a) Rhombohedral unit cell, with the polarization along the [111] direction in yellow. b) First Brillouin zone with the high-symmetry $k$-path that is under focus indicated by the blue lines.}
\end{figure}
At high temperatures GeTe has the standard rocksalt structure.
When temperatures are lowered below $T_c \approx 720$~K \cite{DiSante2013}, a displacive phonon instability occurs, freezing in a $\bm{q} = 0$ phonon that shifts the central Te ion along the [111] direction \cite{Rabe1987}.
The space group after this inversion symmetry breaking phase transition becomes R$3m$ (\#160 in International Tables), and a nonzero electric polarization along the $z$-direction develops as indicated by the yellow arrow in Fig.~\ref{fig:Rashba_crystal}(a). The first BZ for this low temperature structure is shown in panel (b), with the high-symmetry path under focus indicated by the blue line.


\begin{figure}[b!]
	\begin{subfigure}[b]{0.49\textwidth}
	\caption{Non-relativistic}
	\IncludeGraphics{NSOC_dos.png}
	\end{subfigure}
	\begin{subfigure}[b]{0.49\textwidth}
	\caption{Fully-relativistic}
	\IncludeGraphics{SOC_dos.png}
	\end{subfigure}
\caption{\label{fig:Rashba_bands_dos}{\bf Bandstructure of GeTe.} In both panels the bandstructure was colored according to the contribution of the constituent orbitals, as indicated in the flanking density of state plots. The bands that have larger $s$ character are situated 5 eV below the shown window.}
\end{figure}
In the calculations reported below, we used a unit cell with lattice parameter $a=$4.28~\AA, leading to a volume of 53.4~\AA$^3$.
This makes the unit cell slightly larger than the experimental one, which has a volume of 53.3~\AA$^3$~\cite{Serebryanaya1995}.
The Te ion is shifted by 0.25~\AA~ away from the center of the unit cell.

The bands around the Fermi level are formed mostly by $s$- and $p$-orbitals from the Te and Ge ions.
Due to the inversion symmetry breaking, a linear spin-splitting is observed around the $Z$-point of the BZ in the plane perpendicular to the [111] direction.
The $Z-\Gamma$ path does not show any splitting because the electric field is along the $z$-direction, and thus only a variation of $k_x$ and $k_y$ will show a linear splitting, as previously discussed.

This means that from the point of view of our earlier derivation using the tight-binding model in terms of $s$- and $p$-orbitals, it is the perfect test case.
Moreover, since it is a bulk material, we can neglect the contributions from to the relativistic Rashba effects described in Sec.~\ref{sec:Rashba_relativistic}, due to the small potential gradients $\bm{\nabla}V$ resulting from the bulk ferroelectricity.

\section{Methods}
In the calculations presented here, we used the implementation in the {\texttt Quantum ESPRESSO} software package \cite{Giannozzi2009}.
In order to confirm the linear varying OAM even when spin-orbit is not included we performed non-relativistic, as well as fully relativistic DFT calculations.
Both were performed using the Optimized Norm-conserving Vanderbilt Pseudo Potentials \cite{Hamann2013}, and a generalized-gradient approximation with Perdew-Burke-Enzerhof parametrization for the exchange-correlation functional \cite{Perdew1996}.
Plane waves with an energy cutoff of 30 Ry were included, with a 120 Ry cutoff for the density.

The reciprocal space was sampled using a 6x6x6 Monkhorst-Pack grid \cite{Pack1977} for the self-consistent calculations, using a total energy convergence threshold of $10^{-7}$Ry.

Afterwards we used the {\texttt Wannier90} package \cite{Mostofi2014AnFunctions} to perform the Wannierization as described in the theory Chapter \ref{sec:Wannier}, using projections on hydrogenic $s$- and $p$-orbitals for both Ge and Te ions as the initial guess.
A 10x10x10 $k$-grid was used for the sampling of the BZ during the Wannierization. 

% \lp{look in the high throughput paper for a decent definition of the wannier band distance w.r.t the dft one!}

\section{Results and Discussion}

The bands we will focus on most are the three topmost valence bands which consist mostly of Te $p$-orbitals, as can be seen from Fig.~\ref{fig:Rashba_bands_dos}(b).
These demonstrate the largest spin-splitting suggesting that indeed the atomic SOC plays an important rule seen as it is larger on Te compared to Ge.
If one were to fit the dispersion of the topmost band to Eq.~\ref{eq:Rashba_form}, a large prefactor $\alpha_R\approx 30.7$~eV$\cdot$\angstrom \cite{DiSante2013} is found.
As discussed before, a purely relativistic effect would rather have a prefactor of $\alpha_R \approx 10^{-6}$~eV.

Another observation that is hard to explain by purely relativistic means, is the orientation of the spin-polarization inside the different split bands.
It was found that depending on the band, the spin is oriented such that $\alpha_R$ in Eq.~\ref{eq:Rashba_form} is either negative or positive. 
This has been confirmed experimentally \cite{Krempasky2015} and depends on the value of the total angular momentum $j$.
This can only be explained through the OAM and atomic SOC mechanism, the manifestation of which in our results will be discussed further down.

By Wannierizing the bands around the Fermi level, we can construct a tight-binding model in terms of Te and Ge $s$- and $p$-orbitals.
The WFs for the spin-up Te orbitals thus obtained are displayed in Fig.~\ref{fig:Rashba_wannierfunctions}(a), the spin-down orbitals have similar charge distributions.
The total spread of the 16 spinor WFs is 53.4~\AA$^2$~.
\begin{figure}[h]
\IncludeGraphics[width=\linewidth]{Wannierfunctions.png}
\caption{\label{fig:Rashba_wannierfunctions}{\bf Wannier functions} a) The spin-up WFs of the orbitals around the Tellurium ion. The isosurface is constructed from the norm of the wavefunctions, with the coloration denoting the sign of the real part. b) The charge density in the central unit cell of the Bloch function of the top valence band, at $k=$ (0, 0.1, 0.9) in units of \AA$^{-1}$. The orbital angular momentum, polarization, and $k$-vector are shown by the red, blue, and green arrows, respectively. In each panel, the unit cell is shown by the wireframe, and the Te and Ge ions are yellow and grey, respectively.}
\end{figure}

The interpolation of the ab-initio bandstructure by the tight-binding model in terms of the WFs is shown in Fig.~\ref{fig:Rashba_wannierization}, which demonstrates that it agrees very well with the bands in the inner window, and the valence bands in particular. 
\begin{figure}[h]
\IncludeGraphics[width=\linewidth]{wanvsdft.png}
\caption{\label{fig:Rashba_wannierization}{\bf Wannier interpolation.} Bandstructure interpolation by the tight-binding Hamiltonian in terms of the WFs for non-relativistic (a) and relativistic (b) calculations.}
\end{figure}

By diagonalizing the tight-binding model, writing the BFs in terms of WFs
\begin{equation}
	\BlochKet{n} = \sum_{\bm{R}} \eikr{R} \sum_{\alpha} c_{\alpha}^n(\bm{k}) \WanKet{\alpha}{R},
\end{equation}
we can use the contribution inside the central unit cell ($\bm{R} = \bm{0}$) to get an idea of the real space distribution of the BFs.
An example is shown in Fig.~\ref{fig:Rashba_wannierfunctions}(b) for a BF along the $Z-A$ path in the BZ.
Moreover, using this real space distribution, we can calculate the center of mass inside the unit cell and the OAM through $\hat{\bm{L}} = -\frac{i}{\hbar} (\bm{r} \times \bm{k})$, where in our calculations $\hbar = 1$.
This way of calculating the OAM is the so-called atomic centered approximation, in the discussion below we focus on the OAM around the Te ion.
From the modern theory of OAM~\cite{Thonhauser2011}~it was shown that there may be another itinerant contribution to the orbital angular momentum, which is not of importance in our description. 

\begin{figure*}[h]
\centering
\IncludeGraphics[width=0.49\linewidth]{OAMvsK}
\IncludeGraphics[width=0.49\linewidth]{OAMvsK2.png}
\caption{\label{fig:Rashba_oamvseigvalv}{\bf Bloch function properties.} Comparison between the real-space observables and energy dispersion in (a) the first and (b) third valence band. The values are plotted in function of the relative distance from the $Z$ point $\bm{k}_r = \bm{k} - \bm{k}_Z$, towards the A and U points. The green graphs denote the values before turning on atomic SOC, whereas the orange and blue graphs denote the two spin-split bands.}
\end{figure*}

The dispersion, OAM, and SAM of the first valence band are shown in the left panel of Fig.~\ref{fig:Rashba_oamvseigvalv}. 
Confirming our earlier derivation, we can see that non-zero, linearly varying OAM is formed along a $k$-path away from the high-symmetry $Z$-point.
Moreover, the OAM is perpendicular to both the $z$-axis and the $k$ vector, as it should be from Eq.~\ref{eq:oam_from_k}, and can also be seen from panel (b) in Fig. \ref{fig:Rashba_wannierfunctions}.
This leads e.g. to $l_y=0$ along the $A \to Z$ path, where only $k_y$ is nonzero.
When atomic SOC is included (the orange and blue graphs), we see the spin-splitting that results from having the spin oriented either along or opposite to the linearly varying OAM.
The unquenching of the OAM at the $Z$-point when SOC is included is also clearly visible.
This leads to a change in the slope of the linear varying OAM, as compared with the non-relativistic calculations (green graphs), that originates from the corresponding contribution to the dipole moment Eq.~\ref{eq:d_z_from_OAM}.
This correlation between OAM, SAM and dipole moment can also be observed in the top left panels of Fig.~\ref{fig:Rashba_oamvseigvalv} of showing the center of mass $\bar{z}=\int_{\textrm{supercell}}d \bm{r} z |\psi^{\bm{k}}(r)|^2$ of the BFs, which is proportional to the dipole moment around the same reference point.

Due to the existence of two atoms inside the primitive unit cell, another manifestation of this mechanism arises through the overlap of $p_x$ and $p_y$ on Te and $s$ and $p_z$ on Ge.
This leads to a dipole as can be seen from the variation of the charge density of the BFs as we move away from the $Z$-point in Fig.~\ref{fig:Rashba_diffdens}.
\begin{figure}[h]
\IncludeGraphics[width=.7\columnwidth]{diffdens}
\caption{\label{fig:Rashba_diffdens}{\bf Bloch function dipole.} The variation of the charge density of the Bloch function of the first valence band $\left.\frac{\partial |\psi(k)|^2}{\partial k}\right\rvert_{k=Z}$ away from $Z$ towards $A$. Te and Ge ions are in red and blue, respectively. The charge asymmetry around Ge showcases the nonzero dipole moment along $z$, which couples to the local electric field near Ge ion. Figure taken from Ref.~\cite{Ponet2018}}
\end{figure}

When we compare this first valence band with the properties of the third valence band, shown in the right panel of Fig.~\ref{fig:Rashba_oamvseigvalv}, we can clearly see the previously discussed issues with the purely relativistic explanation.
As stated before, we can note that not only the magnitude but also the sign of the prefactor in Eq.~\ref{eq:Rashba_form} is opposite for these two bands, showcased by the size of the splitting, and by the ordering of the spin-up vs the spin-down split bands.
This is because the character of the first and third valence bands are different, where the former is comprised mostly of Te $j_{\frac{3}{2}}$ orbitals, whereas the third valence band is predominantly $j_{\frac{1}{2}}$.
This causes the relative orientation of the OAM and SAM to be parallel in the first band, and anti-parallel for the third, as shown in Fig.~\ref{fig:Rashba_textures}.
This then leads to the different ordering of the spin-split bands and opposite sign of $\alpha_R$.

There is one last very interesting feature one can notice from Fig.~\ref{fig:Rashba_textures} (c) and (f), that is, the switching of the character (and SAM, OAM orientation) of the bands, very close to the $Z$ point.
This is because the crystal field breaks rotationial symmetry causing the atomic $j$ to not be a conserved quantity, i.e. there is a mixing between different atomic $j$ orbitals, which varies strongly in this very narrow region around $Z$.

All these considerations lead to a very nontrivial SAM and OAM texture of the bands as we progress through the BZ.


\begin{figure*}[t!]
  \subfloat[Lower 3rd valence band]{
    \centering
    \IncludeGraphics[width=0.49\linewidth]{Ltexture5.png}
  }
  ~
  \subfloat[Upper 3rd valence band]{
    \centering
    \IncludeGraphics[width=0.49\linewidth]{Ltexture6.png}
  }
  \\
  ~
  \subfloat[Lower 1st valence band]{
    \centering
    \IncludeGraphics[width=0.49\linewidth]{Ltexture9.png}
  }
  ~
  \subfloat[Upper 1st valence band]{
    \centering
    \IncludeGraphics[width=0.49\linewidth]{Ltexture10.png}
  }\\
  ~
  \subfloat[Zoom-in of (b)]{
    \centering
    \IncludeGraphics[width=0.49\linewidth]{Ltexture6small.png}
  }
  ~
  \subfloat[Zoom-in of (d)]{
    \centering
    \IncludeGraphics[width=0.49\linewidth]{Ltexture9small.png}
  }
  \caption{\label{fig:Rashba_textures}{\bf OAM and SAM in the BZ.} The textures of BFs around the $Z$ point are shown for the first and third valence bands of GeTe. The black and green arrows show the OAM and SAM textures, respectively. The length of the arrows was chosen separately for clarity in each figure and should thus not be compared. The color maps signify the energy of the bands, relative to the Fermi level. The small box around the $Z$ point indicates the area, magnified in panels (c) and (f). In the zoomed figures (c) and (f) one can observe the change or relative orientation between the SAM and OAM when moving away from the $Z$ point, signifying a change of character between $j=1/2$ and $j=3/2$. Figure taken from Ref.~\cite{Ponet2018}}
\end{figure*}

\section{Conclusions}

We have explored the microscopic origin of the giant Rashba-like spin splitting in the band structure of bulk ferroelectric GeTe with high atomic SOC. We derived the form of the band dispersion in the Wannier representation, that relates the large spin splitting to the intricate interplay between OAM, atomic SOC, the crystal field and the electric polarization. It turns out that the crucial component, which is not present in the relativistic Rashba effect, is the emergence of a nonzero electric dipole of the Bloch functions due to their OAM. The quantitative analysis based on WFs and atomic-centered approximation confirms this mechanism in GeTe. We find a very good agreement between the proposed band dispersion, Eq.~\ref{eq:Rashba_hami}, and the dispersions of the first and third valence bands, where the effect manifests itself most clearly.

Ultimately, the results suggest that (1) large ferroelectric polarization, (2) high atomic SOC, and (3) highly isotropic environment producing little OAM quenching could be the design rules for new materials with strong Rashba-like spin splitting. These materials could enable spintronic devices with the much needed electric control of spin polarization.
