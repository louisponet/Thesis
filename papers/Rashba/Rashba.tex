\newcommand{\Unkr}{u_n(\bm{k}, \bm{r})}
% \renewcommand{\eikr}{$e^{i\bm{k}\cdot\bm{r}}$}
\newcommand{\Eikr}{e^{i\bm{k}\cdot\bm{r}}}
\chapter{Spin-momentum locking in high spin-orbit coupled ferroelectrics \label{ch:Rashba}}
\section{Introduction}
% Bulk ferroelectrics with large atomic spin-orbit coupling allow for electric control of spin-polarized states~\cite{DiSante2013,Ishizaka2011,Kim2014, Liebmann2016, Krempasky2015SurfaceSemiconductor}, allowing for the switching of the spin texture by an externally applied electric field. 
% The underlying mechanism is, however, not well understood.
% Note: Not sure we should really focus on technology, also it's not particularly clear it wants to use both spin and charge, rather than just spin.
% The research area of spintronics aims to supersede standard electronics by using both the spin and charge degree of freedom of the carriers for information processing and storage. %\sa{https://doi.org/10.1146/annurev-conmatphys-070909-104123}
%
The research field of spintronics aims to understand the behavior of spins inside materials, and translate this understanding into active control of these degrees of freedom for possible technological applications \cite{Joshi2016}.
Many devices have been theorized, for example, spin field-effect transistors (spin-FET)\cite{Datta1990}, and storage devices which utilize spin-current and associated spin-transfer torque to efficiently manipulate magnetic domains \cite{Kent2015}, both in ferromagnets \cite{Nunez2011}, and antiferromagnets \cite{Jungwirth2016}.
In spite of fundamental interest and potential for applications, the actual realization of these spin-FET devices has been rather elusive.
One of the main culprits for the limited success to date is that the devices require very granular, ideally electric, control of the spin.
This is often precluded by the widely separated energy scales and weak coupling between magnetic and charge degrees of freedom.
One class of materials that can allow for such electric control of spin-polarized states are the ferroelectric semiconductors with large atomic spin-orbit coupling (SOC) \cite{Picozzi2014,DiSante2013,Ishizaka2011,Kim2014}.
Inversion symmetry breaking together with SOC results in a linear energy splitting of spin-polarized bands, manifested in the bulk band structure as a conical intersection surrounding a time-reversal symmetric point of the Brillouin Zone (BZ) (see Fig.~\ref{fig:Rashba_intro_dispersion}).
Due to the definite spin-polarization of these bands, current carriers \footnote{holes in the case of Fig.~\ref{fig:Rashba_intro_dispersion}} traveling through the material will tend to align their spins to this spin-polarization.
As we discuss in more detail in the following, the direction of the spin polarization depends on the orientation and strength of the electric field $\bm{E}$.
In ferroelectrics, an internal field results from polarization $P$ and associated electric dipole moments, allowing it to be tuned and switched by an external electric field.
These spin-polarized states have been observed both experimentally \cite{Ishizaka2011,Liebmann2016,Krempasky2015SurfaceSemiconductor}, and from {\it ab-initio} density functional theory (DFT) simulations \cite{DiSante2013}.
It is, however, often not well understood and misattributed what the underlying microscopic mechanisms are that lead to the observed splitting.

We start by investigating multiple origins of this $\bm{k}$ and $\bm{E}$ dependent spin-splitting. The discussed energy contributions all arise due to the Hamiltonian of form
\begin{equation}
	\label{eq:Rashba_form}
	H_R(\bm{k}) = \alpha_R \frac{\bm{E}}{|\bm{E}|} \cdot (\bm{k} \times \hat{\bm{\sigma}}),
\end{equation}
where $\hat{\bm{\sigma}}$ is the electron spin operator, and $\bm{E}$ the electric field.
It will turn out that multiple microscopic effects can lead to contributions of this form, some of these are well known, others are more obscure.
The magnitude of the contribution depends on these microscopics, namely whether the effect is purely relativistic or rather a combination of relativistic and electrostatic terms.

Certain materials demonstrate an exceedingly big splitting, for example in Germanium Telluride (GeTe) as can be seen in Fig.~\ref{fig:Rashba_intro_dispersion}.
We therefore use this material as the main illustration throughout this chapter.
To investigate how these effects occur in GeTe, we utilize DFT followed by a Wannierization in order to get an idea of the local, real-space, properties of the Bloch functions around the time reversal symmetric points of the BZ where the linear splitting occurs.

\begin{figure}[h]
	\begin{subfigure}[b]{0.49\textwidth}
	\caption{}
	\IncludeGraphics{intro_dispersion_svg.png}
	\end{subfigure}
	\begin{subfigure}[b]{0.49\textwidth}
	\caption{}
	\IncludeGraphics{cone.png}
	\end{subfigure}
	\caption{\label{fig:Rashba_intro_dispersion}
		{\bf Large Rashba splitting} (a) The band dispersion of the first valence band in GeTe around the $Z$-point of the Brillouin zone (see Fig.~\ref{fig:Rashba_BZBSDOS} for details). Two graphs show both non spin-orbit coupled (NSOC, blue), and spin-orbit coupled (SOC, red) case. The $u-d$ and $u'-d'$ labels designate the up and down spin-polarized bands, where the prime signifies that the orientation of the spin quantization axis depends on $\bm{k}$. b) Shifted parabolic energy leves due to Rashba splitting, and associated spin texture indicated by the arrows.}
\end{figure}

We finish with a conclusion on the origin of most observed Rashba-like phenomena \lp{could be better}.

\section{Rashba-Bychkov Effect}
%############################ NEW ############################
The first discovered, and most well known effect bearing the form of Eq.~\ref{eq:Rashba_form} is the Rashba-Bychkov effect, first derived in their seminal 1959 paper\cite{Rashba1959SymmetryAr}.
It is a purely relativistic effect that is derived from an expansion to second order in $1/c$, $c$ denoting the speed of light, of the electronic Dirac equation (where the Fouldy-Wouthuysen transformation was applied):

\begin{equation}
	\label{eq:Rashba_dirac}
	\mathcal{H} \Ket{\psi} = \left[\frac{\bm{p}^2}{2m} - e V - \frac{e \hbar}{4m^2c^2}(\bm{\sigma}\cdot[\bm{\nabla}V \times \bm{p}]) - \frac{\hbar^2}{8m^2c^2} \Delta V - \frac{\bm{p}^4}{8m^3c^2}\right]\Ket{\psi} = E\Ket{\psi}
\end{equation}
where $\psi$ is a two component spinor, $V$ denotes the electric potential, $\bm{\sigma}$ a vector of Pauli-matrices $(\sigma_x, \sigma_y, \sigma_z)$, $m$ and $e$ the electron mass and charge respectively, and $\bm{p}$ the canonical momentum.
The first two terms are the non-relativistic part of the Hamiltonian, the third one represents the SOC, the fourth is known as the Darwin effect and the fifth is the relativistic correction to the effective electron mass. 
As is common in literature, we introduce the spin-orbit coupling constant $\lambda = \frac{e \hbar}{4m^2c^2}$.
In a crystal with a perodic potential $V(\bm{r})$, the electronic wavefunctions are Bloch wavefunctions $\BlochKet{n} = \eikr{r} \Ket{\unkr{n}{k}}$, where $u_n$ denotes the cell-periodic part, and $n$ is the band index.
In the following we drop $\bm{r}$ in the wavefunctions, and inner products are assumed to be integrated over all space.
To obtain the eigenvalue equation for $\Ket{\unk{n}{k}}$, we insert $\BlochKet{n}$ in Eq.~\ref{eq:Rashba_dirac}, and carry out the differentiation $\bm{p} \, \Eikr\Ket{\unk{n}{k}} = \Eikr(\bm{p} + \bm{k})\Ket{\unk{n}{k}}$, and similarly substitute $\bm{p}^2 \rightarrow (\bm{p}+\bm{k})^2$.
This leads to the following equation for $\Ket{\unk{n}{k}}$:
\begin{align}
	\label{eq:Rashba_unk_Vs}
	E_n \Ket{\unk{n}{k}} =& \left(V_0 + V_1  +  V_2 + V_3 \right) \Ket{\unk{n}{k}} \\
	V_0^{\bm{k}} =& \frac{p^2}{2m} - eV + \frac{\hbar^2 k^2}{2m} \\
	V_1^{\bm{k}} =& \hbar\frac{\bm{k}\cdot\bm{p}}{m} \\
	V_2^{\bm{k}} =& -\lambda \bm{\sigma} \cdot ( \bm{\nabla}V \times \bm{k}) \\
	V_3^{\bm{k}} =& -\lambda \bm{\sigma} \cdot ( \bm{\nabla}V \times \bm{p}).
\end{align}
We neglected the last two terms of Eq.~\ref{eq:Rashba_dirac} since they are exceedingly small and don't contribute to the linear form of Eq.~\ref{eq:Rashba_form}.

Before turning to a $\bm{k} \cdot \bm{p}$ perturbative expansion of the above equations, we would like to address the question of which electric fields $\bm{\nabla}V$ will contribute to $V_2$ and $V_3$, i.e. that the former is the well-known first-order Rashba-Bychkov term, whereas the latter represents the atomic SOC.
Both terms originate from the partial application of the third contribution in Eq.~\ref{eq:Rashba_dirac}.
To do this, we separate $\bm{\nabla}V$ into two contributions, one coming from the potential wells created by the atoms, and another originating from the internal field created by the ferroelectric polarization (it is assumed that no external fields are applied):
\begin{equation}
	\bm{\nabla} V = \bm{E} = \bm{E}_{at} + \bm{E}_{P}
\end{equation}
These contributions, together with the two parts of the Bloch functions (i.e. the cell periodic $\unkr{n}{k}$, and envelope function $\eikr{r}$) are pictorially shown in Fig.~\ref{fig:Efield_cell_drawing}. 
\begin{figure}[h]
~\centering
\IncludeGraphics[width=\linewidth]{Rashba_potentials.png}\caption{\label{fig:Efield_cell_drawing} Pictorial representation of the different electric potentials ($V$) and fields ($E$) in a 2D square lattice. The blue spheres show the atoms, which produce the spherically symmetric red potential $V_a$ and field $E_a$. The purple slab denotes a uniform ferroelectric polarzation potential $V_p$ which in this case is directed along the crystalline $y$-direction, creating field $E_p$. The green and yellow surfaces denote two components of the Bloch function: $u_n(k) e^{ikx}$, where the yellow is the envelope plane wave part and the green the cell periodic part. $k$ and $p_u$ denote the two components to the momentum of the bloch function, the former from the envelope function, and the latter the contribution from the periodic part.}
\end{figure}

Looking at the picture, it becomes clear that the contribution of the atomic potential applied to the envelope function is zero because while $\bm{k}$ is a, $\bm{E}_{at}$ is odd throughout the unit cell, leading to the contributions on either side of the potential wells to cancel out.
Thus, the only contribution to $V_2$ comes from the uniform (even) $\bm{E}_P$, which is in general very small compared to the atomic one $\bm{E}_{at}$.
A similar argument can be applied to the contribution to $V_3$: due to the shape of the periodic part of the wavefunction, necessarily having zero total $\bm{p}$ \footnote{The system would not be in an equilibrium state if the periodic part had nonzero momentum.}, only the contribution from $\bm{E}_{at}$ will be nonzero.
This is essentially the well-known atomic SOC and can be rather large due to the large fields close to the ions. 

We now turn to a perturbative expansion around a high-symmetry, time reversal (TR) invariant point $\bm{k}_0$ inside the first BZ.
It is assumed that Eq.~\ref{eq:Rashba_unk_Vs} can be solved for such a point, and where the two spin states are necessarily degenerate due to the TR symmetry.
Due to the broken inversion symmetry and inclusion of the SOC terms ($V_2$,$V_3$), this degeneracy will be lifted for $\bm{k}$-points which are not TR invariant.
We denote the periodic parts of the two degenerate Bloch states by $\Ket{u_n^\downarrow}$ and $\Ket{u_n^\uparrow}$.
It is important to realize that the orientation of spin axis of the eigenstates of Eq.~\ref{eq:Rashba_unk_Vs} depends on the direction of both $\bm{k}$ and $\bm{P}$, as will become clear later.
This means that for each $\bm{k}$ point, the actual orientation of the up and down spins varies.
Without loss of generality we take $\bm{k}_0 = \bm{0}$ and $E^{\uparrow,\downarrow}_n(\bm{0}) = 0$.
Expanding in the deviation away from the high-symmetry point, in the usual $\bm{k}\cdot\bm{p}$ sense, and keeping only linear $\bm{k}$ terms, and terms up to second order in $1/c$, we get
\begin{align}
	\label{eq:kp_expansion}
	E_n^{\sigma_1}(\bm{k}) =& - V_2^{\sigma_1}(\bm{k}) + \\
		& \sum_{m,\sigma_2 \neq n,\sigma_1}\frac{\Bra{u_n^{\sigma_1}} V_1(\bm{k}) \Ket{u_m^{\sigma_2}}\Bra{u_m^{\sigma_2}} V_3(\bm{k}) \Ket{u_n^{\sigma_1}} + h.c.}{E_n^{\sigma_1} - E_m^{\sigma_2}}.
\end{align}

Referring back to the earlier discussion on the origin of various terms in Eq.~\ref{eq:Rashba_unk_Vs}, and in particular what fields contribute to $V_2$ or $V_3$, we find that, if $u_n$ and $u_m$ have contributions that originate from an atom with a strong SOC, the first and second term contribute with the same order of magnitude.
This is true even though the latter is of higher order in the perturbation theory.

From the point of view of symmetries, another requirement for the second term to be nonzero is that $u_n$ and $u_m$ have contributions with different parity, since $\bm{p}$ is odd in spatial coordinates, $\bm{k} \cdot \bm{p}$ is only non-zero when one of the constituent orbitals is odd with respect to a spatial direction and the other even.
One example could be a $p_y$ orbital and a $s$-$p_z$ hybdridized one, which would be created by the ferroelectricity with electric polarization along the $z$-axis.
This hybridization will be discussed in further detail in the context of a toy Tight-Binding model below.

We can thus arrive to the conclusion that, while these contributions are allowed and therefore present in inversion broken systems, they will contribute very little to the linear spin splitting due to the tiny prefactors (i.e. $\lambda \approx 10^{-6}$eV) involved.
It was thought for a long time that at interfaces, where the uniform $\bm{\nabla}V$ in $V_2$ can be relatively big, the spin-splitting could be explained from this purely relativistic argument, but even there the contributions we discussed up to now are generally much smaller than what is observed in reality. 

We thus need to seek a different explanation for the large splittings that are observed, especially for those in bulk materials, which brings us to the rest of this Chapter in which it will be shown that the combination of electrostatics with strong atomic SOC ($V_3$ in Eq.~\ref{Rashba_unk_Vs}) can lead to the observed behavior.
This will lead to a similar second order perturbative effect as in the second term of Eq.~\ref{eq:kp_expansion}, but where the relativistic $V_1$ is substituted by a contribution from electrostatics, which is not plagued by the small relativistic prefactor.

\section{Orbital Rashba Effect}
Unlike the purely relativistic Rashba-Bychkov effect, the orbital effect combines strong atomic SOC with the generation of nonzero orbital angular momentum (OAM) through electrostatic means.
This contribution, therefore, is not limited by the small prefactors of Eq.~\ref{eq:kp_expansion}. As will be explicitely shown below, electric dipole matrix element s can cause Bloch functions to acquire nonzero OAM when any electric field is present (e.g. from electric polarization), and vice versa Bloch functions with nonzero OAM  acquire nonzero electric dipoles between unit cells\cite{Petersen2000,Park2011,Go2016}. This leads to two seperate effects.

Firstly, when atomic SOC is included, OAM is unquenched at the high symmetry k-points. This leads to a correction to the band dispersion that varies linearly with $\bm{k}$. Secondly, even without including the contribution of the atomic SOC, the OAM of the Bloch functions to appear in a chiral texture as one moves away from the high-symmetry k-point, similarly to how the relativistic Rashba effect leads to a chiral spin texture. If one then includes the atomic SOC from Eq.~\ref{eq:Rashba_unk_Vs}, this linear-in-k $l$ will lead to a linear variation of the energy with either positive or negative slope, depending on the spin orientation.

We now proceed by giving a pedagogical derivation of this mechanism based on a tight-binding model \cite{Petersen2000,Go2016}.

\subsection{Tight-Binding model}
The tight-binding model is defined on a 2D square layer with lattice parameter $a$, one atom per unit cell, and four Wannier orbitals centered on that atom.
These orbitals resemble the angular character of an $s$-orbital and three $p$-orbitals: $\Ket{s^{\bm{n}}}$, $\Ket{x^{\bm{n}}}$, $\Ket{y^{\bm{n}}}$, $\Ket{z^{\bm{n}}}$, where $\bm{n}$ denotes the unit-cell indices ($n_x$,$n_y$) to which the Wannier function belongs to.
To simplify notation, we omit $\bm{0}$ in writing the Wannier functions of the central unit cell.
We furthermore assume that these orbitals have a Gaussian radial shape $\Sandwich{\bm{r}}{s_{\bm{n}}} = e^{-\frac{|\bm{r}-\bm{n}a|^2}{a_0}}$, $\Sandwich{\bm{r}}{\alpha_{\bm{n}}} = \alpha e^{-\frac{|\bm{r}-\bm{n}a|^2}{a_0}}$ with $\alpha = x, y, z$, and inner products implying integrals over space.
The reason for choosing gaussians is to make solving the overlap integrals easier, changing to a different radial shape for the orbitals does not lead to any qualitative changes to the derivation below.
The bare tight-binding Hamiltonian is denoted as $\hat{H}_0$ and includes the usual hopping parameters due to overlap $t_{\alpha\beta}^{\bm{n}\bm{n}'} = \Bra{\alpha^{\bm{n}}}\frac{\hat{\bm{p}}^2}{2m} + \hat{V}\Ket{\beta^{\bm{n}'}}$.
To mimick the inversion symmetry breaking in ferroelectric materials (i.e. with a polar space group), an electric field perpendicular to the layer ($z$ direction) is applied.
This allows extra hopping terms associated with $\hat{H}_{isb} = e (\hat{\bm{d}}\cdot \bm{E})$, with $\hat{\bm{d}}$ the electric dipole moment:
\begin{align}
	\label{eq:dipole}
	\Bra{s} \hat{H}_{isb} \Ket{z} &= 2 e E_z \theta_z^0\\
	\Bra{z} \hat{H}_{isb} \Ket{x^{\bm{n}}} &= e E_z \theta_z^n n_x\\
	\Bra{z} \hat{H}_{isb} \Ket{y^{\bm{n}}} &= e E_z \theta_z^n n_y
	% \hat{d}_z(\bm{n}) =& -ae^{-\frac{1}{2}\left(\frac{a|\bm{n}|}{a_0}\right)^2}\frac{\pi^{\frac{3}{2}}}{16\sqrt{2}}\left(\begin{matrix}0&0&0&2\\\\0&0&0&n_x\\\\0&0&0&n_y\\\\2&n_x&n_y&0\end{matrix}\right)\\
	% =&\theta_z^n\hat{d}^1_z(\bm{n}),
\end{align}
with $\theta_z^n = -ae^{-\frac{1}{2}\left(\frac{a|\bm{n}|}{a_0}\right)^2}\frac{\pi^{\frac{3}{2}}}{16\sqrt{2}}$, other terms of $\hat{H}_{isb}$ are zero.
Fig.~\ref{fig:overlapdip} shows pictorially how these terms arise from the electric dipoles between the shifted orbitals. 
\begin{figure}[t]
~\centering
\IncludeGraphics[width=\linewidth]{overlapdip.png}\caption{\label{fig:overlapdip} {\bf Overlap dipoles.} (a) On-site dipole from $\Ket{s}$ and $\Ket{z}$ hybridization, (b) Dipole due to overlap of shifted $\Ket{p}$ orbitals, the dashed line signifies a unit cell boundary.}
\end{figure}
Since we are interested in ferroelectric materials, the internal field $E_z$ caused by the electric polarization is usually small.
This warrants a perturbative approach where $\hat{H}_{isb}$ is the perturbation on $\hat{H}_0$, leading to a hybridization between the $\Ket{s}$ and $\Ket{z}$ orbitals:
\begin{align}
	\Ket{\tilde{z}} &= \Ket{z} + \frac{\Bra{s} 2 e E_z \theta^n_z\Ket{z}}{\varepsilon_z - \varepsilon_s}\Ket{s}\\
	\Ket{\tilde{s}}   &= \Ket{s} + \frac{\Bra{z} 2 e E_z \theta^n_z\Ket{s}}{\varepsilon_s - \varepsilon_z}\Ket{z},
\end{align}
where $\varepsilon_s = \Bra{s} \hat{H}_0 \Ket{s}$ and $\varepsilon_z = \Bra{z} \hat{H}_0 \Ket{z}$.

We can then write the kinetic energy part of $\hat{H}_0$ and terms of this hybrid $\Ket{\tilde{z}}$ orbital in the central unit cell and the shifted $\Ket{x}$, $\Ket{y}$ orbitals, leading to:
\begin{align}
	\Bra{\tilde{z}} \hat{H}_0 \Ket{x^{\bm{n}}} &= \frac{2 e E_z \theta_z^n}{\varepsilon_z - \varepsilon_s}\Bra{s} \frac{-\bm{\nabla}^2}{2} \Ket{x^{\bm{n}}}\\
	&= \frac{4 e E_z (\theta_z^n)^2}{\varepsilon_z - \varepsilon_s} n_x (-5 + a^2 |\bm{n}|^2)\\
	\Bra{\tilde{z}} \hat{H}_0 \Ket{y^{\bm{n}}} &= \frac{4 e E_z (\theta_z^n)^2}{\varepsilon_z - \varepsilon_s} n_y (-5 + a^2 |\bm{n}|^2)
\end{align}

To construct $\hat{H}_0^{\bm{k}}$ and $\hat{H}_{isb}^{\bm{k}}$, one can fourier transform the Wannier Functions following Eq.~\ref{eq:Theory_wantok}:
\begin{equation}
	\label{eq:Rashba_wantok}
	\Ket{\alpha^{\bm{k}}} = \frac{1}{\sqrt{N}}\sum_{\bm{n}} e^{i \bm{k}\cdot \bm{n}}\Ket{\alpha^{\bm{n}}}
\end{equation}
with $\Ket{\alpha}$ one of the four aforementioned orbitals, $\bm{k}$ written in terms of crystalline coordinates ($\frac{2\pi}{a}$), and $N$ denoting the total amount of unit cells in the material. This results in:
\begin{align}
	\label{eq:Rashba_wantokH}
	\hat{H}_0^{\bm{k}} + \hat{H}_{isb}^{\bm{k}} &= \sum_{\bm{n}} \eikr{n} \left(\hat{H}_0^{\bm{n}} + \hat{H}_{isb}^{\bm{n}}\right)%\\
	% &= \sum_{\bm{n}} i \sin(\bm{k}\cdot \bm{n})(\hat{H}_0(\bm{n}) + \hat{H}_{isb}(\bm{n})) + k\mathrm{-even\,terms},
\end{align}
% keeping only the $\sin(\bm{k}\cdot\bm{n})$ part of the exponent since the $\cos(\bm{k}\cdot\bm{n})$ part does not result in linear-in-$k$ terms, in the small $\bm{k}$ expansion below we keep only these linear-in-$k$ terms.

If we then assume that at $k=0$ the Bloch functions are formed from $p_x$, $p_y$, $\tilde{p}_z$ and $\tilde{s}$ orbitals through Eq.~\ref{eq:Rashba_wantok}, a perturbation theory for small deviations $\bm{k}$ away from zero can be formulated as ($\bm{k}=0$ superscripts are omitted for simplicity):
\begin{equation}
	\Ket{\alpha^{\bm{k}}}=\Ket{\alpha} + \sum_{\beta \neq \alpha} \frac{\Bra{\beta} \hat{H}_0^{\bm{k}} + \hat{H}_{isb}^{\bm{k}} \Ket{\alpha}}{\varepsilon_{\alpha} - \varepsilon_{\beta}} \Ket{\beta}
\end{equation}

Gathering the linear-in-$k$ terms from the expansion of $\eikr{n}$ in Eq.~\ref{eq:Rashba_wantokH}, and assuming $\varepsilon_p = \Bra{x} \hat{H}_0 \Ket{x} = \Bra{y} \hat{H}_0 \Ket{y}$ at $\bm{k}=0$, we find
\begin{align}
	\Ket{x^{\bm{k}}} &= \Ket{x} + i \Theta \frac{k_x}{\varepsilon_p - \varepsilon_{\tilde{z}}}\Ket{\tilde{z}} \\
	\Ket{y^{\bm{k}}} &= \Ket{y} + i \Theta \frac{k_y}{\varepsilon_p - \varepsilon_{\tilde{z}}}\Ket{\tilde{z}}\\
	\Ket{\tilde{z}^{\bm{k}}} &= \Ket{\tilde{z}} + i \Theta \frac{1}{\varepsilon_{\tilde{z}} - \varepsilon_p} \left(k_x\Ket{x} + k_y\Ket{y}\right),
\end{align}
with $\Theta = \frac{\pi^{5/2}}{256 a^3 }\left(-16\sqrt{2} + \frac{3a\pi^{3/2}}{\varepsilon_z - \varepsilon_s}\right) e E_z$.
Then, using the definition of the OAM operators for $p$-orbitals:
\begin{equation}
	\hat{L}_x =\left(\begin{matrix}0&0&0\\\\0&0&-i\\\\0&i&0&\end{matrix}\right), \hat{L}_y =\left(\begin{matrix}0&0&i\\\\0&0&0\\\\-i&0&0&\end{matrix}\right), \hat{L}_z =\left(\begin{matrix}0&-i&0\\\\i&0&0\\\\0&0&0&\end{matrix}\right)
\end{equation}
we find that,
\begin{align}
	\label{eq:oam_from_k}
	\Bra{\tilde{z}^{\bm{k}}} \hat{L}_x \Ket{\tilde{z}^{\bm{k}}} &= - 2 \Theta \frac{k_y}{\varepsilon_{\tilde{z}} - \varepsilon_p}\\ 
	\Bra{\tilde{z}^{\bm{k}}} \hat{L}_y \Ket{\tilde{z}^{\bm{k}}} &= 2 \Theta \frac{k_x}{\varepsilon_{\tilde{z}}-\varepsilon_p} 
\end{align}
These expressions for $\hat{\bm{L}}$ can be filled into the expression for the atomic SOC $\hat{H}_{soc}= \lambda \hat{\bm{L}} \cdot \hat{\bm{\sigma}}$ to find the energy for $\Ket{\tilde{z}^{\bm{k}}}$:
\begin{equation}
	\label{eq:Rashba_from_OAM}
	\varepsilon^{\bm{k}} = \frac{2 \lambda \Theta}{\varepsilon_{\tilde{z}}-\varepsilon_p}(\bm{k} \times \bm{\sigma})
\end{equation}
which has the form of Eq.~\ref{eq:Rashba_form}.
We want to emphasize here that the only influence of the initial choice of the Gaussian radial shape of the orbitals is reflected in the prefactor in front of $e E_z$ in the definition of $\Theta$.

From this qualitative derivation, it is clear that the main reason behind the orbital Rashba effect can be traced back to the observation that Bloch functions with nonzero OAM have electric dipoles that couple to the inversion symmetry breaking electric field. We identified two sources that lead to the effect: the first one comes from the direct overlap dipoles between $p_x$ and $p_y$ orbitals, and the $p_z$ orbital\cite{Petersen2000}, the second is due to the hybridization of the $s$ and $p_z$ overlap, and the kinetic energy term between the $s$ and neighboring $p_x$ and $p_y$ orbitals\cite{Go2016}. These two terms are reflected in the two terms that contribute to $\Theta$, and lead to a chiral, linear-in-$k$ texture of the OAM.
If there then exists strong atomic SOC, this linear-in-$k$ OAM will couple to the spin and result in the final Rashba-like form of Eq.~\ref{eq:Rashba_from_OAM}.


There is one final term that contributes to the linear variation of OAM with $\bm{k}$, which is present due to the unquenching of the OAM at the high symmetry point (e.g. $|k|=0$) due to atomic SOC \ref{Park2011}.
If the Bloch functions at $\bm{k}=0$ in this case are written as a linear combination of Wannier functions that have angular character of $p$-orbitals
\begin{equation}
\Ket{\psi} = \sum_{\bm{n},\alpha} c_\alpha \Ket{\alpha^{\bm{n}}},
\end{equation}
the general formula for the OAM in terms of these $c_\alpha$ becomes
\begin{equation}
	\hat{L}_{\gamma}= i \epsilon_{\alpha \beta \gamma} c^*_\alpha c_\beta
\end{equation}
when $\alpha,\beta,\gamma$ designate $x,y,z$.
Thus, through the gain on $\hat{H}_{soc}$ if $\bm{j} = \bm{l} + \bm{\sigma}$, the orbitals at the high-symmetry point will now be a linear combination of what were orginally pure $p$-orbitals (due to the quenching we assumed that the bands at $\bm{k}=0$ had a single $c_\alpha \neq 0$ in the previous derivation).
This unquenching will not be complete in the sense that it won't create orbitals with maximal $\bm{j}$ as is the case for an isolated ion through the Hund's rules, since in a crystal the neighboring charge will limit the creation of OAM\footnote{This is the reason why without SOC the OAM is fully quenched.}.


Similar to the above derivation, a small-$k$ expansion for orbitals $\Ket{\psi^{\bm{k}}}$ with nonzero OAM can be performed around $|k|=0$ leading in general to:
\begin{align}
	\Ket{\psi^{\bm{k}}} =& \sum_{\bm{n},\alpha} c_\alpha^{\bm{k}} \eikr{n} \Ket{\alpha^{\bm{n}}}\\
	=& \sum_{\bm{n},\alpha} \left(c_\alpha + \bm{k}\left.\frac{\partial c_\alpha^{\bm{k}}}{\partial_{\bm{k}}}\right\rvert_{\bm{k}=0} \right)\left(1 + i\bm{k}\cdot\bm{n}\right)\Ket{\alpha^{\bm{n}}}.
\end{align}
Again we omit terms with $\bm{k}=0$ for clarity.
Focusing on the terms that vary linearly with $k$, the second term in the $c_{\alpha}$ expansion together with the first term in the exponent expansion is exactly the contribution that was discussed before.
The other term for the Orbital Rashba effect, due to unquenching, originates from combining the first term in the $c_\alpha$ expansion with the second in the exponent expansion, leading to the contribution
\begin{align}
	\label{eq:d_z_from_OAM}
	\varepsilon^{\bm{k}} = \Bra{\psi^{\bm{k}}} \hat{H}_{isb} \Ket{\psi^{\bm{k}}} =&  i\sum_{\bm{n},\alpha,\beta} c_\alpha^* c_\beta \bm{k}\cdot\bm{n}\Bra{\alpha} \hat{H}_{isb} \Ket{\beta^{\bm{n}}}\\
	=&\frac{-\pi^{5/2}}{8\sqrt{2}a^3}i(c^*_xc_zk_x + c^*_yc_z k_y) e E_z  \\
	=&\frac{\pi^{5/2}}{8\sqrt{2}a^3}(L_y k_x - L_x k_y) e E_z.
\end{align}
These expressions show us that, due to ${\hat{H}_{soc} = \lambda \hat{\bm{L}}\cdot\hat{\bm{\sigma}}}$ and $\hat{H}_{isb} = e E_z d_z$, an additional Rashba-like term appears in the energy dispersion, due to the creation of orbitals with nonzero $L_y$ and $L_x$ at the high-symmetry point.

With the understanding that one can find Rashba like dispersions, coming not from the usually considered purely relativistic, but also from electrostatic mechanisms, we now look at a concrete example that behaves very similar to the above toy model, GeTe.

\begin{figure}[h]
% ~\centering
\IncludeGraphics[width=\linewidth]{BZBSDOS.pdf}
\caption{\label{fig:Rashba_BZBSDOS}(a) Rhombohedral unit cell and Brillouin zone of GeTe, with the polarization direction in yellow. (b) Band structure obtained from a DFT calculation with and without SOC, along the blue path in panel (a). (c) partial DOS for Te and Ge p orbitals computed without SOC.}
\end{figure}
\section{Overview: Germanium Telluride}
The space group of GeTe is R$3m$ (\#160 in International Tables), with ferroelectric polarization along the threefold rotation $z$-axis caused by an off-centering of the central Te atom \cite{Rabe1987} as displayed in Fig.~\ref{fig:Rashba_BZBSDOS}-(a). The valence and conduction bands are formed mostly by $s$ and $p$ orbitals from Te and Ge, respectively.
This, together with the large spin-splitting in the vicinity of the $Z$ point \cite{DiSante2013} as observed from the bandstructure in Fig.~\ref{fig:Rashba_BZBSDOS}-(b), lends it as a perfect test-case for the above described mechanisms. Moreover, since it is a bulk material, the relativistic Rashba effects contribute negligibly due to the small potential gradients caused by bulk ferroelectricity. The band structure,  presents a distinct large linear spin splitting around the $Z$ point, along the $Z-A$ and $Z-U$ paths. The $Z-\Gamma$ path shows no splitting because the electric field is along the $z$ direction, and as shown by the above derivation, only $k_x$ and $k_y$ will show a linear splitting.
The density of states (DOS), displayed in Fig.~\ref{fig:Rashba_BZBSDOS}-(c), confirms the orbital character of the bands.
% The valence bands are comprised mostly of Te 5p orbitals, whereas the conduction bands are mostly formed by Ge 4p orbitals. This orbital character of valence and conduction bands, together with the stronger atomic SOC on Te atom, results in a more pronounced spin splitting in the valence bands. We will focus on the first three valence bands, the top one of which mainly has $p_z$ character whereas the two lower bands, which are degenerate at the $Z$-point, are formed by $p_x$ and $p_y$ orbitals. The crystal field splitting between bands of $p_z$ and $p_{x,y}$ characters results from the distortion of the ideal Te-Ge$_6$ octahedron.
% \begin{figure*}[ht!]
% \centering
% \begin{subfigure}[b]{0.49\textwidth}
% \centering
% \IncludeGraphics[width=\linewidth]{OAMvsK}
% \caption{}
% \end{subfigure}
% ~
% \begin{subfigure}[b]{0.49\textwidth}
% \centering
% \IncludeGraphics[width=\linewidth]{OAMvsK2.png}
% \caption{}
% \end{subfigure}
% %{oamvseigvalsv1.pdf}
% \caption{Comparison between the real-space observables and energy dispersion in (a) the first and (b) third valence band. The values are plotted in function of the relative distance from the $Z$ point $\bm{k}_r = \bm{k} - \bm{k}_Z$, towards the A and U points. The green graphs denote the values before turning on atomic SOC, whereas the orange and blue graphs denote the two spin-split bands.}\label{fig:oamvseigvalv}
% \end{figure*}
% Wannier representation allows us to elucidate the $k$-dependent electric dipole moment arising due to interference of the atomic orbitals on the neighboring sites \cite{Marzari2012MaximallyApplications}. In order to obtain the effective Hamiltonian, we write the Bloch functions as linear combinations of the Wannier functions that describe the bands of interest: $$\psi_{\bm{k}}(\bm{r})=\sum_{\alpha,\bm{R}} c_\alpha(\bm{k}) w_\alpha(\bm{r}-\bm{R})e^{i\bm{k}\bm{R}},$$ where the Wannier functions $w_\alpha(r)$ are chosen to be real and $\bm{R}$ denote the unit cell in which the orbital is centered.

% In the case of GeTe the electric polarization is along the $z$ direction, making it sufficient to focus on the corresponding dipole moment
% \begin{align}
% d_z(\bm{k}) &=e\sum_{\alpha,\beta,\bm{R}}A_{\alpha,\beta}(\bm{k})e^{i\bm{k}.\bm{R}} \, \mathcal{Z}_{\alpha,\beta}^{\bm{R}} \\
% A_{\alpha,\beta}(\bm{k}) &= c_{\alpha}^*(\bm{k}) c_\beta(\bm{k}) \\
% \mathcal{Z}_{\alpha,\beta}^{\bm{R}} &= \int d\bm{r}   w_{\alpha}(\bm{r})w_\beta(\bm{r}-\bm{R}) z. \label{eq:Znn}
% \end{align}
% Expanding each term around $k_Z$, and keeping contributions up to first order in $\bm{k}^r = (k_Z - \bm{k})$, one obtains
% \begin{align}\label{eq:dip}
% d_z(\bm{k}) =e\sum_{\alpha,\beta,\bm{R}} &\left(A_{\alpha,\beta}(Z) + \bm{k}^r \left.\frac{\partial A_{\alpha,\beta}(\bm{k}) }{\partial \bm{k}}\right\rvert_{\bm{k}=Z}\right) \times \nonumber \\ 
% &(1 + i\bm{k}^r \cdot \bm{R}) \mathcal{Z}_{\alpha,\beta}^{\bm{R}}.
% \end{align}
% The term that combines the first order variation of $A_{\alpha,\beta}(\bm{k})$ with the zeroth order term from the exponent leads to a nonzero contribution only if $w_\alpha, w_\beta = s, p_z$ in the same unit cell ($\bm{R}=0$).
% This is the charge asymmetry that comes from $s$-$p_z$ hybridization and was previously considered in the context of giant spin splitting at surfaces in Refs.~\citenum{Petersen2000SimpleStates,Go2016SurfacePhysics} and for bulk perovskites in Ref.~\citenum{Kim2014}.
% The new terms that we consider here combine the zeroth and first order contributions to $A_{\alpha,\beta}(\bm{k})$ with the first order $i\bm{k}^r \cdot \bm{R}$ term.
% Using indices $1,2,3$ to denote $p_x, p_y, p_z$ Wannier functions one obtains the following relation between the Bloch function coefficients and the OAM: $l_k(\bm{k}) = -i \epsilon_{ijk} A_{ij}$.%\sa{only for pure $p_x$ etc. functions}

% \begin{figure}[t]
% ~\centering
% \IncludeGraphics[width=\linewidth]{1_1.png}\caption{\label{fig:overlaparrows} Dipoles due to interference for $p$, $d$ and $f$ orbitals, with maximal angular momentum projection, $\psi\sim e^{i l\phi}$, are shown in panels (a) - c) respectively. Black and blue arrows denote the complex phases of wavefunctions in neighboring unit cells, where the red arrow shows the resulting complex amplitude of the total Bloch function due to interference of the orbitals. The overlapping orbitals where spaced apart for visual clarity.}
% \end{figure}
% Upon closer inspection of the overlap dipole, $\mathcal{Z}_{\alpha,\beta}^{\bm{R}}$, one concludes that for $d_z$ to be nonzero, at least one of the orbitals $\alpha,\beta$ needs to be the $p_z$ orbital and the shift-vector $\bm{R}$ has to have a component along the second orbital ($p_x$ or $p_y$). This is illustrated in Fig.~\ref{fig:overlapdip}(a). 
% Another requirement for this term to be nonzero is that the orbitals have odd parity, such as $p$ and $f$. This is highlighted in Fig.~\ref{fig:overlaparrows}. If they have even parity (as is the case for the $d$ orbitals displayed in Fig.~\ref{fig:overlapdip}(b) and ~\ref{fig:overlaparrows}(b)), the overlap dipoles from neighboring unit cells will cancel. 
% Filling this into Eq.~(\ref{eq:dip}) and summing over the nearest neighbor unit cells ($\bm{R} = \pm 1$), we arrive at
% \begin{align}\label{eq:dipfinal}
%     d_z(\bm{k}) &= 4e \left(l_x(Z) + \bm{k}^r \left.\frac{\partial l_x(\bm{k}) }{\partial \bm{k}}\right\rvert_{\bm{k}=Z}\right) k^r_yR_y \mathcal{Z}_{y,z}^{R_y}\nonumber \\
%     &  - 4e\left(l_y(Z) + \bm{k}^r \left.\frac{\partial l_y(\bm{k}) }{\partial \bm{k}}\right\rvert_{\bm{k}=Z}\right) k^r_xR_x \mathcal{Z}_{x,z}^{R_x}.
% \end{align}
% The coupling of the electric dipole moment with an external electric field, or in the case of GeTe, with the electric polarization, leads to a contribution to the Hamiltonian of the form:
% \begin{equation}\label{HOR}
%     \mathcal{H}_{OR} \propto \bm{l} \cdot ( \bm{k} \times \bm{E})
% \end{equation}
% The terms with $l_{x,y}(Z)$ lead to a linear $k$-dependence of the band energy if the atomic SOC partially unquenches the OAM.
% The second term leads to energy contribution, quadratic in $k$, but results in a linear variation of the OAM even in the absence of SOC. This linear variation of the OAM will then couple to $\bm{\sigma}$ [also referred to as spin angular momentum (SAM)] through the atomic SOC, leading to the linear spin splitting.

% The energy dispersion close to the $Z$ point can thus be modeled by
% \begin{equation}\label{eq:hami}
% \epsilon(\bm{k}) = \epsilon(Z) + \lambda_{so} \bm{l}\cdot \bm{\sigma} + c_1 \bm{l} \cdot ( \bm{k} \times \bm{E})+ c_2 l_z^2,
% \end{equation}
% where $\epsilon(Z)$ contains a constant band shift, while the term with $\lambda$ is the atomic spin-orbit coupling, and the latter terms represent the contribution due to the threefold symmetric crystal field for $l=1$.
% To explicitly show how the linear variation of the OAM arises from the terms in Eq.~(\ref{eq:dipfinal}), we performed a $k\cdot p$-expansion of the first valence band which leads to following expressions for the OAM around $k=Z$:
% \begin{align}\label{eq:OAM}
% l_x &= -\frac{\lambda_{so}\sigma_x - c_1 E_z k_y}{c_2} \\
% l_y &= -\frac{\lambda_{so}\sigma_y + c_1 E_z k_x}{c_2},\nonumber
% \end{align}
% where we only included terms with the $E_z$ component.

\section{Methods}
We performed ab-initio DFT calculations using the Quantum-Espresso software package\cite{Giannozzi2009}. In order to confirm the linear varying OAM even when spin-orbit is not included we performed non-relativistic, as well as fully relativistic calculations.
Both were performed using ONCVPSP pseudopotentials, with 30 Ry for the energy cutoff, and 120 Ry for the density cutoff.
The reciprocal space was sampled using a 6x6x6 Monkhorst-Pack grid, for the self-consistent and non self-consistent calculations, using an energy convergence threshold of $10^{-7}$Ry.
Afterwards we used the Wannier90 package \cite{Mostofi2014AnFunctions} to perform the Wannierization, using projections on atomic $s$- and $p$-orbitals for both Ge and Te ions. The result is showcased in Fig.~\ref{fig:GeTe_wannierization}.
\begin{figure}[b!]
\IncludeGraphics[width=\linewidth]{wanvsdft.png}
\caption{\label{fig:GeTe_wannierization} Result of Wannierization procedure for non-relativistic (a) and relativistic (b) calculations.}
\end{figure}

Since the present theory depends crucially on local, real space, properties of the Bloch functions, the next step is to generate the Wannier orbitals $\Ket{w_\alpha(\bm{r})}$.



This provides us with the tools to analyze the real space properties of Bloch functions such as the dipole moment and OAM.



\section{Band structure}
% % ~\centering
\begin{figure}[b!]
\IncludeGraphics[width=\linewidth]{interference.png}
\caption{\label{fig:interference}Interference between orbitals with nonzero OAM. Three neighboring unit cells are displayed, each with the same $p_x + ip_z$ orbital (thus having nonzero $l_y$). The wave functions of the left and right unit cells have their phase rotated by the plane-wave part $e^{i k_x R_x}$. The amplitude and phase of the wave function are encoded with the length and polar angle of the arrows.}
\end{figure}
\begin{figure}[b!]
	\begin{subfigure}[b]{0.49\textwidth}
	\caption{Non-relativistic}
	\IncludeGraphics{NSOC_dos.png}
	\end{subfigure}
	\begin{subfigure}[b]{0.49\textwidth}
	\caption{Fully-relativistic}
	\IncludeGraphics{SOC_dos.png}
	\end{subfigure}
\caption{\label{fig:GeTe_bands_dos} Bandstructure and projected density of states. In both panels the bandstructure was colored according to the contribution of the constituent orbitals, as indicated in the density of state plots. The bands that are formed by the $s$-orbitals are situated 5 eV below the shown window.}
\end{figure}
The bands we will focus on most are the three topmost valence bands. As seen in Fig. \ref{fig:Rashba_BZBSDOS}-(b), these bands have the largest spin-splitting. This suggests that indeed the atomic SOC plays an important rule, seen as these bands are comprised mostly of Te orbitals, and SOC is largest on Te. If one were to fit the dispersion of the topmost band to Eq.~\ref{eq:Rashba_form},
it results in a large prefactor $\alpha_R\approx 30.7$~eV$\cdot$\angstrom \cite{DiSante2013}. As discussed before, a more realistic prefactor would be $\alpha_R=10^{-6}$~eV in the purely relativistic case inside the vacuum.

% This is even more abnormal considering the fact that this is a bulk material, and that it is known that proper ferroelectrics have zero polarization in open boundary conditions because the depolarizing fields are not screened. Moreover, in the periodic boundary conditions utilized by DFT, the potential mimics short circuit boundary conditions \cite{Meyer2008}, $\bm E=0$, which results in zero contribution from Eq.~\ref{eq:Rashba_form}. 

The last issue with the purely relativistic explanation, which has been confirmed experimentally\cite{Krempasky2015SurfaceSemiconductor}, lies in the orientation of the spin polarization of the split bands. According to Eq.~\ref{eq:Rashba_form}, the bands should all be split equally and Bloch functions at the same $k$-point should have the same spin orientations, since their character does not enter Eq.~\ref{eq:Rashba_form}. However, as has been shown and will be confirmed by our results below, the orientation depends on the character of the band, more specifically on the value of the total angular momentum $j$. 

% We performed ab initio calculations to study how the discussed effects manifest themselves in GeTe. To arrive at the desired basis of Wannier functions and tight-binding Hamiltonian, we first performed a collinear DFT calculation using the Quantum-Espresso package \cite{Giannozzi2009}, followed by WANNIER90 \cite{Mostofi2014AnFunctions}. For the DFT calculation we used a 10x10x10 Monkhorst-Pack k-grid, as well as an energy convergence threshold of $10^{-8}$~Ry. In the Wannierization step the gauge freedom of Wannier functions was exploited to arrive at a set of basis functions which are localized \cite{Marzari2012MaximallyApplications}, and resemble closely the atomic orbitals (spherical harmonics). Afterwards we added atomic SOC in the form $\lambda_{so} \bm{l} \cdot \bm{\sigma}$ using the muffin-tin approximation, with $\lambda_{so}$ for both atoms used as fitting parameters. Using this basis and Hamiltonian, we can the calculate the observables of interest, namely the OAM around Te and the dipole of the Bloch functions. 
\section{Results and Discussion}
The dispersion, OAM, and SAM of the first valence band are shown in the left panel of Fig.~\ref{fig:Rashba_oamvseigvalv}. Confirming our earlier conclusions, we can see that non-zero, linearly varying OAM is formed as we move away from the high-symmetry $Z$-point. Moreover, the OAM is perpendicular to both the $z$-axis and the $k$ vector, as it should be from Eq.~\ref{eq:oam_from_k}, and can also be seen from the panels in Fig. \ref{fig:Rashba_textures}. This leads e.g. to $l_y=0$ along the $A \to Z$ path, where only $k_y$ is nonzero. When atomic SOC is included in the orange and blue graphs, we see the spin-splitting that results from having the spin oriented either along or opposite to the already linearly varying OAM. The unquenching of the OAM at the $Z$-point when SOC is included is also clearly visible, together with the resulting change in the slope that originates from the corresponding contribution to the dipole moment Eq.~\ref{eq:d_z_from_OAM}. This correlation can also be observed in the panel showing the center of mass $\bar{z}=\int_{\textrm{supercell}}z |\psi(k)|^2$ of the Bloch-functions, which is proportional to the dipole moment around the same reference point.

When we compare this first valence band with the third valence band, shown in the right panel of Fig.~\ref{fig:Rashba_oamvseigvalv}, we can clearly see the previously discussed issues with the purely relativistic explanation. As stated before, we can note that not only the magnitude but also the sign of the prefactor in Eq.~\ref{eq:Rashba_form} is opposite for these two bands, showcased by the size of the splitting, and by the ordering of the spin-up vs the spin-down splitted part. This is because the character of the first and third valence bands are different. The first valence band is mostly coming from Te $j_{\frac{3}{2}}$ orbitals, whereas the third valence band is predominantly $j_{\frac{1}{2}}$. This causes the orientation of the OAM and SAM to be along eachother in the first band, and opposite for the third, as shown in Fig.~\ref{fig:Rashba_textures}. This then leads to the different ordering of the spin-split bands.

There is one last very interesting feature one can notice from Fig.~\ref{fig:Rashba_textures} (c) and (f), that is, the switching of the character (and SAM, OAM orientation) of the bands, very close to the $Z$ point. This is because the crystal field breaks rotationial symmetry causing the atomic $j$ to not be a conserved quantity, i.e. there is a mixing between different atomic $j$ orbitals, which varies strongly in this very narrow region around $Z$.

All these considerations lead to a very nontrivial SAM and OAM texture of the bands as we progress through the BZ.

%Looking back to Eq.~(\ref{eq:hami}), we can identify three main mechanisms that can result in a large linear spin splitting in the band dispersion.
  %The first is large OAM unquenching effect, resulting in a significant $\bm{l}(Z)$ which couples to $\bm{\sigma}(Z)$, causing a linear variation of the dipole energy through the first term of Eq.~(\ref{eq:dipfinal}). The second is a linear variation of OAM a constant SAM, and the third -- vice versa. 
%In the topmost valence band the linear variations of the dipole and SAM are very small. This suggests that the origin of the giant Rashba-like splitting is the large linear variation of the OAM, caused by the coupling to the electric polarization through the dipole, together with the large atomic SOC which then couples to the nonzero SAM through $\lambda_{so} \bm{l} \cdot \bm{\sigma}$. In the third valence band, however, the variation of the OAM is less than that of the SAM, and the contribution due to the charge asymmetry plays a bigger role, making it hard to assign the splitting to a single contribution. 
\begin{figure*}[h]
\centering
\IncludeGraphics[width=0.49\linewidth]{OAMvsK}
\IncludeGraphics[width=0.49\linewidth]{OAMvsK2.png}
\caption{\label{fig:Rashba_oamvseigvalv}Comparison between the real-space observables and energy dispersion in (a) the first and (b) third valence band. The values are plotted in function of the relative distance from the $Z$ point $\bm{k}_r = \bm{k} - \bm{k}_Z$, towards the A and U points. The green graphs denote the values before turning on atomic SOC, whereas the orange and blue graphs denote the two spin-split bands.}
\end{figure*}

\begin{figure}
\IncludeGraphics[width=.7\columnwidth]{diffdens}
\caption{\label{Fig:diffdens}The variation of the charge density of the Bloch function of the first valence band $\left.\frac{\partial |\psi(k)|^2}{\partial k}\right\rvert_{k=Z}$ away from $Z$ towards $A$. Te and Ge ions are in red and blue, respectively. The charge asymmetry around Ge showcases the nonzero dipole moment along $z$, which couples to the local electric field near Ge ion.}
\end{figure}
\begin{figure*}[t!]
  \subfloat[Lower 3rd valence band]{
    \centering
    \IncludeGraphics[width=0.32\linewidth]{Ltexture5.png}
  }
  ~
  \subfloat[Upper 3rd valence band]{
    \centering
    \IncludeGraphics[width=0.32\linewidth]{Ltexture6.png}
  }
  ~
  \subfloat[Zoom-in of (b)]{
    \centering
    \IncludeGraphics[width=0.34\linewidth]{Ltexture6small.png}
  }\\
  ~
  \subfloat[Lower 1st valence band]{
    \centering
    \IncludeGraphics[width=0.32\linewidth]{Ltexture9.png}
  }
  ~
  \subfloat[Upper 1st valence band]{
    \centering
    \IncludeGraphics[width=0.32\linewidth]{Ltexture10.png}
  }
  ~
  \subfloat[Zoom-in of (d)]{
    \centering
    \IncludeGraphics[width=0.34\linewidth]{Ltexture9small.png}
  }
  \caption{\label{fig:Rashba_textures} OAM and SAM textures around the $Z$ point in the first and third valence bands of GeTe. The black and green arrows show the OAM and SAM textures, respectively. The length of the arrows was chosen separately for clarity in each figure and should thus not be compared. The color maps signify the energy of the bands, relative to the Fermi level. The small box around the $Z$ point indicates the area, magnified in panels (c) and (f). In the zoomed figures (c) and (f) one can observe the change or relative orientation between the SAM and OAM when moving away from the $Z$ point, signifying a change of character between $j=1/2$ and $j=3/2$.}
\end{figure*}

\section{Conclusions}

We have explored the microscopic origin of the giant Rashba-like spin splitting in the band structure of bulk ferroelectric GeTe with high atomic SOC. We derived the form of the band dispersion in the Wannier representation, that relates the large spin splitting to the intricate interplay between OAM, atomic SOC, the crystal field and the electric polarization. It turns out that the crucial component, which is not present in the relativistic Rashba effect, is the emergence of a nonzero electric dipole of the Bloch functions due to their OAM. The quantitative analysis based on Wannier functions and atomic-centered approximation confirms this mechanism in GeTe. We find a very good agreement between the proposed band dispersion, Eq.~(\ref{eq:hami}), and the dispersions of the first and third valence bands, where the effect manifests itself most clearly.

Ultimately, the results suggest that (1) large ferroelectric polarization, (2) high atomic SOC, and (3) highly symmetric environment producing little OAM quenching could be the design rules for new materials with strong Rashba-like spin splitting. These materials could enable spintronic devices with the much needed electric control of spin polarization.

\printbibliography

