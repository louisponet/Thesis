\newcommand{\Unkr}{u_n(\bm{k}, \bm{r})}
% \renewcommand{\eikr}{$e^{i\bm{k}\cdot\bm{r}}$}
\newcommand{\Eikr}{e^{i\bm{k}\cdot\bm{r}}}
\chapter{Spin-momentum locking in high spin-orbit coupled ferroelectrics \label{ch:Rashba}}
\section{Introduction}
% Bulk ferroelectrics with large atomic spin-orbit coupling allow for electric control of spin-polarized states~\cite{DiSante2013,Ishizaka2011,Kim2014, Liebmann2016, Krempasky2015}, allowing for the switching of the spin texture by an externally applied electric field. 
% The underlying mechanism is, however, not well understood.
% Note: Not sure we should really focus on technology, also it's not particularly clear it wants to use both spin and charge, rather than just spin.
% The research area of spintronics aims to supersede standard electronics by using both the spin and charge degree of freedom of the carriers for information processing and storage. %\sa{https://doi.org/10.1146/annurev-conmatphys-070909-104123}
%
The research field of spintronics aims to understand the behavior of spins inside materials, and translate this understanding into the ability to actively control these degrees of freedom for possible technological applications \cite{Joshi2016}.
Many devices have been theorized, for example, spin field-effect transistors (spin-FET)\cite{Datta1990}, and storage devices which utilize spin-current and associated spin-transfer torque to efficiently manipulate magnetic domains \cite{Kent2015}, both in ferromagnets \cite{Nunez2011}, and antiferromagnets \cite{Jungwirth2016}.
In spite of fundamental interest and potential for applications, the actual realization of these spin-FET devices has been rather elusive.
One of the main culprits for this limited success is that the envisioned devices require very granular, ideally electric, control of the spin.

This is not straightforward, since the only direct coupling between the spins of the current carriers and the electric field is through relativistic effects. When an electron moves through an electric field, this field is rather perceived as a magnetic field, in the electron's co-moving frame. This magnetic field can be written as
\begin{equation}
	\bm{B} = - \frac{\bm v \times \bm E}{c^2},
\end{equation}
where $\bm v$ denotes the velocity of the electron, $\bm E$ the electric field, and $c$ the speed of light.
The spin of the electron will interact with this magnetic field through the Zeeman energy $H_Z = -\frac{g_s\mu_B}{\hbar} \bm{\sigma}$, where $g_s \approx 2$, $\mu_B$ is the Bohr magneton and $\bm{\sigma}$ is the spin of the electron.
The spins will thus start to precess around $\bm B$ created by the applied $\bm E$, and eventually align with it.
However, in nearly all real materials, the speed $\bm v$ of the current carriers is so low, and the applied electric fields so small, that the $c^2$ in the denominator completely overpowers the numerator, leading ultimately to a tiny effect.
One class of materials for which the carrier speeds are relatively high are the topological insulators \cite{Kane2005a,Novoselov2005,CastroNeto2009,Fu2007,Fu2006,Pesin2012}.
In these materials, a peculiar band structure effect leads to Dirac cones and Dirac points. They get their name from the necessity of using the Dirac equation in order to accurately describe the relativistic behavior of electrons (and holes) at these points.
It turns out that due to the band structure peculiarities, electrons (and holes) appear to have zero effective mass leading to a large Fermi velocity, on the order of \SI{10e6}m/s$^2$ for graphene \cite{Novoselov2005}.   
However, even in this case, the influence of external electric fields on the carriers' spin is extremely limited.
Indeed, for a field strength of 1 V/nm, the size of the magnetic field $\bm B$ that results from Eq.~\ref{eq:Rashba_B} is only on the order of \SI{1e-2} T, leading to a Zeeman energy of the spins of only \SI{1e-3} meV.
This is all to say that such a direct electric control of the spins is quite infeasible.

Leveraging the properties of particular materials, on the other hand, does lead to the sought after electric control, albeit indirectly.
One class of such materials are the ferroelectric semiconductors with large atomic spin-orbit coupling (SOC) \cite{Picozzi2014,DiSante2013,Ishizaka2011,Kim2014}.
Spin-orbit coupling is essentially identical to the previously discussed effect.
The ``atomic'' denotion refers to the manifestation of this effect due to the potential (and electric fields) around the atomic cores.
The atomic potentials are much greater than those from applied external fields, and the electrons have a much higher momenta close to the ionic cores.
Consequently, the spin-orbit coupling in atoms can lead to significant consequences and is one of the main drivers of the effect discussed in this Chapter.
This will be discussed in much greater detail in the next section.

Because of the ferroelectricity, and inversion symmetry breaking, an internal electric field is generated through the dipoles associated with the polarization.
Combined with the high atomic SOC, a sizeable linear energy splitting between spin-polarized states can appear in these materials \cite{DiSante2013}, contrary to the tiny effect described above.
This $k$ dependent splitting manifests itself in the bandstructure as conical intersections surrounding time-reversal (TR) invariant points of the Brillouin Zone (BZ) (see Fig.~\ref{fig:Rashba_intro_dispersion}).
These points are special in the sense that the Kramers degeneracy \cite{Kramerstheorem} is enforced through the translational symmetry of the lattice, rather than by inversion symmetry.
If we take a spin up Bloch state at such a TR invariant point, by applying the time-reversal operator to the wavevector ($\bm k \rightarrow \bm {-k}$, $\bm{\sigma} \rightarrow -\bm{\sigma}$), it is brought to a point on the opposite side of the BZ, and can be translated back to the original point through a reciprocal lattice translation.
Since the system obeys both TR symmetry and translational symmetry, this down spin state has to have the same energy as the original up spin state, enforcing the Kramers degeneracy.
So, even though the inversion symmetry brakes the Kramers degeneracy at general $k$-points, leading to the spin splitting, it is still present at these TR invariant points, leading to the intersection of cones in Fig.~\ref{fig:Rashba_intro_dispersion}.

Due to the definite spin-polarization of these bands, current carriers \footnote{holes in the case of Fig.~\ref{fig:Rashba_intro_dispersion}} traveling through the material will tend to align their spins to this spin-polarization.
Since the sign of the electric field creating the spin-splitting is a result of the orientation of the ferroelectric polarization, it can be reversed through the application of a sufficiently strong external electric field.
The spin-polarized states have been observed both experimentally \cite{Ishizaka2011,Liebmann2016,Krempasky2015} and in {\it ab-initio} density functional theory (DFT) simulations \cite{DiSante2013, Kim2014, Picozzi2014}.
It is, however, often not well understood what the underlying microscopic mechanisms are that lead to the observed large splitting.

We start by investigating multiple contributions to the $\bm{k}$ dependent spin-splitting, and comment on their respective magnitudes and symmetry requirements. 
All lead to an energy term of the form:
\begin{equation}
	\label{eq:Rashba_form}
	H_R(\bm{k}) = \alpha_R \frac{\bm{E}}{|\bm{E}|} \cdot (\bm{k} \times \bm{\sigma}),
\end{equation}
with $\bm{\sigma}$ denoting the electron spin operator, and $\bm{E}$ the electric field.
Some of the microscopic effects that will be discussed are well known, while others are more obscure.
We will see that the more obscure ones actually dominate in most cases, and that the largest contributions stem from a combination of electrostatic and relativistic effects.

As a hallmark example of materials that demonstrate a large spin-splitting, we focus in this Chapter on Germanium Telluride (GeTe).
Its dispersion is exactly the one we used to demonstrate the conical shape of the bandstructure, displayed Fig.~\ref{fig:Rashba_intro_dispersion}.
To investigate how the microscopic effects manifest themselves in GeTe, we use DFT followed by a Wannierization in order to gain access to the local, real-space, properties of the Bloch functions.
\begin{figure}[h]
	\begin{subfigure}[b]{0.49\textwidth}
	\caption{}
	\IncludeGraphics{intro_dispersion_svg.png}
	\end{subfigure}
	\begin{subfigure}[b]{0.49\textwidth}
	\caption{}
	\IncludeGraphics{cone.png}
	\end{subfigure}
	\caption{\label{fig:Rashba_intro_dispersion}
		{\bf Large Rashba splitting} a) The band dispersion of the first valence band in GeTe around the $Z$-point of the Brillouin zone (see Fig.~\ref{fig:Rashba_bands_dos} for details). The blue and red graphs shows the non-relativistic (NSOC) and relativistic (SOC) bandstructure, respectively. The $u-d$ and $u'-d'$ labels designate the up and down spin-polarized bands, where the prime signifies that the orientation of the spin quantization axis depends on $\bm{k}$. b) Intersecting conical energy surface due to the spin-splitting. The spin texture is indicated by the arrows. This chiral spin-texture is what leads to the changing up-down spin axis.}
\end{figure}

We finish with a conclusion, summarizing the observations and make the claim that the linear spin-splitting attributed to the relativistic Rashba effect is almost exclusively a result of different effects. 

\section{Rashba-Bychkov Effect \label{sec:Rashba_relativistic}}
%############################ NEW ############################
Before turning to the less well known effects, we summarize the main points of the original Rashba-Bychkov effect, first derived in their seminal 1959 paper \cite{Rashba1959SymmetryAr}.
It is a purely relativistic effect that can be derived from a second order expansion of the Dirac-equation in small parameter $1/c$:
\begin{equation}
	\label{eq:Rashba_dirac}
	H \Ket{\psi} = \left[\frac{\bm{p}^2}{2m} - e V - \frac{e \hbar}{4m^2c^2}(\bm{\sigma}\cdot[\bm{\nabla}V \times \bm{p}]) + \frac{\bm{p}^2}{8m^2c^2} V - \frac{\bm{p}^4}{8m^3c^2}\right]\Ket{\psi} = E\Ket{\psi}
\end{equation}
where $\Ket{\psi}$ is a two component spinor, $V$ denotes the electric potential, $\bm{\sigma}$ a vector of Pauli-matrices $(\sigma_x, \sigma_y, \sigma_z)$, $m$ and $e$ the electron mass and charge, respectively.
$\bm \nabla = (\frac{\partial}{\partial x}, \frac{\partial}{\partial y}, \frac{\partial }{\partial z})$ denotes spatial derivatives, and $\bm{p} = -i\hbar \bm{\nabla}$ the canonical momentum.
The first two terms are the non-relativistic part of the Hamiltonian, the third one represents the SOC, the fourth is known as the Darwin effect and the fifth is the relativistic correction to the effective electron mass. 
As is common in literature, we introduce the spin-orbit coupling constant $\lambda = \frac{e \hbar}{4m^2c^2}$.
The potential $V(\bm r)$ is periodic in a crystal, leading to eigenstates given by the Bloch wavefunctions $\BlochKetr{n} = \eikr{r} \Ket{\unkr{n}{k}}$, where $\Ket{\unkr{n}{k}}$ denotes the cell-periodic part, $\eikr{r}$ the envelope function, and $n$ the band index.
In the following we drop $\bm{r}$ in the wavefunctions, and inner products correspond to an integration over all space:

\begin{equation}
	\Sandwich{\phi_n^{\bm{k}}}{\psi_m^{\bm{k}'}} = \int d\bm{r} e^{-i (\bm k - \bm k') \cdot \bm r} u_n^{\bm k }(\bm r)^* u_m^{\bm k'}(\bm r).
\end{equation}

To obtain the eigenvalue equation for $\Ket{\unk{n}{k}}$, we insert $\BlochKet{n}$ in Eq.~\ref{eq:Rashba_dirac}, and carry out the differentiation
\begin{equation}
	\label{eq:Rashba_momentum}
\bm{p} \, \Eikr\Ket{\unk{n}{k}} = -i \hbar \bm \nabla \Eikr\Ket{\unk{n}{k}} = \Eikr(\bm p + \hbar \bm k)\Ket{\unk{n}{k}},
\end{equation}
and similarly substitute $\bm{p}^2 \rightarrow (\bm{p}+\hbar \bm{k})^2$.
This leads to the following equations for $\Ket{\unk{n}{k}}$:
\begin{align}
	\label{eq:Rashba_unk_Vs}
	E_n \Ket{\unk{n}{k}} =& \left(V_0 + V_1  +  V_2 + V_3 \right) \Ket{\unk{n}{k}} \\
	V_0^{\bm{k}} =& \frac{p^2}{2m} - eV + \frac{\hbar^2 k^2}{2m} \\
	V_1^{\bm{k}} =& \hbar\frac{\bm{k}\cdot\bm{p}}{m} \\
	V_2^{\bm{k}} =& -\lambda \bm{\sigma} \cdot ( \bm{\nabla}V \times \bm{k}) \\
	V_3^{\bm{k}} =& -\lambda \bm{\sigma} \cdot ( \bm{\nabla}V \times \bm{p}).
\end{align}
We neglected the last two terms of Eq.~\ref{eq:Rashba_dirac} since they are exceedingly small and do not contribute to the linear $k$-dependent energy of the form in Eq.~\ref{eq:Rashba_form}.

Before turning to a $\bm{k} \cdot \bm{p}$ expansion of the above equations, we take a moment to address how $V^{\bm k}_2$ and $V^{\bm k}_3$ act on the Bloch functions, more specifically, which contribution to the electric field $\bm{\nabla}V$ is important.
Both terms originate from the application of Eq.~\ref{eq:Rashba_momentum} to the third contribution in Eq.~\ref{eq:Rashba_dirac}.
We study the case of a 2D square lattice, for a Bloch function with $\bm k = (k_x, 0)$, and with the polarization oriented along the $y$-direction.
A pictorial depiction of this situation is shown in Fig.~\ref{fig:Efield_cell_drawing}.
As is shown there, we separate the electric field into two contributions:
\begin{equation}
	\bm{\nabla} V = \bm{E} = \bm{E}_{\rm a} + \bm{E}_{\rm P},
\end{equation}
with $E_{\rm a}$ resulting from the atomic potentials and $E_{\rm P}$ the field associated with the polarization\footnote{We have assumed that no external fields are applied, these can be included trivially by summing them to the internal field.}.
\begin{figure}[h]
~\centering
\IncludeGraphics[width=\linewidth]{Rashba_potentials.png}\caption{\label{fig:Efield_cell_drawing}{\bf Potentials and wavefunctions inside the unit cell.} Pictorial representation of the different electric potentials ($V$, surfaces) and fields ($E$, arrows) in a 2D square lattice. The blue spheres represent the atoms, which produce the spherically symmetric red potential $V_a$ and field $E_a$. The purple plane denotes the potential $V_p$ with resulting field $E_p$, associated with the polarization. The green and yellow surfaces depict the cell periodic and envelope part of the Bloch function $u_n(k) e^{ikx}$, respectively. $k$ and $p_u$ denote the crystalline and canonical momentum of the Bloch function.
The former results from the envelope function, and the latter from the periodic part, see Eq.~\ref{eq:Rashba_momentum}.}
\end{figure}

Focusing first on $V^{\bm k}_2$:
\begin{equation}
	V^{\bm k}_2 = -\lambda \bm{\sigma} \cdot \left[ (\bm E_a + \bm E_P) \times \bm{k}\right],
\end{equation}
we can see from the figure that the contribution of the atomic potential, $\bm E_a$, is zero since it is odd throughout the unit cell, whereas $k$ is a constant.
On the contrary, $\bm E_P$ is uniform throughout the unit cell, leading to a nonzero contribution.
However, both $\bm E_P$ and $k$ are generally small, meaning that $V^{\bm k}_2$ is also exceedingly small due to the tiny spin-orbit coupling constant $\lambda$ (usually $V^{\bm k}_2 \sim$ \SI{e-6}eV).

Following a similar train of thought for $V^{\bm k}_3$:
\begin{equation}
	V^{\bm k}_3 = -\lambda \bm{\sigma} \cdot \left[ (\bm E_a + \bm E_P) \times \bm{p}\right],
\end{equation}
we find that $\bm p$ applied to the periodic $\Ket{\unk{n}{k}}$ part of the Bloch function is odd throughout the unit cell, and the contribution from $\bm E_P$ is zero.
The shape of the atomic potential is such that $\bm E_a$ is also odd throughout the unit cell so that its contribution to $V^{\bm k}_3$ does not vanish.
Moreover, due to the size of the atomic potential, $\bm E_a \gg \bm E_P$, and similarly $\bm p\,\Ket{\unk{n}{k}} \gg \bm k$, meaning that $V^{\bm k}_3$ tends to contribute much more to the total energy than $V^{\bm k}_2$.

As it turns out, $V^{\bm k}_3$ is exactly the energy term that describes the atomic SOC, which is known to not be negligible for heavier ions.
This term by itself does not lead any linear spin-splitting, however.
We saw that $V_2^{\bm k}$ does have the correct form of Eq.~\ref{eq:Rashba_form}, and we therefore identify it as the first contribution to the purely relativistic Rashba spin-splitting.
\\\\
In order to obtain the second contribution, we perform a $\bm k \cdot \bm p$ expansion of Eq.~\ref{eq:Rashba_unk_Vs} around a TR invariant point of the BZ.
Without loss of generality we choose this point to be the gamma point: $\bm{k}_\Gamma = (0, 0)$.
As we mentioned in the introduction, the spin up and spin down bands are necessarily degenerate at such a TR invariant point, and we take $E^{\sigma}_n(\Gamma) = 0$.
We furthermore denote the periodic parts of the corresponding Bloch functions as $\Ket{u_n, \sigma}$, where we have omitted $\bm k$ to distinguish the $\Ket{u_n}$ at $\Gamma$ from the $\Ket{\unk{n}{k}}$ at a general $k$-point.

% The Kramers degeneracy of the up and down spin states is broken due to the loss of inversion symmetry and inclusion of the SOC terms ($V_2$,$V_3$), this degeneracy will be lifted for $\bm{k}$-points which are not TR invariant.
% It is important to realize that the orientation of spin axis of the eigenstates of Eq.~\ref{eq:Rashba_unk_Vs} depends on the direction of both $\bm{k}$ and $\bm{P}$, as will become clear later.
% This means that for each $\bm{k}$ point, the actual orientation of the up and down spins varies.
Expanding then Eq.~\ref{eq:Rashba_unk_Vs} in small $\bm{k}$, while keeping terms linear in $\bm{k}$ and up to second order in $1/c$, we find:
\begin{align}
	\label{eq:kp_expansion}
	E_n^{\sigma}(\bm{k}) =& -\lambda \bm{\sigma} \cdot ( \bm{\nabla}V \times \bm{k}) + \\
		& \sum_{m \neq n}\frac{\hbar}{m}\frac{\Bra{u_n,\sigma} \bm k \cdot \bm p \Ket{u_m, \sigma}\Bra{u_m, \sigma} V_3^{\bm{k}} \Ket{u_n, \sigma} + h.c.}{E_n^{\sigma} - E_m^{\sigma}}.
\end{align}
The first term at the right hand side is identical to the earlier discussed $V^{\bm k}_2$.
We know that the atomic SOC ($V^{\bm k}_3$) is much larger than $V^{\bm{k}}_2$, leading to the same order of magnitude for both terms, even though the second is of a higher order in the perturbation theory.

A requirement for the second term to be nonzero is that $u_n$ and $u_m$ have different parity along the spatial direction defined by $\bm{k}$, since otherwise $\Bra{u_n} k_x \partial_x + k_y \partial_y + k_z \partial_z \Ket{u_m} = 0$.
One example could be a $p_y$ orbital and a $s$-$p_z$ hybridized one, which would be created by an electric polarization along the $z$-axis.
How exactly this hybridization arises will be discussed in further detail below, in the context of a Tight Binding model.
The combination of the two linear-in-$\bm{k}$ terms in Eq.~\ref{eq:kp_expansion} can be regarded as the fully relativistic Rashba-effect.

In summary, we have given the derivation of the relativistic Rashba-effect starting from the Dirac equation applied to the Bloch functions, followed by a $\bm k \cdot \bm p$ expansion.
By carefully considering the microscopics we have found that both contributions have a similar order of magnitudeof typically $10^{-6}$ eV.

This leads us to the conclusion that the purely relativistic effect can not explain the observed spin-splitting which is around 0.3 eV in GeTe, especially since this splitting appears in the bulk band structure where no large interface contributions to the electric field play a role.

In the remainder of this Chapter, we will therefore focus on a different origin to the effect, one that lies in the combination of electrostatic effects with the strong atomic SOC.
These electrostatics are a result of the creation of dipoles by Bloch functions with nonzero orbital angular momentum, which we describe in the following section. 
\lp{What about nagaosa?}
\section{Orbital Rashba Effect}
Until relatively recently, the exact role that the orbital angular momentum (OAM) plays in the Rashba-like linearspin-splitting was not well understood. Even up until now there seems to be some confusion on how the many different mechanisms conspire.   
In an attempt to alleviate this confusion we will try to disentangle them and give a pedagogical description based on the works in Refs.~\cite{Petersen2000,Park2011,Park2012,Park2015,Go2016}.

As mentioned before, Bloch functions at non TR invariant points can have nonzero OAM.
Through the overlap of the cell periodic parts, this generates inter-cell dipoles that in turn interact with any electric field such as the one associated with the polarization.
The inverse effect then results in the generation of Bloch functions with nonzero OAM.
This OAM then couples to the spin of the electrons throught the atomic SOC, leading to a splitting with the size determined (in part) by the weight of the ion around which the OAM develops.
We will try to figure out which mechanisms can cause this OAM, when, and how much each of them contributes.

One of the reasons why OAM is often overlooked in crystals is that, ignoring atomic SOC, it is usually completely quenched.
This is because OAM is associated with a particular charge distribution that interacts with the ones surrounding the neighboring ions.
In most cases the most favorable charge distribution leads to orbitals with zero OAM, like the $p_x$, $p_y$ and $p_z$ orbitals in the case of $L=1$, where $\hat{L} = \hat{\bm r} \times \hat{\bm p}$ denotes the OAM operator.
This process is sometimes referred to as the ``quenching'' of OAM, so called because in isolated atoms the atomic SOC generates orbitals with maximum OAM to maximize the energy gain through $\hat{H}_{\rm SOC} = \lambda \hat{\bm \sigma} \cdot \hat{\bm L}$.

There are thus two mechanisms that lead to the unquenching of OAM: the atomic SOC, and the electrostatic interaction with the electric field through the associated dipoles.
We will see that the former leads to a small OAM even at TR invariant points, while the latter leads to a linearly varying chiral OAM as $k$ is varied away from the TR invariant point.
This chirality of the OAM turns out to be very similar to the chirality of the spin in the purely relativistic Rashba terms (see Fig.~\ref{fig:Rashba_intro_dispersion}).
Both will be shown to result in a linear contribution to the energy dispersion, the slope of which depends on the orientation of the electron's spin relative to the OAM\footnote{Positive for parallel orientation of the spin and OAM and negative for antiparallel orientation.}.

We now proceed by giving a pedagogical derivation of these mechanisms based on a tight-binding model \cite{Petersen2000,Kim2014,Go2016}. This is an extension on our earlier work in Ref.~\cite{Ponet2018}.

\subsection{Tight-Binding model}
The tight-binding model is defined on a 2D square lattice with lattice parameter $a$, one atom per unit cell, and four Wannier orbitals centered on that atom.
They resemble the angular character of an $s$-orbital and three $p$-orbitals, and are thus written as: $\Ket{s^{\bm{n}}}$, $\Ket{x^{\bm{n}}}$, $\Ket{y^{\bm{n}}}$, $\Ket{z^{\bm{n}}}$, where $\bm{n}$ denotes the unit cell indices ($n_x$,$n_y$) to which the Wannier orbital belongs to.
To simplify notation, we omit $\bm{n} = (0, 0)$ in writing the WFs of the central unit cell.
We furthermore assume that these orbitals have a Gaussian radial shape $\Sandwich{\bm{r}}{s_{\bm{n}}} = e^{-\left(\frac{|\bm{r}-\bm{n}a|}{a_0}\right)}$, $\Sandwich{\bm{r}}{\alpha_{\bm{n}}} = \alpha e^{-\left(\frac{|\bm{r}-\bm{n}a|}{a_0}\right)^2}$ with $\alpha = x, y, z$.
The reason for choosing Gaussians is to make solving the overlap integrals easier.
Using any other radial shape does not lead to any qualitative changes.
The bare tight-binding Hamiltonian is denoted as $\hat{H}_0$ and includes the usual hopping parameters due to overlap $t_{\alpha\beta}^{\bm{n}_1\bm{n}_2} = \Bra{\alpha^{\bm{n}_1}}\frac{\hat{\bm{p}}^2}{2m} + \hat{V}\Ket{\beta^{\bm{n}_2}}$.
To mimick the inversion symmetry breaking in ferroelectric materials, i.e. with a polar space group, an electric field is applied perpendicular to the layer, the $z$-direction in this case.
This allows extra hopping terms associated with $\hat{H}_{isb} = e (\hat{\bm{d}}\cdot \bm{E})$, with $\hat{\bm{d}}$ the electric dipole moment:
\begin{align}
	\label{eq:dipole}
	\Bra{s} \hat{H}_{isb} \Ket{z} &= 2 e E_z \theta_z^0,\\
	\Bra{z} \hat{H}_{isb} \Ket{x^{\bm{n}}} &= e E_z \theta_z^n n_x,\\
	\Bra{z} \hat{H}_{isb} \Ket{y^{\bm{n}}} &= e E_z \theta_z^n n_y,
	% \hat{d}_z(\bm{n}) =& -ae^{-\frac{1}{2}\left(\frac{a|\bm{n}|}{a_0}\right)^2}\frac{\pi^{\frac{3}{2}}}{16\sqrt{2}}\left(\begin{matrix}0&0&0&2\\\\0&0&0&n_x\\\\0&0&0&n_y\\\\2&n_x&n_y&0\end{matrix}\right)\\
	% =&\theta_z^n\hat{d}^1_z(\bm{n}),
\end{align}
with $\theta_z^n = -ae^{-\frac{1}{2}\left(\frac{a|\bm{n}|}{a_0}\right)^2}\frac{\pi^{\frac{3}{2}}}{16\sqrt{2}}$, other terms of $\hat{H}_{isb}$ are zero.
Fig.~\ref{fig:Rashba_overlapdip} shows pictorially how these terms arise from the electric dipoles between the shifted orbitals. 
\begin{figure}[t]
~\centering
\IncludeGraphics[width=\linewidth]{overlapdip.png}\caption{\label{fig:Rashba_overlapdip} {\bf Overlap dipoles.} (a) On-site dipole from $\Ket{s}$ and $\Ket{z}$ hybridization, (b) Dipole due to overlap of shifted $\Ket{p}$ orbitals, the dashed line signifies a unit cell boundary.}
\end{figure}
The internal field $E_z$ caused by the electric polarization is usually small, warranting a perturbative approach where $\hat{H}_{isb}$ is the perturbation on $\hat{H}_0$.
This leads to the first type of hybridization, i.e. between the $\Ket{s}$ and $\Ket{z}$ orbitals in the central unit cell:
\begin{align}
	\Ket{\tilde{z}} &= \Ket{z} + \frac{\Bra{s} 2 e E_z \theta^0_z\Ket{z}}{\varepsilon_z - \varepsilon_s}\Ket{s},\\
	\Ket{\tilde{s}}   &= \Ket{s} + \frac{\Bra{z} 2 e E_z \theta^0_z\Ket{s}}{\varepsilon_s - \varepsilon_z}\Ket{z},
\end{align}
where $\varepsilon_s = \Bra{s} \hat{H}_0 \Ket{s}$ and $\varepsilon_z = \Bra{z} \hat{H}_0 \Ket{z}$.

We can then write the kinetic energy part of $\hat{H}_0$ in terms of this hybrid $\Ket{\tilde{z}}$ orbital and the shifted $\Ket{x^{\bm n}}$, $\Ket{y^{\bm y}}$ orbitals, leading to:
\begin{align}
	\Bra{\tilde{z}} \hat{H}_0 \Ket{x^{\bm{n}}} &= \frac{2 e E_z \theta_z^n}{\varepsilon_z - \varepsilon_s}\Bra{s} \frac{-\bm{\nabla}^2}{2} \Ket{x^{\bm{n}}}\\
	&= \frac{4 e E_z \theta_z^0 \theta_z^n}{\varepsilon_z - \varepsilon_s} n_x (-5 + a^2 |\bm{n}|^2),\\
	\Bra{\tilde{z}} \hat{H}_0 \Ket{y^{\bm{n}}} &= \frac{4 e E_z\theta_z^0 \theta_z^n}{\varepsilon_z - \varepsilon_s} n_y (-5 + a^2 |\bm{n}|^2).
\end{align}
To construct $\hat{H}_0^{\bm{k}}$ and $\hat{H}_{isb}^{\bm{k}}$, we first fourier transform the WFs following Eq.~\ref{eq:Theory_wantok}:
\begin{equation}
	\label{eq:Rashba_wantok}
	\Ket{\alpha^{\bm{k}}} = \frac{1}{\sqrt{N}}\sum_{\bm{n}} e^{i \bm{k}\cdot \bm{n}}\Ket{\alpha^{\bm{n}}},
\end{equation}
with $\Ket{\alpha}$ one of the four orbitals, $\bm{k}$ written in terms of crystalline coordinates ($\frac{2\pi}{a}$), and $N$ denoting the total amount of unit cells in the material.
Filling these functions into the expression for the total Hamiltonian, similar to what was done in Eq.~\ref{eq:Theory_waninter}, results in:
\begin{align}
	\label{eq:Rashba_wantokH}
	\hat{H}_0^{\bm{k}} + \hat{H}_{isb}^{\bm{k}} &= \sum_{\bm{n}} \eikr{n} \left(\hat{H}_0^{\bm{n}} + \hat{H}_{isb}^{\bm{n}}\right),%\\
	% &= \sum_{\bm{n}} i \sin(\bm{k}\cdot \bm{n})(\hat{H}_0(\bm{n}) + \hat{H}_{isb}(\bm{n})) + k\mathrm{-even\,terms},
\end{align}
% keeping only the $\sin(\bm{k}\cdot\bm{n})$ part of the exponent since the $\cos(\bm{k}\cdot\bm{n})$ part does not result in linear-in-$k$ terms, in the small $\bm{k}$ expansion below we keep only these linear-in-$k$ terms.

If we then assume that at $k=0$ the Bloch functions are formed from $p_x$, $p_y$, $\tilde{p}_z$ and $\tilde{s}$ orbitals through Eq.~\ref{eq:Rashba_wantok}, a perturbation theory for small deviations $\bm{k}$ away from zero can be formulated as ($\bm{k}=0$ superscripts are omitted for simplicity):
\begin{equation}
	\Ket{\alpha^{\bm{k}}}=\Ket{\alpha} + \sum_{\beta \neq \alpha} \frac{\Bra{\beta} \hat{H}_0^{\bm{k}} + \hat{H}_{isb}^{\bm{k}} \Ket{\alpha}}{\varepsilon_{\alpha} - \varepsilon_{\beta}} \Ket{\beta}.
\end{equation}

Gathering the linear-in-$k$ terms from the expansion of $\eikr{n}$ in Eq.~\ref{eq:Rashba_wantokH}, and assuming $\varepsilon_p = \Bra{x} \hat{H}_0 \Ket{x} = \Bra{y} \hat{H}_0 \Ket{y}$ at $\bm{k}=0$, we find the second source of hybridization:
\begin{align}
	\Ket{x^{\bm{k}}} &= \Ket{x} + i \Theta \frac{k_x}{\varepsilon_p - \varepsilon_{\tilde{z}}}\Ket{\tilde{z}}, \\
	\Ket{y^{\bm{k}}} &= \Ket{y} + i \Theta \frac{k_y}{\varepsilon_p - \varepsilon_{\tilde{z}}}\Ket{\tilde{z}},\\
	\Ket{\tilde{z}^{\bm{k}}} &= \Ket{\tilde{z}} + i \Theta \frac{1}{\varepsilon_{\tilde{z}} - \varepsilon_p} \left(k_x\Ket{x} + k_y\Ket{y}\right),
\end{align}
with $\Theta = \frac{\pi^{5/2}}{256 a^3 }\left(-16\sqrt{2} + \frac{3a\pi^{3/2}}{\varepsilon_z - \varepsilon_s}\right) e E_z$ \lp{verify!}.
Then, using the definition of the OAM operators for $p$-orbitals:
\begin{equation}
	\hat{L}_x =-i\hbar\left(\begin{matrix}0&0&0\\0&0&1\\0&-1&0\end{matrix}\right), \hat{L}_y = -i\hbar \left(\begin{matrix}0&0&-1\\0&0&0\\1&0&0\end{matrix}\right), \hat{L}_z =-i\hbar\left(\begin{matrix}0&1&0\\-1&0&0\\0&0&0\end{matrix}\right),
\end{equation}
we find that,
\begin{align}
	\label{eq:oam_from_k}
	\Bra{x^{\bm{k}}} \hat{L}_y \Ket{x^{\bm{k}}} &= -2 \Theta \frac{k_x}{\varepsilon_p - \varepsilon_{\tilde{z}}},\\
	\Bra{y^{\bm{k}}} \hat{L}_x \Ket{y^{\bm{k}}} &= 2 \Theta \frac{k_y}{\varepsilon_p - \varepsilon_{\tilde{z}}},\\
	\Bra{\tilde{z}^{\bm{k}}} \hat{L}_x \Ket{\tilde{z}^{\bm{k}}} &= - 2 \Theta \frac{k_y}{\varepsilon_{\tilde{z}} - \varepsilon_p},\\ 
	\Bra{\tilde{z}^{\bm{k}}} \hat{L}_y \Ket{\tilde{z}^{\bm{k}}} &= 2 \Theta \frac{k_x}{\varepsilon_{\tilde{z}}-\varepsilon_p}.
\end{align}
These expressions for $\hat{\bm{L}}$ can be filled into the expression for the atomic SOC $\hat{H}_{soc}= \lambda \hat{\bm{L}} \cdot \hat{\bm{\sigma}}$ to find the energy for $\Ket{\tilde{z}^{\bm{k}}}$:
\begin{equation}
	\label{eq:Rashba_from_OAM}
	\varepsilon^{\bm{k}} = \frac{2 \lambda \Theta}{\varepsilon_{\tilde{z}}-\varepsilon_p}(\bm{k} \times \bm{\sigma}).
\end{equation}
which has the chiral form of Eq.~\ref{eq:Rashba_form}. Similar terms can be found for the $\Ket{x^{\bm k}}$ and $\Ket{y^{\bm k}}$ Bloch functions.
We want to emphasize here again that the only influence of the initial choice of the Gaussian radial shape is reflected in the prefactor in front of $e E_z$ in the definition of $\Theta$.

From this qualitative derivation, it is clear that the main reason behind the orbital Rashba effect can be traced back to the observation that Bloch functions with nonzero OAM have electric dipoles that couple to the inversion symmetry breaking electric field.
\\\\
In summary, there are two sources causing this behavior.
The first one came from the direct overlap dipoles between $p_x$ and $p_y$ orbitals with the $p_z$ orbital (panel (b) in Fig.~\ref{fig:Rashba_overlapdip})\cite{Petersen2000}.
An exaggerated demonstration for the case of $p_x + ip_z$ orbitals is shown in Fig.~\ref{fig:Rashba_interference}.
\begin{figure}[t!]
\IncludeGraphics[width=\linewidth]{interference.png}
\caption{\label{fig:Rashba_interference}{\bf Interference between orbitals with nonzero OAM.} Three neighboring unit cells are displayed, each with the same $p_x + ip_z$ orbital (thus having nonzero $l_y$). The wave functions of the left and right unit cells have their phase rotated by the plane-wave part $e^{i k_x R_x}$. The amplitude and phase of the wave function are encoded with the length and polar angle of the arrows. It is clear from the resulting phases (red arrows) that there exists constructive interference on the bottom half of the material, and destructive interference at the top half, which leads to the intercell dipoles. This figure was taken from Ref.~\cite{Ponet2018}.}
\end{figure}
The second was due to the hybridization of the $s$ and $p_z$ orbitals (panel (a) in Fig.~\ref{fig:Rashba_overlapdip}), and the kinetic energy term between the $s$ and neighboring $p_x$ and $p_y$ orbitals \cite{Go2016}.
These two terms are reflected by the two terms that contribute to the sum in $\Theta$, and lead to a chiral texture of the OAM that varies linearly with $k$ .
Through the atomic SOC, this linear-in-$k$ OAM then couples to the spin which results in the final splitting with the Rashba-like form of Eq.~\ref{eq:Rashba_from_OAM}.
\\\\
There is one other term that contributes to the linear variation of OAM with $\bm{k}$, which is present due to the previously alluded to unquenching of the OAM at the TR invariant $k$-point (e.g. $|k|=0$) due to atomic SOC~\cite{Park2011,Park2012,Park2015}.
If the Bloch functions at $\bm{k}=0$ are written as a linear combination of WFs that have the angular character of $p$-orbitals
\begin{equation}
\Ket{\psi} = \sum_{\bm{n},\alpha} c_\alpha \Ket{\alpha^{\bm{n}}},
\end{equation}
the general formula for the OAM operators in terms of these $c_\alpha$ then becomes
\begin{equation}
	\hat{L}_{\gamma}= i \epsilon_{\alpha \beta \gamma} c^*_\alpha c_\beta,
\end{equation}
where $\alpha,\beta,\gamma$ designate $x,y,z$.

Thus, through the gain on $\hat{H}_{\rm soc}$ if the total angular momentum $\bm{j} = \bm{l} + \bm{\sigma}$ is nonzero, the orbitals at the high-symmetry point will now be a linear combination of what were orginally pure $p$-orbitals (due to the quenching we assumed that the bands at $\bm{k}=0$ had a single $c_\alpha \neq 0$ in the previous derivation).
This unquenching will not be complete in the sense that it won't create orbitals with maximal $\bm{j}$ as is the case for an isolated ion through the Hund's rules, since in a crystal the neighboring charge will limit the creation of OAM\footnote{This is the reason why without SOC the OAM is fully quenched.}.

Similar to the above derivation, a small-$k$ expansion for Bloch functions $\Ket{\psi^{\bm{k}}}$ with nonzero OAM can be performed around $|k|=0$ leading in general to:
\begin{align}
	\Ket{\psi^{\bm{k}}} =& \sum_{\bm{n},\alpha} c_\alpha^{\bm{k}} \eikr{n} \Ket{\alpha^{\bm{n}}}\\
	=& \sum_{\bm{n},\alpha} \left(c_\alpha + \bm{k}\left.\frac{\partial c_\alpha^{\bm{k}}}{\partial_{\bm{k}}}\right\rvert_{\bm{k}=0} \right)\left(1 + i\bm{k}\cdot\bm{n}\right)\Ket{\alpha^{\bm{n}}}.
\end{align}
Again we omit terms with $\bm{k}=0$ for clarity.
Focusing on the terms that vary linearly with $k$, the second term in the $c_{\alpha}$ expansion together with the first term in the exponent expansion is exactly the contribution that was discussed before.
The other term for the Orbital Rashba effect, due to unquenching, originates from combining the first term in the $c_\alpha$ expansion with the second in the exponent expansion, leading to the contribution
\begin{align}
	\label{eq:d_z_from_OAM}
	\varepsilon^{\bm{k}} = \Bra{\psi^{\bm{k}}} \hat{H}_{isb} \Ket{\psi^{\bm{k}}} =&  i\sum_{\bm{n},\alpha,\beta} c_\alpha^* c_\beta \bm{k}\cdot\bm{n}\Bra{\alpha} \hat{H}_{isb} \Ket{\beta^{\bm{n}}}\\
	=&\frac{-\pi^{5/2}}{8\sqrt{2}a^3}i(c^*_xc_zk_x + c^*_yc_z k_y) e E_z  \\
	=&\frac{\pi^{5/2}}{8\sqrt{2}a^3}(L_y k_x - L_x k_y) e E_z.
\end{align}
These expressions show us that, due to ${\hat{H}_{soc} = \lambda \hat{\bm{L}}\cdot\hat{\bm{\sigma}}}$ and $\hat{H}_{isb} = e E_z d_z$, an additional Rashba-like term appears in the energy dispersion, due to the creation of orbitals with nonzero $L_y$ and $L_x$ at the high-symmetry point.

With the understanding that one can find Rashba like dispersions, coming not from the usually considered purely relativistic, but also from electrostatic mechanisms, we now look at a concrete example that behaves very similar to the above toy model, GeTe.

\section{Germanium Telluride}
\begin{figure}[h]
\IncludeGraphics[width=\linewidth]{crystal.png}
\caption{\label{fig:Rashba_crystal}{\bf Crystal structure of GeTe.}~a) Rhombohedral unit cell, with the polarization along the [111] direction in yellow. b) First Brillouin zone with the high-symmetry $k$-path that is under focus indicated by the blue lines.}
\end{figure}
At high temperatures GeTe has the standard rocksalt structure.
When temperatures are lowered below $T_c \approx 720$~K \cite{DiSante2013}, a displacive phonon instability occurs, freezing in a $\bm{q} = 0$ phonon that shifts the central Te ion along the [111] direction \cite{Rabe1987}.
The space group after this inversion symmetry breaking phase transition becomes R$3m$ (\#160 in International Tables), and a nonzero electric polarization along the $z$-direction develops as indicated by the yellow arrow in Fig.~\ref{fig:Rashba_crystal}(a). The first BZ for this low temperature structure is shown in panel (b), with the high-symmetry path under focus indicated by the blue line.


\begin{figure}[b!]
	\begin{subfigure}[b]{0.49\textwidth}
	\caption{Non-relativistic}
	\IncludeGraphics{NSOC_dos.png}
	\end{subfigure}
	\begin{subfigure}[b]{0.49\textwidth}
	\caption{Fully-relativistic}
	\IncludeGraphics{SOC_dos.png}
	\end{subfigure}
\caption{\label{fig:Rashba_bands_dos}{\bf Bandstructure of GeTe.} In both panels the bandstructure was colored according to the contribution of the constituent orbitals, as indicated in the flanking density of state plots. The bands that have larger $s$ character are situated 5 eV below the shown window.}
\end{figure}
In the calculations reported below, we used a unit cell with lattice parameter $a=$4.28~\AA, leading to a volume of 53.4~\AA$^3$.
This makes the unit cell slightly larger than the experimental one, which has a volume of 53.3~\AA$^3$~\cite{Serebryanaya1995}.
The Te ion is shifted by 0.25~\AA~ away from the center of the unit cell.

The bands around the Fermi level are formed mostly by $s$- and $p$-orbitals from the Te and Ge ions.
Due to the inversion symmetry breaking, a linear spin-splitting is observed around the $Z$-point of the BZ in the plane perpendicular to the [111] direction.
The $Z-\Gamma$ path does not show any splitting because the electric field is along the $z$-direction, and thus only a variation of $k_x$ and $k_y$ will show a linear splitting, as previously discussed.

This means that from the point of view of our earlier derivation using the tight-binding model in terms of $s$- and $p$-orbitals, it is the perfect test case.
Moreover, since it is a bulk material, we can neglect the contributions from to the relativistic Rashba effects described in Sec.~\ref{sec:Rashba_relativistic}, due to the small potential gradients $\bm{\nabla}V$ resulting from the bulk ferroelectricity.

\section{Methods}
In the calculations presented here, we used the implementation in the {\texttt Quantum ESPRESSO} software package \cite{Giannozzi2009}.
In order to confirm the linear varying OAM even when spin-orbit is not included we performed non-relativistic, as well as fully relativistic DFT calculations.
Both were performed using the Optimized Norm-conserving Vanderbilt Pseudo Potentials \cite{Hamann2013}, and a generalized-gradient approximation with Perdew-Burke-Enzerhof parametrization for the exchange-correlation functional \cite{Perdew1996}.
Plane waves with an energy cutoff of 30 Ry were included, with a 120 Ry cutoff for the density.

The reciprocal space was sampled using a 6x6x6 Monkhorst-Pack grid \cite{Pack1977} for the self-consistent calculations, using a total energy convergence threshold of $10^{-7}$Ry.

Afterwards we used the {\texttt Wannier90} package \cite{Mostofi2014AnFunctions} to perform the Wannierization as described in the theory Chapter \ref{sec:Wannier}, using projections on hydrogenic $s$- and $p$-orbitals for both Ge and Te ions as the initial guess.
A 10x10x10 $k$-grid was used for the sampling of the BZ during the Wannierization. 

% \lp{look in the high throughput paper for a decent definition of the wannier band distance w.r.t the dft one!}

\section{Results and Discussion}

The bands we will focus on most are the three topmost valence bands which consist mostly of Te $p$-orbitals, as can be seen from Fig.~\ref{fig:Rashba_bands_dos}(b).
These demonstrate the largest spin-splitting suggesting that indeed the atomic SOC plays an important rule seen as it is larger on Te compared to Ge.
If one were to fit the dispersion of the topmost band to Eq.~\ref{eq:Rashba_form}, a large prefactor $\alpha_R\approx 30.7$~eV$\cdot$\angstrom \cite{DiSante2013} is found.
As discussed before, a purely relativistic effect would rather have a prefactor of $\alpha_R \approx 10^{-6}$~eV.

Another observation that is hard to explain by purely relativistic means, is the orientation of the spin-polarization inside the different split bands.
It was found that depending on the band, the spin is oriented such that $\alpha_R$ in Eq.~\ref{eq:Rashba_form} is either negative or positive. 
This has been confirmed experimentally \cite{Krempasky2015} and depends on the value of the total angular momentum $j$.
This can only be explained through the OAM and atomic SOC mechanism, the manifestation of which in our results will be discussed further down.

By Wannierizing the bands around the Fermi level, we can construct a tight-binding model in terms of Te and Ge $s$- and $p$-orbitals.
The WFs for the spin-up Te orbitals thus obtained are displayed in Fig.~\ref{fig:Rashba_wannierfunctions}(a), the spin-down orbitals have similar charge distributions.
The total spread of the 16 spinor WFs is 53.4~\AA$^2$~.
\begin{figure}[h]
\IncludeGraphics[width=\linewidth]{Wannierfunctions.png}
\caption{\label{fig:Rashba_wannierfunctions}{\bf Wannier functions} a) The spin-up WFs of the orbitals around the Tellurium ion. The isosurface is constructed from the norm of the wavefunctions, with the coloration denoting the sign of the real part. b) The charge density in the central unit cell of the Bloch function of the top valence band, at $k=$ (0, 0.1, 0.9) in units of \AA$^{-1}$. The orbital angular momentum, polarization, and $k$-vector are shown by the red, blue, and green arrows, respectively. In each panel, the unit cell is shown by the wireframe, and the Te and Ge ions are yellow and grey, respectively.}
\end{figure}

The interpolation of the ab-initio bandstructure by the tight-binding model in terms of the WFs is shown in Fig.~\ref{fig:Rashba_wannierization}, which demonstrates that it agrees very well with the bands in the inner window, and the valence bands in particular. 
\begin{figure}[h]
\IncludeGraphics[width=\linewidth]{wanvsdft.png}
\caption{\label{fig:Rashba_wannierization}{\bf Wannier interpolation.} Bandstructure interpolation by the tight-binding Hamiltonian in terms of the WFs for non-relativistic (a) and relativistic (b) calculations.}
\end{figure}

By diagonalizing the tight-binding model, writing the BFs in terms of WFs
\begin{equation}
	\BlochKet{n} = \sum_{\bm{R}} \eikr{R} \sum_{\alpha} c_{\alpha}^n(\bm{k}) \WanKet{\alpha}{R},
\end{equation}
we can use the contribution inside the central unit cell ($\bm{R} = \bm{0}$) to get an idea of the real space distribution of the BFs.
An example is shown in Fig.~\ref{fig:Rashba_wannierfunctions}(b) for a BF along the $Z-A$ path in the BZ.
Moreover, using this real space distribution, we can calculate the center of mass inside the unit cell and the OAM through $\hat{\bm{L}} = -\frac{i}{\hbar} (\bm{r} \times \bm{k})$, where in our calculations $\hbar = 1$.
This way of calculating the OAM is the so-called atomic centered approximation, in the discussion below we focus on the OAM around the Te ion.
From the modern theory of OAM~\cite{Thonhauser2011}~it was shown that there may be another itinerant contribution to the orbital angular momentum, which is not of importance in our description. 

\begin{figure*}[h]
\centering
\IncludeGraphics[width=0.49\linewidth]{OAMvsK}
\IncludeGraphics[width=0.49\linewidth]{OAMvsK2.png}
\caption{\label{fig:Rashba_oamvseigvalv}{\bf Bloch function properties.} Comparison between the real-space observables and energy dispersion in (a) the first and (b) third valence band. The values are plotted in function of the relative distance from the $Z$ point $\bm{k}_r = \bm{k} - \bm{k}_Z$, towards the A and U points. The green graphs denote the values before turning on atomic SOC, whereas the orange and blue graphs denote the two spin-split bands.}
\end{figure*}

The dispersion, OAM, and SAM of the first valence band are shown in the left panel of Fig.~\ref{fig:Rashba_oamvseigvalv}. 
Confirming our earlier derivation, we can see that non-zero, linearly varying OAM is formed along a $k$-path away from the high-symmetry $Z$-point.
Moreover, the OAM is perpendicular to both the $z$-axis and the $k$ vector, as it should be from Eq.~\ref{eq:oam_from_k}, and can also be seen from panel (b) in Fig. \ref{fig:Rashba_wannierfunctions}.
This leads e.g. to $l_y=0$ along the $A \to Z$ path, where only $k_y$ is nonzero.
When atomic SOC is included (the orange and blue graphs), we see the spin-splitting that results from having the spin oriented either along or opposite to the linearly varying OAM.
The unquenching of the OAM at the $Z$-point when SOC is included is also clearly visible.
This leads to a change in the slope of the linear varying OAM, as compared with the non-relativistic calculations (green graphs), that originates from the corresponding contribution to the dipole moment Eq.~\ref{eq:d_z_from_OAM}.
This correlation between OAM, SAM and dipole moment can also be observed in the top left panels of Fig.~\ref{fig:Rashba_oamvseigvalv} of showing the center of mass $\bar{z}=\int_{\textrm{supercell}}d \bm{r} z |\psi^{\bm{k}}(r)|^2$ of the BFs, which is proportional to the dipole moment around the same reference point.

Due to the existence of two atoms inside the primitive unit cell, another manifestation of this mechanism arises through the overlap of $p_x$ and $p_y$ on Te and $s$ and $p_z$ on Ge.
This leads to a dipole as can be seen from the variation of the charge density of the BFs as we move away from the $Z$-point in Fig.~\ref{fig:Rashba_diffdens}.
\begin{figure}[h]
\IncludeGraphics[width=.7\columnwidth]{diffdens}
\caption{\label{fig:Rashba_diffdens}{\bf Bloch function dipole.} The variation of the charge density of the Bloch function of the first valence band $\left.\frac{\partial |\psi(k)|^2}{\partial k}\right\rvert_{k=Z}$ away from $Z$ towards $A$. Te and Ge ions are in red and blue, respectively. The charge asymmetry around Ge showcases the nonzero dipole moment along $z$, which couples to the local electric field near Ge ion. Figure taken from Ref.~\cite{Ponet2018}}
\end{figure}

When we compare this first valence band with the properties of the third valence band, shown in the right panel of Fig.~\ref{fig:Rashba_oamvseigvalv}, we can clearly see the previously discussed issues with the purely relativistic explanation.
As stated before, we can note that not only the magnitude but also the sign of the prefactor in Eq.~\ref{eq:Rashba_form} is opposite for these two bands, showcased by the size of the splitting, and by the ordering of the spin-up vs the spin-down split bands.
This is because the character of the first and third valence bands are different, where the former is comprised mostly of Te $j_{\frac{3}{2}}$ orbitals, whereas the third valence band is predominantly $j_{\frac{1}{2}}$.
This causes the relative orientation of the OAM and SAM to be parallel in the first band, and anti-parallel for the third, as shown in Fig.~\ref{fig:Rashba_textures}.
This then leads to the different ordering of the spin-split bands and opposite sign of $\alpha_R$.

There is one last very interesting feature one can notice from Fig.~\ref{fig:Rashba_textures} (c) and (f), that is, the switching of the character (and SAM, OAM orientation) of the bands, very close to the $Z$ point.
This is because the crystal field breaks rotationial symmetry causing the atomic $j$ to not be a conserved quantity, i.e. there is a mixing between different atomic $j$ orbitals, which varies strongly in this very narrow region around $Z$.

All these considerations lead to a very nontrivial SAM and OAM texture of the bands as we progress through the BZ.


\begin{figure*}[t!]
  \subfloat[Lower 3rd valence band]{
    \centering
    \IncludeGraphics[width=0.49\linewidth]{Ltexture5.png}
  }
  ~
  \subfloat[Upper 3rd valence band]{
    \centering
    \IncludeGraphics[width=0.49\linewidth]{Ltexture6.png}
  }
  \\
  ~
  \subfloat[Lower 1st valence band]{
    \centering
    \IncludeGraphics[width=0.49\linewidth]{Ltexture9.png}
  }
  ~
  \subfloat[Upper 1st valence band]{
    \centering
    \IncludeGraphics[width=0.49\linewidth]{Ltexture10.png}
  }\\
  ~
  \subfloat[Zoom-in of (b)]{
    \centering
    \IncludeGraphics[width=0.49\linewidth]{Ltexture6small.png}
  }
  ~
  \subfloat[Zoom-in of (d)]{
    \centering
    \IncludeGraphics[width=0.49\linewidth]{Ltexture9small.png}
  }
  \caption{\label{fig:Rashba_textures}{\bf OAM and SAM in the BZ.} The textures of BFs around the $Z$ point are shown for the first and third valence bands of GeTe. The black and green arrows show the OAM and SAM textures, respectively. The length of the arrows was chosen separately for clarity in each figure and should thus not be compared. The color maps signify the energy of the bands, relative to the Fermi level. The small box around the $Z$ point indicates the area, magnified in panels (c) and (f). In the zoomed figures (c) and (f) one can observe the change or relative orientation between the SAM and OAM when moving away from the $Z$ point, signifying a change of character between $j=1/2$ and $j=3/2$. Figure taken from Ref.~\cite{Ponet2018}}
\end{figure*}

\section{Conclusions}

We have explored the microscopic origin of the giant Rashba-like spin splitting in the band structure of bulk ferroelectric GeTe with high atomic SOC. We derived the form of the band dispersion in the Wannier representation, that relates the large spin splitting to the intricate interplay between OAM, atomic SOC, the crystal field and the electric polarization. It turns out that the crucial component, which is not present in the relativistic Rashba effect, is the emergence of a nonzero electric dipole of the Bloch functions due to their OAM. The quantitative analysis based on WFs and atomic-centered approximation confirms this mechanism in GeTe. We find a very good agreement between the proposed band dispersion, Eq.~\ref{eq:Rashba_hami}, and the dispersions of the first and third valence bands, where the effect manifests itself most clearly.

Ultimately, the results suggest that (1) large ferroelectric polarization, (2) high atomic SOC, and (3) highly isotropic environment producing little OAM quenching could be the design rules for new materials with strong Rashba-like spin splitting. These materials could enable spintronic devices with the much needed electric control of spin polarization.
